%%%%%%%%%%%%%%%%%%%%%%%%%%%%%%%%%%%%%%%%%%%%%%%%%%%%%%
%   PI1 - ANÁLISIS DE RESULTADOS
%%%%%%%%%%%%%%%%%%%%%%%%%%%%%%%%%%%%%%%%%%%%%%%%%%%%%%
\subsubsection{Costos de implementación}

%Zehra et al. \cite{Zehra2023} analyze the implementation of AI in low- and middle-income countries (LMICs). They remark that, while AI systems there face similar technical challenges and issues to those of more developed countries, there are further difficulties in LMICs. The authors perceive that when labs struggle due to financial constraints to hire trained histopathologists, and when there is scarcity of trained laboratory technologists even for conventional histopathology, it would be extremely difficult to obtain funds and manpower for implementing AI and digital pathology.
Zehra et al. \cite{Zehra2023} analizan la implementación de inteligencia artificial en países de bajos y medianos ingresos (LMIC). Destacan que, a pesar de que los desafíos técnicos que deben enfrentar los sistemas IA/IAX son los similares en la mayoría de los países más desarrollados, los LMIC deben enfrentar dificultades adicionales. Los autores perciben que es extremadamente difícil obtener fondos para implementar IA y patología digital, dado que los laboratorios no solo se enfrentan a restricciones financieras para contratar histopatólogos, sino también a la escasez de estos profesionales.
%In the face of this, they propose a series of low-cost alternatives to venture into AI systems. For instance, low resource organizations can access available open-source whole-slide image archives provided by the Cancer Genome Atlas \cite{CancerGenomeAtlasWebsite}, the Cancer Imaging Archive \cite{CancerImagingArchiveWebsite} and the Digital Pathology Association’s Whole-Slide Imaging Repository \cite{DAPAWebsite}, among others. Another alternative would be when facing internet glitches and the downloading of these images provided by these repositories. As they are large in size, many of the professionals in LMICs may find it difficult to obtain the data for the AI training. To solve this issue, Pathologists can photograph a region of interest for a particular pathology by using the data of their own patients. After taking the photograph, these images can be classified and uploaded in AI systems either open-source or commercially available, although the latter are usually expensive.
Para atender este problema, los autores proponen una serie de alternativas de bajo costo para incursionar en los sistemas de IA. Por ejemplo, las organizaciones de bajos recursos pueden acceder a archivos WSI libres (open-source) provistos por el Cancer Genome Atlas \cite{CancerGenomeAtlasWebsite}, Cancer Imaging Archive \cite{CancerImagingArchiveWebsite} y el repositorio WSI de la Digital Pathology Association \cite{DAPAWebsite}, entre otros.

Ante fallas en la conexión a internet, los autores sugieren descargar las imágenes de estos repositorios. Debido a que las WSI suelen ser inmensas en tamaño, muchos profesionales en LMIC podrían tener dificultades para obtener datos de entrenamiento para los modelos de IA. Para resolver este problema, los autores recomiendan que los patólogos fotografíen la región de interés de una patología en concreta usando los datos de sus propios pacientes. Estas fotografías pueden posteriormente clasificarse y subirse a sistemas IA libres o comerciales, teniendo en cuenta que esas últimas suelen ser costosas.

Shawi et al. \cite{Shawi2022} comentan que uno de los principales obstáculos para los algoritmos de aprendizaje profundo es que requieren una gran cantidad de datos etiquetados para afinar su arquitectura y parámetros internos, que son costosos y difíciles de obtener en la práctica médica. Además, objetan que desarrollar una red neuronal con sus hiperparámetros bien sintonizados es una tarea desafiante y que insume mucho tiempo.
