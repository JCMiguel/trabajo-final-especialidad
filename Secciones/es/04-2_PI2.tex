%%%%%%%%%%%%%%%%%%%%%%%%%%%%%%%%%%%%%%%%%%%%%%%%%%%%%%
%   PI2 - ANÁLISIS DE RESULTADOS
%%%%%%%%%%%%%%%%%%%%%%%%%%%%%%%%%%%%%%%%%%%%%%%%%%%%%%
\subsection{PI2 - ¿Qué postura adoptan los autores para el uso de IA/IAX?} \label{section:es/resultados-pi2}

Se realizó una búsqueda de frases y expresiones claves para identificar la postura de los diversos autores respecto al uso de sistemas IA/IAX. Se confeccionaron dos grupos de posturas: colaborativo y no colaborativo. Se considera \enquote{colaborativo} a todas aquellas expresiones, frases o párrafos que transparenten aceptación, muestren intencionalidad de integrar la tecnología al flujo de trabajo actual de los patólogos y profesionales del área médica, que presenten a la inteligencia artificial como una herramienta o asistente, menciones paradigmas human-in-the-loop, etc. Por otro lado, se considera \enquote{no colaborativo} a todo aquello que implique un enfoque competitivo, de superioridad, rivalidad entre tecnología y humanidad, que planteen la inteligencia artificial como un sustituto al patólogo, etc.

Es importante destacar que no solo se incluyeron posturas de los autores de los estudios primarios, sino también de aquellos a quienes hayan citado en el cuerpo de cada artículo. La síntesis de estos datos permitió obtener el histograma de la Figura \ref{figura:es/rpi2}, en le que se observa una clara tendencia a considerar los sistemas IA/IAX como un complemento para la práctica diaria de los patólogos y el personal médico. Los datos en crudo con los que se confeccionó este gráfico se pueden consultar en el Apéndice \ref{appendix:es/rpi2_datos}.

%- No colaborativo --> competitivo, enfoque de superioridad, rivalidad, reemplazar al personal médico.
%- Colaborativo --> integración entre profesionales, enfoque de IA como herramienta o asistente, paradigmas human-in-the-loop
%En la tabla XXX se eliminaron las referencias bibliograficas del material original para que no se confundan con las referencias de este trabajo. ¿Debería aclarar esto?

\begin{figure}[htbp]
    \centerline{\includegraphics[width=\textwidth]{Imagenes/rpi2.png}}
    \caption{Histograma de posturas acerca del uso de la IA y su integración con los patólogos.}
    \label{figura:es/rpi2}
\end{figure}

