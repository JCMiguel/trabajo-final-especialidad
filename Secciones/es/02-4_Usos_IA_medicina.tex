%%%%%%%%%%%%%%%%%%%%%%%%%%%%%%%%%%%%%%%%%%%%%%%%%%%%%%
%   NOCIONES DE INTELIGENCIA ARTIFICIAL
%%%%%%%%%%%%%%%%%%%%%%%%%%%%%%%%%%%%%%%%%%%%%%%%%%%%%%
\subsection{Usos de inteligencia artificial en medicina} \label{section:es/uso-ia-medicina}

%- Usos de Inteligencia artificial en medicina
%	- Promesas o ventajas tentativas
%           Computational pathology is poised to revolutionize
%           digital pathology by delivering meaningful automation of
%           anatomic pathology by augmenting pathologists without
%           replacing them.
%	- Diagnóstico diferencial (90-Lu-Trustworthy AI Tuberculosis)
%	- Aprendizaje Federado (112-Tariq-Trustworthy Federated Learning Review)
%   - LLM en medicina (90-Lu-Trustworthy AI Tuberculosis) (no sé bien dónde poner esto, pero siento que debería introducirlo)

% TODO: Acá se pueden comentar (objetivamente y citando fuentes) los beneficios de incorporar xAI en la pràctica médica. Tambièn se puede hacer un listado de controversias y principales dificultades (también citando fuentes)

% TODO: se puede hablar de las metodologías de trabajo propuestas para el uso de AI

La inteligencia artificial puede aplicarse a un sin fin de situaciones para resolver problemas tecnológicos o facilitar tareas que requieran un excesivo trabajo manual. Abdelsamea et al. reúnen en su revisión múltiples aplicaciones de estas tecnologías en histología, algunas de las cuales son: clasificación de tipos de cáncer de pulmón a partir de WSIs de histopatología, clasificación de cáncer de mama a partir de imágenes, clasificación de cáncer cervical, segmentación de cáncer multiorgánico, segmentación de cáncer cerebral, entre otros \cite{Abdelsamea2022}.

Por otro lado, en la comunidad de Kaggle es frecuente encontrar múltiples propuestas de algoritmos y modelos que se encargan clasificar o predecir tipos de cáncer \cite{KaggleSearch}. Esto demuestra un claro interés generalizado en investigar y desarrollar aún más estas tecnologías.

Fuera del ámbito de la histopatología, se destacan dos propuestas recientes. En primer lugar, Lu et al. proponen un flujo de trabajo en seis pasos que utiliza modelos de lenguaje preentrenados e incorpora un chequeo intermedio manual por parte de médicos, con el propósito de lograr explicaciones confiables basadas en texto, interpretables y robustas \cite{Lu2023}. Por último, Tariq et al. elaboran una revisión sistemática acerca del estado del arte del aprendizaje federado confiable, exponen sus ventajas, su taxonomía y un posible marco de trabajo de aplicación. Los autores sostienen que el uso de aprendizaje federado puede permitir el entrenamiento de modelos IA colaborativos entre dispositivos, manteniendo la seguridad de la información \cite{Tariq2024}.
