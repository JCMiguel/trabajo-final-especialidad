%%%%%%%%%%%%%%%%%%%%%%%%%%%%%%%%%%%%%%%%%%%%%%%%%%%%%%
%   CONTEXTO PRELIMINAR - ESTADO DEL ARTE
%%%%%%%%%%%%%%%%%%%%%%%%%%%%%%%%%%%%%%%%%%%%%%%%%%%%%%
\section{Contexto preliminar} \label{section:es/contexto-preliminar}

Durante la última década la inteligencia artificial fue foco de investigación y desarrollo. En particular en medicina, según una estadística de la Universidad de Stanford en 2022 el FDA (Food and Drugs Administration) aprobó un 12,1\% más de equipamiento médico basado en IA respecto al año 2021. Históricamente esta tasa fue creciendo paulatinamente, según se muestra en la Figura \ref{figura:es/stanford_fda_approved}.

\begin{figure}[htbp]
\centerline{\includegraphics[width=\textwidth]{Imagenes/stanford_fda_approved.png}}
\caption{Número de dispositivos médicos aprobados por la FDA. Gráfico extraído del AI Index, de la Stanford University, Chapter 5 \cite{AIIndexStanfordWebsite}.}
\label{figura:es/stanford_fda_approved}
\end{figure}

No obstante esto, la opinión pública de los profesionales respecto a la integración de la IA en sus puestos de trabajo está dividida entre quienes lo consideran un beneficio y los que lo ven como una amenaza, hecho reflejado en la Figura \ref{figura:es/stanford_global_opinion}. Esto bien puede estar fundamentado en la falta de transparencia en la mayoría de los algoritmos de IA populares actualmente. La Figura \ref{figura:es/stanford_transparency} refleja este hecho. En ella se muestra el puntaje FMTI (Foundation Model Transparency Index) para algunos IA actuales de usos varios \cite{TMFIWebsite}. En términos generales, cuanto más alto es este valor, más transparente es el modelo. 

\begin{figure}[htbp]
\centerline{\includegraphics[width=\textwidth]{Imagenes/stanford_global_opinion.png}}
\caption{Opinión pública acerca de la integración de la IA en puestos de trabajo. Gráfico extraído del AI Index, de la Stanford University, Chapter 9 \cite{AIIndexStanfordWebsite}.}
\label{figura:es/stanford_global_opinion}
\end{figure}

\begin{figure}[htbp]
\centerline{\includegraphics[width=\textwidth]{Imagenes/stanford_transparency.png}}
\caption{Índices de transparencia FMTI, evaluados en algunos de los modelos de inteligencia artificial actuales. Gráfico extraído del AI Index, de la Stanford University, Chapter 3 \cite{AIIndexStanfordWebsite}.}
\label{figura:es/stanford_transparency}
\end{figure}

Por todo lo dicho en los párrafos precedentes, es pertinente brindar un breve marco teórico que sirva de sustento a la revisión sistemática presentada en este este trabajo. Se comienza con una introducción conceptual teórica acerca de la inteligencia artificial en el apartado \ref{section:es/nociones-ia}. En la sección \ref{section:es/nociones-xai}, se presentan los orígenes de IAX y el estado del arte actual. En la sección \ref{section:es/taxonomía-xai} se expone una posible taxonomía para todas las técnicas IAX existentes hasta la fecha. La sección \ref{section:es/uso-ia-medicina} reúne una serie de ejemplos de implementaciones de AI y IAX en medicina. Finalmente, en la sección \ref{section:es/lineamientos-tfe} se introducen las adopciones, criterios y lineamientos utilizados durante la revisión sistemática. Se añade además en el apéndice \ref{appendix:es/glosario} un glosario complementario que reúne términos de interés que brindan mayor sustento a los resultados de la revisión sistemática.

%TODO: Hablar de los trabajos de Montavon, Holzinger, Lipton, Guidoti, Samek y Roscher.
%TODO del TODO: Necesito repasar qué hicieron estos tipos...

%Temario posible para estado del arte o contexto preliminar
%- Nociones de Inteligencia Artificial
%	- Clasificaciones
%- Nociones de xAI
%	- Nomenclatura habitual (interpretability, trustworthy, explainability, causality, etc.)
%- Taxonomía de xAI
%- Usos de Inteligencia artificial en medicina
%	- Promesas o ventajas tentativas
%	- Diagnóstico diferencial
%	- Aprendizaje Federado
%- LLM en medicina (no sé bien dónde poner esto, pero siento que debería introducirlo)
%
%- Lineamientos y adopciones para este trabajo
%	- esclarecer porqué uso la sigla AI/xAI para referirme a AI y XAI en simultáneo o ambas por separado.
%	- Tengo que añadir una sección de conceptos importantes para informar la manera en que traduje cada término en inglés. ¿O debería dejarlos en inglés?
% Màs ideas en la parte de abajo...

% FUENTES: (¿Por qué anoté esto acá?)
% 1-Journal-Explainable AI Healthcare.pdf
% 4-Ayorinde-AI Renal Histopathology.pdf ???
% 5-Liu-AI in Medicine.pdf
% 6-Tosun-xAI Anatomic Pathology.pdf
% 7-Jarrahi-Workflow pathologists-IA.pdf
% 8-Kiehl-Computational Pathology.pdf !!!!
% 10-Verma-Rethinking Role AI.pdf









