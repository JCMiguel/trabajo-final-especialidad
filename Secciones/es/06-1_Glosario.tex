%%%%%%%%% SECTION %%%%%%%%%
\subsection{Glosario de términos} \label{appendix:es/glosario}



\subsubsection{Términos específicos de informática}

\textbf{Aprendizaje automático (Machine Learning)}: \textit{abrev. ML}. \enquote{Es una rama de la inteligencia artificial que estudia como dotar a las máquinas de capacidad de aprendizaje, basándose en algoritmos capaces de identificar patrones en grandes bases de datos y aprender de ellos} \cite{tiposIA-ML-DL}.  

\textbf{Aprendizaje profundo (Deep Learning)}: \textit{abrev. DL}. Es un subconjunto del aprendizaje automático que usa redes neuronales de múltiples capas para simular la toma de decisiones que realiza el cerebro humano \cite{defDeepLearning}.

\textbf{Set o conjunto de datos (Dataset)}: En el ámbito de IA, se le denomina así al conjunto de datos que se utiliza durante el entrenamiento y pruebas de un modelo de IA.

\textbf{Aumento de datos (Data augmentation)}: Son un conjunto de técnicas y procesos que permiten generar artificialmente nuevos datos a partir de otros existentes \cite{defDataAugmentation}. Aplicadas a imágenes, se las suele clasificar en dos tipos: transformaciones geométricas y transformaciones fotométricas. Las primeras consisten en modificar el espacio y disposición de la imagen, mientras que las segundas implican modificar su distribución de color \cite{tiposDataAugmentation}.

\textbf{Aprendizaje supervisado (Supervised learning)}: Es una forma de entrenamiento de IA que utiliza datos muy estructurados y completamente etiquetados para elaborar resultados consistentes \cite{defDeepLearning}.

\textbf{Aprendizaje semi-supervisado (Semi-supervised o weakly supervised learning)}: Es un conjunto de técnicas que modifican o complementan el aprendizaje supervisado incorporando datos no etiquetados \cite{defSemiSupervised}.

\textbf{Red neuronal artificial (Artificial Neural Network)}: Una red neuronal artificial, también llamada solo red neuronal, es una estructura de nodos, llamadas neuronas, interconectados entre sí, cada uno con un valor, denominado peso. Esta estructura consta como mínimo de una serie de capas de entrada, al menos una capa intermedia, que se la conoce como capa oculta, y una última capa de neuronas de salida. A través de estas interconexiones, y mediante cálculos matemáticos, es posible convertir datos de entrada en otros de salida. Los pesos se ajustan en forma automática durante el entrenamiento de la red \cite{defRedNeuronal}.

\textbf{Red neuronal convolucional (Convolutional Neural Network)}: \textit{abrev. CNN}. Son un tipo de red neuronal que se suele utilizar en reconocimiento de imagen y visión computarizada. Estas redes utilizan operaciones matriciales de convolución para identificar patrones en una imagen \cite{defCNN}.

\textbf{Mapas de saliencia (Saliency maps)}: El concepto de mapa de saliencia, en esencia, se refiere al conjunto de características de un objeto que captan la atención de las personas. Estas características pueden estar relacionadas con el color, la forma, tamaño, iluminación, brillo y profundidad del objeto \cite{defMapaSaliencia}. Es un concepto que se utiliza como base para la extracción de características e interpretabilidad de redes neuronales convolucionales \cite{usosMapaSaliencia}.

\textbf{Modelos de lenguaje extensos (Large-Language modelos)}: \textit{abrev. LLM}. \enquote{Es un modelo estadístico de lenguaje entrenado con una gran cantidad de datos que puede utilizarse para generar y traducir texto y otros tipos de contenido, así como para llevar a cabo otras tareas de procesamiento del lenguaje natural (PLN)} \cite{defLLM-Google}.

\textbf{Procesamiento de lenguaje natural (Natural Language Processing)}: \textit{abrev. NLP}. Es un subcampo de la informática y la inteligencia artificial que permite a las computadoras y dispositivos digitales reconocer, comprender y generar texto y voz. Para lograr este cometido, se recurren a modelos estadísticos y modelos de aprendizaje automático \cite{defProcLengNatu}.

\textbf{Lógica difusa (Fuzzy Logic)}: Es un modelo de razonamiento lógico ambiguo, impreciso y multivariable, en el que los valores de verdad se interpretan en función de grados o niveles de verdad \cite{defFuzzyLogic}. Este concepto tiene aplicación en informática como parte del diseño de sistemas de control y predicción, definiendo funciones de pertenencia que convierte un conjunto de datos de entrada en otros de salida \cite{usosFuzzyLogic}.

\textbf{Integración Humano-Computadora (Human-Computer Integration)}: \textit{abrev. HCI}. Se refiere al conjunto de técnicas de diseño e implementación de sistemas informáticos con los que los usuarios pueden interactuar \cite{defHCI}. Para que exista una integración exitosa con el usuario, se debe velar por lograr la funcionalidad deseada y maximizar la usabilidad del sistema.

\textbf{Sistema de IA intervenido por expertos (expert-in-the-loop o human-in-the-loop)}: \enquote{Es un enfoque colaborativo que integra la aportación humana y la experiencia en el ciclo de vida de los sistemas de aprendizaje automático e inteligencia artificial. Los seres humanos participan activamente en el entrenamiento, la evaluación o el funcionamiento de los modelos de aprendizaje automático ofreciendo directrices, comentarios y anotaciones valiosos} \cite{defHumanInTheLoop}.



\subsubsection{Términos específicos de medicina e histología}

\textbf{Herramientas de diagnóstico asistido por computadora (Computer Aided Diagnosis)}: \textit{abrev. CAD}. Son sistemas que pretenden asistir a los médicos de distintas disciplinas proveyéndoles de una \enquote{segunda perspectiva} de diagnóstico \cite{defCAD}.

\textbf{Imágenes de cortes histológicos completos (Whole Slide Image)}: \textit{abrev. WSI}. Se trata de una imagen digitalizada de un corte histológico microscópico, de suma utilidad en histoquímica y citoquímica. Se ha demostrado que su uso en diagnóstico patológico no es inferior al diagnóstico basado en microscopios \cite{defWSI}.

