%%%%%%%%%%%%%%%%%%%%%%%%%%%%%%%%%%%%%%%%%%%%%%%%%%%%%%
%   Lineamientos y adopciones para este trabajo
%%%%%%%%%%%%%%%%%%%%%%%%%%%%%%%%%%%%%%%%%%%%%%%%%%%%%%
\subsection{Lineamientos y adopciones para el presente trabajo} \label{section:es/lineamientos-tfe}

%- Lineamientos y adopciones para este trabajo
%	- esclarecer porqué uso la sigla AI/xAI para referirme a AI y XAI en simultáneo o ambas por separado.
%	- Tengo que añadir una sección de conceptos importantes para informar la manera en que traduje cada término en inglés. ¿O debería dejarlos en inglés?
% Màs ideas en la parte de abajo...

% TODO: Justificar por qué este tema es importante. Por ejemplo: ¿hay pocos diagnósticos? ¿Se tarda mucho en diagnosticar? ¿Se suelen hacer mal? ¿Cuesta especializarse en determinados temas patológicos? Todo eso es importante porque le da solidez al tema.

Si bien a veces en documentación hispanohablante se utilizan las siglas AI y XAI, derivadas del inglés, para referirse a la inteligencia artificial y a la inteligencia artificial explicable, respectivamente, en este trabajo se opta por sus siglas traducidas al español. Así, a lo largo de todo este documento se hablará de IA e IAX para referirse a ambos términos. Ocasionalmente, se utilizará la sigla IA/IAX para referir a ambas en simultáneo. Esto particularmente es útil para describir conceptos que son aplicales a ambas tecnologías por igual. Sin embargo, otras siglas referidas a algoritmos concretos, técnicas específicas y conceptos médicos o informáticos se conservaron en inglés con el propósito de evitar ambigüedad por repetición de acrónimos.

Los conceptos técnicos y médicos más utilizados durante el desarrollo de la revisión sistemática se adjuntan en el glosario del Apéndice \ref{appendix:es/glosario}. En el caso de que se utilicen siglas o acrónimos en inglés para referirse a algoritmos, metodologías o técnicas específicas, se incluye también una posible traducción en dicho glosario.