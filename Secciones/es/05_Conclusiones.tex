%%%%%%%%%%%%%%%%%%%%%%%%%%%%%%%%%%%%%%%%%%%%%%%%%%%%%%
%   CONCLUSIONES
%%%%%%%%%%%%%%%%%%%%%%%%%%%%%%%%%%%%%%%%%%%%%%%%%%%%%%
\section{Conclusiones} \label{section:es/conclusiones}

% Though the implementation of IA/IAX seems promising, this article brings to light the challenges that must be solved so that an effective implementation of these technologies may take place in histopathology. These difficulties are not limited to technical ones only, but they also cover other equally important perspectives.
%The coupling workflow is the main factor that must be considered when designing an AI system. It is necessary to establish a list of requirements with specialists and pathologists with the purpose of understanding their needs and finding the best way they can benefit themselves from these technologies. IA/IAX systems must be conceived as assistants to pathologists, and allow for sufficient transparency so their results can be supervised. 
%It is possible that, to achieve an adequate integration with the workflow, it is vital to incorporate the expert-in-the-loop approach. Another possibility is to consider IA/IAX algorithms like components of a more complex system that implements pieces of traditional programming and results expressed in natural language. With the recent advent of Large Language Models (LLM), the aforementioned bonding between pathologists and xAI systems could be enhanced.
%IA/IAX systems are highly sensitive to the quality of their data. Adequately classified samples are crucial for the success of the AI training. Due to this task being exhaustive, it may be accelerated by resorting to weakly supervised learning methods. However, to obtain a precise classification in oncology and renal histopathology sometimes is complex. In the face of this difficulty, designing AI systems from an approach centered in causability instead of explainability might contribute significantly to the professionals. 
%There is still a necessity for a set of regulations that guarantee a standard in relation to the testing and the use of AI systems in histopathology, in the case that these systems require the utilization of patient-related data for training.
%The cost of implementation is also a limiting factor in some organizations. Against this, open-source alternatives might become the starting point for the implementation of AI-based solutions.
%As for the next steps of this article, this work will be expanded to create a systematic mapping study of literature. It is expected for this to be the foundations of a future investigation.

Pese a que la implementación de IA/IAX es prometedora, en este trabajo se evidencian los desafíos que se deben resolver para una implementación efectiva de estas tecnologías en histopatología. Estas dificultades no se limitan a lo técnico, sino también a aspectos regulatorios y de integración.

La integración al flujo de trabajo es el principal factor que se debe considerar durante el diseño de un sistema de IA. Si bien se han demostrado avances en la recolección de requisitos con especialistas y patólogos, tarea que permite comprender sus necesidades y cómo podrían beneficiarse de la IA, la falta de lenguaje común entre los desarrolladores y el personal médico sigue siendo una brecha limitante. Aspirar a disponer de capacitaciones o niveles de estudio integrales que favorezcan la comunicación entre ambos tipos de profesionales puede ser útil para reducir esta brecha.

La tendencia actual, según los datos reportados en PI3, es favorecer un ámbito colaborativo entre la IA y los profesionales del área médica. Pese al esfuerzo por parte de algunos autores de reemplazar al patólogo por sistemas de inteligencia artificial, esta tarea es riesgosa y poco aconsejable debido a los sesgos que pueden cometerse y los daños potenciales que pueden sufrir los pacientes. Por lo tanto, es primordial que los sistemas IA/IAX sean concebidos como asistentes supervisables, explicables y transparentes, para que los patólogos puedan beneficiarse de su uso en la práctica diaria. No obstante, se ha reportado el riesgo de que el personal puede confiar en exceso en estas tecnologías. Teniendo este factor en cuenta, es imprescindible llevar adelante estudios exhaustivos a mediano plazo que permitan evaluar que el uso de IA en tareas de diagnóstico no sea contraproducente.

%Es probable que para lograr una integración adecuada con el flujo de trabajo sea preciso incorporar el enfoque expert-in-the-loop. Otra posibilidad es plantear los IA/IAX algorithms como componentes de un sistema màs complejo que implemente pieces of traditional programming y resultados expresados in natural language.

Los sistemas IA/IAX son altamente sensibles a la calidad de sus datos. Si bien las técnicas de aumento de datos y el aprendizaje semi-supervisado han demostrado buenos resultados, es aconsejable contar con conjuntos de datos recolectados de distintos centros médicos y digitalizados con equipamiento diferente para que las predicciones sean lo más fidedignas posible. Una posible solución arquitectónica a esta necesidad es implementar aprendizaje federado o implementaciones en la nube. Independientemente de cuál sea la implementación final, se debe velar en todo momento por garantizar la privacidad de los pacientes y la seguridad de sus datos y los del sistema.

Ninguno de los estudios primarios encontrados inspecciona el impacto ambiental de entrenar herramientas de diagnóstico basadas en IA, orientadas a un uso masivo en la práctica diaria. Esta cuestión debería ser explorada para poder aspirar a una arquitectura y orquestación eficientes.

El costo de implementación es un factor limitante en algunas organizaciones. Frente a esto, las alternativas de software libre pueden ser el punto de partida para implementar sistemas basados en IA de uso masivo.

Dado el crecimiento de las herramientas de lenguaje natural, es preciso contar con reglamentación actualizada que garantice la seguridad de los pacientes antes el uso y estudio de estas tecnologías en tareas de diagnóstico.

Se aconseja explorar más en profundidad el potencial de soluciones basadas en lógica difusa, debido a su naturaleza interpretable.

Pese a las propuestas y esfuerzo de varios autores, aún hay una notable carencia de definiciones consensuadas y un marco teórico uniforme para las técnicas IAX. Esto se manifiesta sobre todo en el hecho de que las palabras \enquote{interpretabilidad} y \enquote{explicabilidad} suelen usarse indistintamente en las investigaciones como si fueran sinónimos, característica con la que algunos autores no están de acuerdo.

%TODO: Temas interesantes a plantear
% - La cuestión reglamentaria
% - La falta de definiciones concensuadas y marco teórico uniforme.
% - Ningún artículo plantea el impacto ambiental que ocasiona el entrenamiento de algoritmos de IA. El aprendizaje federado podría ser una manera eficiente de perjudicar lo menos posible??
% - Pese a los esfuerzo por parte de algunos autores de reemplazar al patólogo por una inteligencia artificial, esta tarea es riesgosa y poco aconsejable, dados los sesgos que pueden cometerse y los daños potenciales que pueden sufrir los pacientes. La forma más acopable al flujo de trabajo es conseguir un asistente tecnológico que aporte verdadero valor a la práctica diaria.
% - No todos las muestras por región tiene los mismos patrones de tejido, lo que podría ocasionar en limitaciones de aprendizaje. Una buena forma de mitigar este problema es utilizar un sistema inteligente o multiagente que elabore conclusiones con la muestra y con los datos del paciente. (¿No había un artículo que planteaba esto? ¿Era estudio primario?). Esto podría ser un paso futuro de investigación.
% - El potencial de lógica difusa como sistema interpretable y extractor de conclusiones (ver TP de SMA).



