%%%%%%%%% SECTION %%%%%%%%%
\subsection{PRISMA: Lista de verificación} \label{appendix:es/prisma_checklist}

La lista de verificación usada para la revisión sistemática se presenta en la Tabla \ref{tabla:es/prisma_checklist}. La tabla está basada en la traducción al español, publicada por Page et al. \cite{PRISMAespañol}.

{
\scriptsize
\begin{longtable}[h]{M{1.8cm}|M{0.7cm}|M{8.7cm}|M{2cm}}

\hline
    \textbf{Sección o Tema} & \textbf{Item} & \textbf{Ítem de la lista de verificación} & \textbf{Localización en la publicación} \\
    \hline \hline
    \endhead
\hline
    \caption{Lista de verificación de metodología PRISMA. \label{tabla:es/prisma_checklist}}
    \endfirstfoot
\hline
    \caption[]{(Continuación) Lista de verificación de metodología PRISMA.}
    \endfoot
\multicolumn{4}{l}{\textbf{Título}} \\
\hline
Título & 1 & Identifique la publicación como una revisión sistemática. & Título \\
\hline
\hline
\multicolumn{4}{l}{\textbf{Resumen}} \\
\hline
Resumen estructurado & 2 &  Vea la lista de verificación para resúmenes estructurados de la declaración PRISMA 2020. & Introducción \\
\hline
\hline
\multicolumn{4}{l}{\textbf{Introducción}} \\
\hline
Justificación & 3 &Describa la justificación de la revisión en el contexto del conocimiento existente.& Sección \ref{section:es/introduccion}\\
\hline
Objetivos & 4 & Proporcione una declaración explícita de los objetivos o las preguntas que aborda la revisión. & Sección \ref{section:es/introduccion}\\
\hline
\hline
\multicolumn{4}{l}{\textbf{Métodos}} \\
\hline
Criterios de elegibilidad &5&  Especifique los criterios de inclusión y exclusión de la revisión y cómo se agruparon los estudios para la síntesis. & Sección \ref{section:es/criterios_inc_exc} \\
\hline
Fuentes de información &6&  Especifique todas las bases de datos, registros, sitios web, organizaciones, listas de referencias y otros recursos de búsqueda o consulta para identificar los estudios. Especifique la fecha en la que cada recurso se buscó o consultó por última vez. & Sección \ref{section:es/cadena_busqueda} \\
\hline
Estrategia de búsqueda & 7&  Presente las estrategias de búsqueda completas de todas las bases de datos, registros y sitios web, incluyendo cualquier filtro y los límites utilizados. & Sección \ref{section:es/cadena_busqueda} \\
\hline
Proceso de selección de los estudios &8 & Especifique los métodos utilizados para decidir si un estudio cumple con los criterios de inclusión de la revisión, incluyendo cuántos autores de la revisión cribaron cada registro y cada publicación recuperada, si trabajaron de manera independiente y, si procede, los detalles de las herramientas de automatización utilizadas en el proceso. & Sección \ref{section:es/proceso_seleccion} \\
\hline
Proceso de extracción de los datos & 9 & Indique los métodos utilizados para extraer los datos de los informes o publicaciones, incluyendo cuántos revisores recopilaron datos de cada publicación, si trabajaron de manera independiente, los procesos para obtener o confirmar los datos por parte de los investigadores del estudio y, si procede, los detalles de las herramientas de automatización utilizadas en el proceso. & Sección \ref{section:es/proceso_seleccion} \\
\hline
Lista de los datos &10a & Enumere y defina todos los desenlaces para los que se buscaron los datos. Especifique si se buscaron todos los resultados compatibles con cada dominio del desenlace (por ejemplo, para todas las escalas de medida, puntos temporales, análisis) y, de no ser así, los métodos utilizados para decidir los resultados que se debían recoger. & Sección \ref{section:es/proceso_seleccion} \\
\cline{2-4} 
 &10b & Enumere y defina todas las demás variables para las que se buscaron datos (por ejemplo, características de los participantes y de la intervención, fuentes de financiación). Describa todos los supuestos formulados sobre cualquier información ausente (missing) o incierta. & Sección \ref{section:es/proceso_seleccion} \\
\hline
Evaluación del riesgo de sesgo de los estudios individuales & 11 & Especifique los métodos utilizados para evaluar el riesgo de sesgo de los estudios incluidos, incluyendo detalles de las herramientas utilizadas, cuántos autores de la revisión evaluaron cada estudio y si trabajaron de manera independiente y, si procede, los detalles de las herramientas de automatización utilizadas en el proceso. & Sección \ref{section:es/calidad} \\
\hline
Medidas del efecto & 12 & Especifique, para cada desenlace, las medidas del efecto (por ejemplo, razón de riesgos, diferencia de medias) utilizadas en la síntesis o presentación de los resultados. & No aplica \\
\hline
Métodos de síntesis &13a & Describa el proceso utilizado para decidir qué estudios eran elegibles para cada síntesis (por ejemplo, tabulando las características de los estudios de intervención y comparándolas con los grupos previstos para cada síntesis (ítem 5). &  Sección \ref{section:es/proceso_seleccion} \\
\cline{2-4} 
 &13b & Describa cualquier método requerido para preparar los datos para su presentación o síntesis, tales como el manejo de los datos perdidos en los estadísticos de resumen o las conversiones de datos. &  Sección \ref{section:es/proceso_seleccion} \\
\cline{2-4} 
 & 13c & Describa los métodos utilizados para tabular o presentar visualmente los resultados de los estudios individuales y su síntesis. &  Secciones \ref{section:es/proceso_seleccion} y \ref{section:es/resultados} \\
\cline{2-4} 
 & 13d & Describa los métodos utilizados para sintetizar los resultados y justifique sus elecciones. Si se ha realizado un metanálisis, describa los modelos, los métodos para identificar la presencia y el alcance de la heterogeneidad estadística, y los programas informáticos utilizados. & Sección \ref{section:es/proceso_seleccion} \\
\cline{2-4} 
 & 13e & Describa los métodos utilizados para explorar las posibles causas de heterogeneidad entre los resultados de los estudios (por ejemplo, análisis de subgrupos, metarregresión). & Secciones \ref{section:es/cadena_busqueda} y \ref{section:es/proceso_seleccion} \\
\cline{2-4} 
 & 13f & Describa los análisis de sensibilidad que se hayan realizado para evaluar la robustez de los resultados de la síntesis. & No aplica \\
\hline
Evaluación del sesgo en la publicación & 14 & Describa los métodos utilizados para evaluar el riesgo de sesgo debido a resultados faltantes en una síntesis (derivados de los sesgos en las publicaciones). & Sección \ref{section:es/calidad} \\
\hline
Evaluación de la certeza de la evidencia & 15 & Describa los métodos utilizados para evaluar la certeza (o confianza) en el cuerpo de la evidencia para cada desenlace. & No aplica \\
\hline
Selección de los estudios & 16a & Describa los resultados de los procesos de búsqueda y selección, desde el número de registros identificados en la búsqueda hasta el número de estudios incluidos en la revisión, idealmente utilizando un diagrama de flujo. & Sección \ref{section:es/resultados} y Figura \ref{figura:es/data_extraction} \\
\cline{2-4} 
 & 16b & Cite los estudios que aparentemente cumplían con los criterios de inclusión, pero que fueron excluidos, y explique por qué fueron excluidos. & Sección \ref{section:es/resultados} y Figura \ref{figura:es/data_extraction} \\
\hline
Características de los estudios & 17 & Cite cada estudio incluido y presente sus características. & Sección \ref{section:es/resultados} y Tabla \ref{tabla:es/calidad} \\
\hline
Riesgo de sesgo de los estudios individuales & 18 & Presente las evaluaciones del riesgo de sesgo para cada uno de los estudios incluidos. & Sección \ref{section:es/calidad} \\
\hline
Resultados de los estudios individuales & 19 & Presente, para todos los desenlaces y para cada estudio: a) los estadísticos de resumen para cada grupo (si procede) y b) la estimación del efecto y suprecisión (por ejemplo, intervalo de credibilidad o de confianza), idealmente utilizando tablas estructuradas o gráficos. & Sección \ref{section:es/resultados} \\
\hline
Resultados de la síntesis & 20a &  Para cada síntesis, resuma brevemente las características y el riesgo de sesgo entre los estudios contribuyentes. & Sección \ref{section:es/resultados} \\
\cline{2-4} 
 & 20b & Presente los resultados de todas las síntesis estadísticas realizadas. Si se ha realizado un metanálisis, presente para cada uno de ellos el estimador de resumen y su precisión (por ejemplo, intervalo de credibilidad o de confianza) y las medidas de heterogeneidad estadística. Si se comparan grupos, describa la dirección del efecto. & No aplica \\
\cline{2-4} 
 & 20c & Presente los resultados de todas las investigaciones sobre las posibles causas de heterogeneidad entre los resultados de los estudios. & No aplica \\
\cline{2-4} 
 & 20d & Presente los resultados de todos los análisis de sensibilidad realizados para evaluar la robustez de los resultados sintetizados. & No incluido \\
\hline
Sesgos en la publicación & 21 & Presente las evaluaciones del riesgo de sesgo debido a resultados faltantes (derivados de los sesgos de en las publicaciones) para cada síntesis evaluada. & No incluido \\
\hline
Certeza de la evidencia & 22 & Presente las evaluaciones de la certeza (o confianza) en el cuerpo de la evidencia para cada desenlace evaluado. & Sección \ref{section:es/resultados} \\
\hline
\hline
\multicolumn{4}{l}{\textbf{Discusión}} \\
\hline
Discusión & 23a & Proporcione una interpretación general de los resultados en el contexto de otras evidencias. & Sección \ref{section:es/conclusiones} \\
\cline{2-4} 
 & 23b & Argumente las limitaciones de la evidencia incluida en la revisión. & Sección \ref{section:es/conclusiones} \\
\cline{2-4} 
 & 23c & Argumente las limitaciones de los procesos de revisión utilizados. & No incluido \\
\cline{2-4} 
 & 23d & Argumente las implicaciones de los resultados para la práctica, las políticas y las futuras investigaciones. & Sección \ref{section:es/conclusiones} \\
\hline
\hline
\multicolumn{4}{l}{\textbf{Otra información}} \\
\hline
Registro y protocolo & 24a & Proporcione la información del registro de la revisión, incluyendo el nombre y el número de registro, o declare que la revisión no ha sido registrada. & Revisión no registrada \\
\cline{2-4} 
 & 24b & Indique dónde se puede acceder al protocolo, o declare que no se ha redactado ningún protocolo. & Sección \ref{section:es/protocolo} \\
\cline{2-4} 
 & 24c & Describa y explique cualquier enmienda a la información proporcionada en el registro o en el protocolo. & No aplica \\
\hline
Financiamiento & 25 & Describa las fuentes de apoyo financiero o no financiero para la revisión y el papel de los financiadores o patrocinadores en la revisión. & No aplica \\
\hline
Conflicto de intereses & 26 & Declare los conflictos de intereses de los autores de la revisión. & No declara \\
\hline
Disponibilidad de datos, códigos y otros materiales & 27 & Especifique qué elementos de los que se indican a continuación están disponibles al público y dónde se pueden encontrar: plantillas de formularios de extracción de datos, datos extraídos de los estudios incluidos, datos utilizados para todos los análisis, código de análisis, cualquier otro material utilizado en la revisión. & Sección \ref{section:es/resultados}, Apéndices y Referencias

\end{longtable}
}