%%%%%%%%%%%%%%%%%%%% SECTION %%%%%%%%%%%%%%%%%%%%%%%
\subsection{Coupling workflow}
% AI no como reemplazo. AI no reemplaza la histologìa manual

Ayorinde et al. \cite{Ayorinde2022} postulate that it is not necessary to achieve a perfect AI system for it to be used in the hospitals. In fact, it is more probable that the most useful tools are a combination between AI models and a set of work rules that favor the effective human supervision.

According to Tosun et al. \cite{Tosun2020} the main function of xAI in pathology is to promote safety, reliability, and accountability in addressing issues with bias, transparency, safety, and causality. The authors find, however, that there is not yet a consensus on how pathologists should supervise or work with computational pathology systems. As patient safety in pathology is the result of a complex interaction between pathologists, other physicians, laboratory personnel and computational pathology applications, they conclude that xAI must help the pathologist be more precise and efficient in their work.

Jaharri et al. \cite{Jarrahi2022} claim that the AI/xAI systems employed in medicine present interoperability challenges, given that these systems often show an amount of information or in formats that appear indecipherable to physicians. These authors offer that an expert-in-the-loop AI work system could clarify a mutual workflow between humans and machines.

As for Verma et al. \cite{Verma2023}, they suggest that AI can be included in collective decision-making processes in oncology either as a tool or as a member, each of these alternatives generating different sets of ethical, societal and technological issues. In an interview they conducted for their article \cite{Verma2023} with seven physicians working at the Lausanne University Hospital (CHUV), in relation to AI systems autonomy, they detected a consensus on how trust cannot be put in something that is not trustful or bypasses the doctors. One of the experts that took part in this interview added that an AI model cannot be trusted if it is not able to choose which treatment modality works best for a particular patient.
