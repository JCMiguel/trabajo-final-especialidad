\section{Conclusions and Future work} \label{section:en/conclusiones}

Though the implementation of AI/xAI seems promising, this article brings to light the challenges that must be solved so that an effective implementation of these technologies may take place in histopathology. These difficulties are not limited to technical ones only, but they also cover other equally important perspectives.

The coupling workflow is the main factor that must be considered when designing an AI system. It is necessary to establish a list of requirements with specialists and pathologists with the purpose of understanding their needs and finding the best way they can benefit themselves from these technologies. AI/xAI systems must be conceived as assistants to pathologists, and allow for sufficient transparency so their results can be supervised. 
It is possible that, to achieve an adequate integration with the workflow, it is vital to incorporate the expert-in-the-loop approach. Another possibility is to consider AI/xAI algorithms like components of a more complex system that implements pieces of traditional programming and results expressed in natural language. With the recent advent of Large Language Models (LLM), the aforementioned bonding between pathologists and xAI systems could be enhanced.

AI/xAI systems are highly sensitive to the quality of their data. Adequately classified samples are crucial for the success of the AI training. Due to this task being exhaustive, it may be accelerated by resorting to weakly supervised learning methods. However, to obtain a precise classification in oncology and renal histopathology sometimes is complex. In the face of this difficulty, designing AI systems from an approach centered in causability instead of explainability might contribute significantly to the professionals. 

There is still a necessity for a set of regulations that guarantee a standard in relation to the testing and the use of AI systems in histopathology, in the case that these systems require the utilization of patient-related data for training.

The cost of implementation is also a limiting factor in some organizations. Against this, open-source alternatives might become the starting point for the implementation of AI-based solutions.

As for the next steps of this article, this work will be expanded to create a systematic mapping study of literature. It is expected for this to be the foundations of a future investigation.