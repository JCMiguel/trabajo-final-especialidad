%%%%%%%%%%%%%%%%%%%% SECTION %%%%%%%%%%%%%%%%%%%%%%%
\subsection{Regulation aspects}
% En caso de confiar en un algoritmo de AI que brinda un diagnòstico equivocado, ¿quién es responsable bajo la ley?
% Privacidad de datos. Datasets públicos vs historia clínica de los pacientes (privada por default)

In relation to regulations, Tosun et al. \cite{Tosun2020} reveal that a proposed regulation before the European Union would prohibit “automatic processing” unless people are safeguarded. They foresee that future laws may further restrict AI use in professional practices, which represents a huge challenge to industry.

Dos-Santos et al. \cite{dosSantos2022} affirm that AI algorithms must work under regulatory standards for testing and usage in medical facilities. According to Tosun et al. \cite{Tosun2020}, this is already a known necessity in radiology.

Thus, due to data privacy, Rösler et al. \cite{Roesler2023} observe that professionals must work under an ethical approval and informed consent of the patient prior to the use and evaluation of patient-related data. In addition to this, if possible, anonymized raw data have to be included from the start of the training process.
