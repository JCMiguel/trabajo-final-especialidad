\begin{abstract}
% Un abstract suele ser un resumen del trabajo que presentarás, pero, más importante aún, tiene que ser el “vendedor” que convence al lector (y antes al revisor!) de lo que sigue. Tu mismo has hecho el ejercicio de una revisión, y sabes de lo importante que puede ser el abstract para convencerte de incluir un artículo en algo. De esa misma experiencia puedes deducir cómo escribir un abstract que vende tu artículo. Intenta algo y luego revisamos.

The number of professional pathologists is not high in contrast with the total population, and, worldwide, the cases of cancer have a tendency to rise each year. In the face of this, the implementation of Artificial Intelligence (AI) and, more specifically, Explainable Artificial Intelligence (xAI) techniques could contribute to prevent a work overload on pathologists. Despite recent advances in this subject, AI/xAI systems are still not fully integrated in the histopathology workflow. This could be due to the fact that the implementation of AI/xAI models in histopathology is subject to technical, social and legal requirements, among others. It is necessary to determine these requirements in order to solve this issue. With the intention of providing a wider picture, this article will present a rapid literature review bringing together all current requirements and obstacles that the implementation of AI/xAI faces in histopathology.
\end{abstract}


\begin{IEEEkeywords}
xAI, Digital Pathology, Computational Pathology, Histology, Histopathology
\end{IEEEkeywords}