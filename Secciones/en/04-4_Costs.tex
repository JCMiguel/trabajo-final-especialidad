%%%%%%%%%%%%%%%%%%%% SECTION %%%%%%%%%%%%%%%%%%%%%%%
\subsection{Implementation costs}
% Acerca de LMIC

Zehra et al. \cite{Zehra2023} analyze the implementation of AI in low- and middle-income countries (LMICs). They remark that, while AI systems there face similar technical challenges and issues to those of more developed countries, there are further difficulties in LMICs. The authors perceive that when labs struggle due to financial constraints to hire trained histopathologists, and when there is scarcity of trained laboratory technologists even for conventional histopathology, it would be extremely difficult to obtain funds and manpower for implementing AI and digital pathology. In the face of this, they propose a series of low-cost alternatives to venture into AI systems. For instance, low resource organizations can access available open-source whole-slide image archives provided by the Cancer Genome Atlas \cite{CancerGenomeAtlasWebsite}, the Cancer Imaging Archive \cite{CancerImagingArchiveWebsite} and the Digital Pathology Association’s Whole-Slide Imaging Repository \cite{DAPAWebsite}, among others. Another alternative would be when facing internet glitches and the downloading of these images provided by these repositories. As they are large in size, many of the professionals in LMICs may find it difficult to obtain the data for the AI training. To solve this issue, Pathologists can photograph a region of interest for a particular pathology by using the data of their own patients. After taking the photograph, these images can be classified and uploaded in AI systems either open-source or commercially available, although the latter are usually expensive.
