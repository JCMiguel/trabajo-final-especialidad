\section{Conclusions and Future work} \label{section:en/conclusiones}
% En las conclusiones intenta de resumir los desafíos actuales y, si posible, da pistas de como se pueden resolver con tendencias actuales. Por ejemplo, mencionaste en la primera página lo del “seek to bond with the systems as if they were colleagues”, pues eso a lo mejor se puede relacionar también con todo el área de LLMs.

Pese a que la implementación de AI/xAI es prometedora, en este artículo se evidencian los desafíos que se deben resolver para que una implementación efectiva de estas tecnologías en histopatología. Estas dificultades no se limitan a lo técnico, sino que por el contrario abarcan perspectivas igual de importantes.

La integración al flujo de trabajo es el principal factor que se debe considerar durante el diseño de un AI system. Es necesario elicitar una lista requisitos con especialistas y patólogos a fin de comprender sus necesidades y determinar cuál es la forma en que ellos pueden beneficiarse de estas tecnologías. Los sistemas AI/xAI deben ser concebidos como un asistente para los patólogos y disponer de transparencia suficiente para poder supervisar sus resultados.
Es probable que para lograr una integración adecuada con el flujo de trabajo sea preciso incorporar el enfoque expert-in-the-loop. Otra posibilidad es plantear los AI/xAI algorithms como componentes de un sistema màs complejo que implemente pieces of traditional programming y resultados expresados in natural language.

Los sistemas AI/xAI son altamente sensibles a la calidad de sus datos. Es preciso contar con muestras clasificadas adecuadamente para que el entrenamiento del AI model sea exitoso. Dado que la clasificación es una tarea exhaustiva, es posible que este proceso pueda agilizarse recurriendo a weakly supervised learning methods. However, obtain a adecuada classification in oncology and renal histopathology sometimes is complex. Frente a esta dificultad, diseñar AI systems desde en unfoque centrado en causality instead of explainability podría ser un gran aporte.

%Los AI/xAI systems pueden beneficiarse de la correlación entre datasets y anonimazed patient-related data, ya que esto brinda un contexto mayor obtener un diagnóstico. Sin embargo, es preciso contar con regulaciones que garanticen la privacidad de estos datos y que solo puedan utilizarse con el consentimiento explícito de los pacientes.

Es preciso contar con regulaciones que sienten un estándar respecto al testing y uso de AI systems en histopatología. En caso de que estos sistemas requieran el uso de patient-related data for training, 

El costo de implementación es un factor limitante en algunas organizaciones. Frente a esto, las alternativas open-source pueden ser el punto de partida para implementar AI-based solutions.


As for the next steps of this article, this work will be expanded to create a systematic mapping study of literature. It is expected for this to be the foundations of a future investigation. 