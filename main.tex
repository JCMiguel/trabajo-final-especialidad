%%%%%%%%%%%%%%%%%%%%%%%%%%%%%%%%%%%%%%%%%%%%%%%%%%%%%%%%%%
%%                                                      %%
%%        ARTICULO CON FORMATO GENÈRICO DE ARTÌCULO     %%
%%                                                      %%
%%%%%%%%%%%%%%%%%%%%%%%%%%%%%%%%%%%%%%%%%%%%%%%%%%%%%%%%%%
%%                                                      %%
%%   Si hace falta otro formato, generar otro archivo   %%
%%  main_format.tex y compilar ese archivo.             %%
%%                                                      %%
%%%%%%%%%%%%%%%%%%%%%%%%%%%%%%%%%%%%%%%%%%%%%%%%%%%%%%%%%%
\documentclass[hidelinks,12pt,twoside]{article}
%\documentclass[hidelinks,12pt,oneside]{article}

%%%%%%% DEPENDENCIAS %%%%%%%
\usepackage{svg}
\usepackage{amsmath,amssymb,amsfonts}
\usepackage{algorithmic}
\usepackage{graphicx}
\usepackage{textcomp}
\usepackage{xcolor}
%\usepackage{orcidlink}
\usepackage{hyperref}
%\usepackage{graphicx}
\usepackage{xurl}
%\usepackage{breakurl}
%\def\UrlBreaks{\do\/\do-}

% Para que las secciones siempre empiecen en la hoja par
\let\oldsection\section
\def\section{\clearevenpage\oldsection}

\makeatletter
\def\clearevenpage{\clearpage\if@twoside \ifodd\c@page
    \hbox{}\newpage\if@twocolumn\hbox{}\newpage\fi\fi\fi}
\makeatother

% Separación de párrafos
\usepackage[skip=10pt plus1pt, indent=16pt]{parskip}

% Encoding
\usepackage[latin1,utf8]{inputenc}

%%%%%%
%%   BIBLIOGRAFÍA
%%%%%%
\usepackage[style=ieee]{biblatex} %Imports biblatex package
\addbibresource{referencias.bib} %Import the bibliography file
\usepackage{url}
\appto{\bibsetup}{\sloppy}
\setcounter{biburllcpenalty}{7000}
\setcounter{biburlucpenalty}{8000}
% Red: https://www.overleaf.com/learn/latex/Bibliography_management_in_LaTeX

% Spanish-specific commands
% Esto ajusta el texto de las \label a idioma español
\usepackage[spanish, mexico]{babel}
\usepackage{csquotes}

% ESTO LO PUSE PARA PROBAR
%\usepackage{caption}
%\captionsetup{font=scriptsize} % This will fix captions being resized

%%%%%%
%%   LISTADO DE CADENAS DE BÚSQUEDA
%%%%%%
\usepackage{listings} %Esto para las cadenas de búsqueda
% Ref: https://www.overleaf.com/learn/latex/Code_listing#Reference_guide
%\definecolor{codegreen}{rgb}{0,0.6,0}
%\definecolor{codegray}{rgb}{0.5,0.5,0.5}
%\definecolor{codepurple}{rgb}{0.58,0,0.82}
%\definecolor{backcolour}{rgb}{0.95,0.95,0.92}
\lstdefinestyle{cadena_busqueda_style}{
    %backgroundcolor=\color{backcolour},   
    %commentstyle=\color{codegreen},
    %keywordstyle=\color{magenta},
    %numberstyle=\tiny\color{codegray},
    %stringstyle=\color{codepurple},
    basicstyle=\ttfamily\footnotesize,
    breakatwhitespace=false,         
    breaklines=true,                 
    captionpos=b,                    
    keepspaces=true,                 
    %numbers=left,                    
    numbersep=5pt,                  
    showspaces=false,                
    showstringspaces=false,
    showtabs=false,                  
    tabsize=2
}
\lstset{style=cadena_busqueda_style}
% TODO: Ver cómo centrar los listings
% Ref: https://tex.stackexchange.com/questions/5818/how-to-center-a-listing


%%%%%%
%%  PARA EL FORMATO DE LAS TABLAS
% Ref: https://www.overleaf.com/learn/latex/Tables#Introduction
%%%%%%
\usepackage{array}
\usepackage{longtable}
\makeatletter
\newbox\LT@firstfoot
\def\endfirstfoot{\LT@end@hd@ft\LT@firstfoot}
\newdimen\LT@footdiff
\def\LT@start{%
  \let\LT@start\endgraf
  \endgraf\penalty\z@
  \vskip\LTpre\endgraf
  \LT@footdiff-\ht\LT@foot
  \advance\LT@footdiff\ht\LT@firstfoot
  \dimen@\pagetotal
  \advance\dimen@ \ht\ifvoid\LT@firsthead\LT@head\else\LT@firsthead\fi
  \advance\dimen@ \dp\ifvoid\LT@firsthead\LT@head\else\LT@firsthead\fi
  \advance\dimen@ \ht\ifvoid\LT@firstfoot\LT@foot\else\LT@firstfoot\fi
  \dimen@ii\vfuzz
  \vfuzz\maxdimen
  \setbox\tw@\copy\z@
  \setbox\tw@\vsplit\tw@ to \ht\@arstrutbox
  \setbox\tw@\vbox{\unvbox\tw@}%
  \vfuzz\dimen@ii
  \advance\dimen@ \ht
      \ifdim\ht\@arstrutbox>\ht\tw@\@arstrutbox\else\tw@\fi
  \advance\dimen@\dp
      \ifdim\dp\@arstrutbox>\dp\tw@\@arstrutbox\else\tw@\fi
  \advance\dimen@ -\pagegoal
  \ifdim \dimen@>\z@\vfil\break\fi
  \global\@colroom\@colht
  \ifvoid\LT@firstfoot
    \ifvoid\LT@foot
    \else
      \advance\vsize-\ht\LT@foot
      \global\advance\@colroom-\ht\LT@foot
      \dimen@\pagegoal\advance\dimen@-\ht\LT@foot\pagegoal\dimen@
      \maxdepth\z@
    \fi
  \else
    \advance\vsize-\ht\LT@firstfoot
    \global\advance\@colroom-\ht\LT@firstfoot
    \dimen@\pagegoal\advance\dimen@-\ht\LT@firstfoot\pagegoal\dimen@
    \maxdepth\z@
  \fi
  \ifvoid\LT@firsthead\copy\LT@head\else\box\LT@firsthead\fi\nobreak
  \output{\LT@output}%
}
\def\LT@output{%
  \ifnum\outputpenalty <-\@Mi
    \ifnum\outputpenalty > -\LT@end@pen
      \LT@err{floats and marginpars not allowed in a longtable}\@ehc
    \else
      \setbox\z@\vbox{\unvbox\@cclv}%
      \ifdim \ht\LT@lastfoot>\ht\LT@foot
        \dimen@\pagegoal
        \advance\dimen@-\ht\LT@lastfoot
        \ifdim\dimen@<\ht\z@
          \setbox\@cclv\vbox{\unvbox\z@\copy\LT@foot\vss}%
          \@makecol
          \@outputpage
          \setbox\z@\vbox{\box\LT@head}%
        \fi
      \fi  
      \global\@colroom\@colht
      \global\vsize\@colht   
      \vbox
        {%
          \unvbox\z@
          \box
            \ifvoid\LT@lastfoot
              \ifvoid\LT@firstfoot
                \LT@foot
              \else
                \LT@firstfoot
              \fi
            \else
              \LT@lastfoot
            \fi
        }%
    \fi
  \else
    \ifvoid\LT@firstfoot
      \setbox\@cclv\vbox{\unvbox\@cclv\copy\LT@foot\vss}%
      \@makecol
      \@outputpage
      \global\vsize\@colroom
    \else
      \setbox\@cclv\vbox{\unvbox\@cclv\box\LT@firstfoot\vss}%
      \@makecol
      \@outputpage
      \global\advance\@colroom\LT@footdiff
      \global\vsize\@colroom
    \fi
    \copy\LT@head\nobreak
  \fi
}
\makeatother
\usepackage{tabularx}
    \newcolumntype{L}{>{\raggedright\arraybackslash}m}
    %\newcolumntype{M}[1]{>{\centering\arraybackslash}m{#1}}
    \newcolumntype{M}{>{\centering\arraybackslash}m}
    \newcolumntype{C}{>{\centering\arraybackslash}p}

%%%%%% COMANDO PARA COMENTAR BLOQUES DE CÒDIGO %%%%%%
\newcommand{\commentcode}[1]{}



%%%%%%%%%%%%%%%%%%%%%%%%%%%%%%%%%%%%%%%%%%%%%%%%%%%%%%
%   FORMATO DEL DOCUMENTO
%%%%%%%%%%%%%%%%%%%%%%%%%%%%%%%%%%%%%%%%%%%%%%%%%%%%%%
\usepackage[a4paper,width=150mm,top=25mm,bottom=25mm]{geometry}

\usepackage{fancyhdr}
\pagestyle{fancy}
\fancyhead{}
\fancyhead[RO,LE]{Inteligencia artificial explicable en Histopatología: revisión sistemática}
\fancyfoot{}
\fancyfoot[LE,RO]{\thepage}
\fancyfoot[LO,RE]{Ing. Miguel, Juan Cristian Daniel}

%\usepackage[nottoc,notlot,notlof]{tocbibind}
\usepackage[nottoc,numbib]{tocbibind}



%%%%%%%%%%%%%%%%%%%%%%%%%%%%%%%%%%%%%%%%%%%%%%%%%%%%%%
%   INICIO DEL DOCUMENTO
%%%%%%%%%%%%%%%%%%%%%%%%%%%%%%%%%%%%%%%%%%%%%%%%%%%%%%
\begin{document}

%%%%% CARATULA
\begin{titlepage}
    \begin{center}
        \includegraphics[width=\textwidth]{Imagenes/utnba-logo.png}
        \vspace*{4cm}
            
        
        \Large
        \textbf{TRABAJO FINAL INTEGRADOR}

        \vspace{0.7cm}
        \textbf{ESPECIALIZACIÓN EN INGENIERÍA EN SISTEMAS DE INFORMACIÓN}

        \large
        \vspace{0.7cm}
        \textbf{Título:}

        \textbf{\enquote{Requisitos y dificultades del uso de la inteligencia artificial explicable en histopatología: revisión sistemática de la literatura actual}}
            
        \vspace{0.7cm}
 
        \normalsize
        \textbf{Autor: }        
        \textbf{Ing. Miguel, Juan Cristian Daniel}

        \vspace{0.7cm}

        \textbf{Tutores:}

        \textbf{Dr. Grévisse, Christian}

        \textbf{Dra. Sardella, Antonia}

        \vspace{0.7cm}

        \textbf{Directora de carrera: }
        \textbf{Dra. Pollo-Cattaneo, Ma. Florencia}


        
        \vfill
        \large
        %ESCUELA DE POSGRADO \\
        %FACULTAD REGIONAL BUENOS AIRES \\
        %UNIVERSIDAD TECNOLÓGICA NACIONAL \\
        \vspace{0.8cm}
        Buenos Aires, diciembre de 2024

            
    \end{center}
\end{titlepage}

\tableofcontents
\newpage
\listoffigures
\newpage
\listoftables
\newpage

%%%%%%%%%%%%%%%%%%%%%%%%%%%%%%%%%%%%%%%%%%%%%%%%%%%%%%
%   INCLUSIÓN DE CONTENIDO
%%%%%%%%%%%%%%%%%%%%%%%%%%%%%%%%%%%%%%%%%%%%%%%%%%%%%%
% Referencia: \input vs \include
% https://tex.stackexchange.com/questions/246/when-should-i-use-input-vs-include
%\input{Secciones/es/00_Agradecimientos}
%%%%%%%%%%%%%%%%%%%%%%%%%%%%%%%%%%%%%%%%%%%%%%%%%%%%%%
%   INTRODUCCIÓN
%%%%%%%%%%%%%%%%%%%%%%%%%%%%%%%%%%%%%%%%%%%%%%%%%%%%%%
\section{Introducción} \label{section:es/introduccion}

% Acá tengo que hablar del tema IAX, mencionar brevemente la problemática actual y plantear mis objetivos para este artículo. Es decir, mencionar que voy a hacer un mapeo y tal.
% Esta secciòn es importante para indicar lo que va a contener el artículo, así que tengo que ser breve, pero sin ser redundante con lo que voy a comentar en la sección de estado del arte.

% TODO: Revisar y redactar mejor esta introducción
% Para redactar esta sección, puedo inspirarme un poco en estos artículos:
% 6-Tosun-xAI Anatomic Pathology.pdf

% TODO: este estado del arte debería arrancar primero con las nociones de AI y luego la aparición de xAI. TAmbién se pueden comentar los grandes conjuntos de AI que se usan en la actualidad y ejemplos de investigaciones donde se los utiliza.
% xAI vs Interpretable Machine Learning

% TODO: Tengo que hablar acá de los sinónimos que fui encontrando de IAX. Sirve como introducción también a la cadena de búsqueda.

% TODO: Justificar por qué este tema es importante. Por ejemplo: ¿hay pocos diagnósticos? ¿Se tarda mucho en diagnosticar? ¿Se suelen hacer mal? ¿Cuesta especializarse en determinados temas patológicos? Todo eso es importante porque le da solidez al tema.

% TODO: Acá se pueden comentar (objetivamente y citando fuentes) los beneficios de incorporar xAI en la pràctica médica. Tambièn se puede hacer un listado de controversias y principales dificultades (también citando fuentes)

La inteligencia artificial (IA) es una tecnología que se ha desarrollado mucho en los últimos años. Con el advenimiento del aprendizaje profundo o deep learning, cada vez son más los estudios y aplicaciones de estos modelos en distintos dominios. La mayoría de los algoritmo de IA se basan en modelos de caja negra, lo que implica que no existe una trazabilidad clara de cómo se obtuvo un resultado u otro. En términos generales, puede decirse que cuanto más complejo es el modelo de IA, más difícil es interpretar sus resultados. Un claro ejemplo de esto son las redes neuronales, usadas con frecuencia en la actualidad. Por otro lado, en contraposición a los modelos de caja negra existen los de caja blanca, que pretenden garantizar un entendimiento claro y directo del proceso \cite{Ye2021}.

%“While DL for medical imaging evolved from successful applications in computer vision, its use for sparse, tabular genomic data was less preva- lent, facing substantial competition from traditional bio- informatics tools. Another crucial point to consider is the human interpretability of histopathological images com- pared to genomic information. Te human eye can detect distinct patterns in histology, which form the basis for patient diagnosis, making it less abstract and more intui- tive than genomic data. As a refection of this, the feld of explainable AI is emerging, aiming to elucidate the black- box properties of DL models.”
Se define a la histopatología como el estudio de las células y el tejido enfermos a través un microscopio \cite{defHistopatologia}. Pese a los avances técnicos de los modelos de IA, actualmente aplicarlos en histopatología como parte de sistemas diagnóstico asistidos por computadora (o en inglés, Computer-Aided Diagnosis, CAD) aún es difícil debido a los requisitos específicos que se deben satisfacer en el ámbito médico, entre los cuales la falta de transparencia del proceso es el principal obstáculo \cite{Abdelsamea2022}. Otros requisitos que se deben satisfacer son de índole técnico, social, e incluso legal en términos regulatorios. Para solventar estos inconvenientes se desarrollaron un conjunto de técnicas que se conocen como inteligencia artificial explicable (IAX) \cite{Unger2024}, cuyo enfoque teórico es muy diverso. Estas técnicas consisten en una colección de procesos y métodos que permiten a los usuarios comprender cómo un algoritmo de IA obtiene un resultado específico \cite{Giuste2023}.

La Organización Mundial de la Salud estima que la cantidad de muertes por cáncer, proyectadas al año 2045, se incrementará un 73,6\% en promedio a nivel mundial \cite{OMSproy2045}. Por otro lado, la cantidad de patólogos al día de hoy no es muy alta en proporción a la población total \cite{OMSpatologosHoy}. Frente a esto, la implementación de IA podría contribuir a disminuir la carga excesiva de trabajo que afrontan los patólogos en la actualidad \cite{Liu2021}. No obstante, a pesar de que actualmente existen múltiples técnicas IAX, éstas no son suficientes aún para implementar modelos de IA en histopatología, dado que no existe aún un consenso claro en el área. Algunos investigadores sostienen que la forma de analizar el comportamiento de un algoritmo de IA no consiste en buscar explicaciones a sus predicciones, sino en determinar la presencia de sesgos al obtenerlas \cite{Ahmad2018}. Existe además el debate no resuelto acerca del uso de modelos IAX post-hoc (caja negra) frente a los ante-hoc. Las explicaciones obtenidas a partir de un modelo de caja negra no se corresponden con la forma real en que el modelo de IA predice sus resultados, hecho que genera escepticismo para incorporar estas tecnologías en escenarios donde la toma de decisiones es crítica \cite{Ahmad2018}. Por otro lado, los sistemas IA/IAX deben ser consistentes en sus resultados; es decir, debe ser posible reproducir o reconocer un resultado siempre de la misma forma.

El presente trabajo es una extensión de la revisión rápida publicada en por Miguel et al. \cite{Miguel2024}. Con el propósito de brindar un panorama más amplio, se realizó una revisión sistemática de literatura completa con el objetivo de reunir todos los requisitos, obstáculos y aportes existentes hasta la fecha, relacionados la implementación de IA/IAX en histopatología.

Este trabajo se estructura de la siguiente manera: 
La sección \ref{section:es/contexto-preliminar} presenta conceptos teóricos y contexto preliminar acerca de las tecnologías de IA e IAX.
La sección \ref{section:es/protocolo} describe el protocolo utilizado para el desarrollo de la revisión sistemática.
En la sección \ref{section:es/resultados}, se analizan los resultados obtenidos y se brindan respuestas a las pregunta de investigación, directrices de la revisión sistemática. Finalmente, en la sección \ref{section:es/conclusiones} se presentan las conclusiones de este trabajo.
\newpage
%%%%%%%%%%%%%%%%%%%%%%%%%%%%%%%%%%%%%%%%%%%%%%%%%%%%%%
%   CONTEXTO PRELIMINAR - ESTADO DEL ARTE
%%%%%%%%%%%%%%%%%%%%%%%%%%%%%%%%%%%%%%%%%%%%%%%%%%%%%%
\section{Contexto preliminar} \label{section:es/contexto-preliminar}

Durante la última década la inteligencia artificial fue foco de investigación y desarrollo. En particular en medicina, según una estadística de la Universidad de Stanford en 2022 el FDA (Food and Drugs Administration) aprobó un 12,1\% más de equipamiento médico basado en IA respecto al año 2021. Históricamente esta tasa fue creciendo paulatinamente, según se muestra en la Figura \ref{figura:es/stanford_fda_approved}.

\begin{figure}[htbp]
\centerline{\includegraphics[width=\textwidth]{Imagenes/stanford_fda_approved.png}}
\caption{Número de dispositivos médicos aprobados por la FDA. Gráfico extraído del AI Index, de la Stanford University, Chapter 5 \cite{AIIndexStanfordWebsite}.}
\label{figura:es/stanford_fda_approved}
\end{figure}

No obstante esto, la opinión pública de los profesionales respecto a la integración de la IA en sus puestos de trabajo está dividida entre quienes lo consideran un beneficio y los que lo ven como una amenaza, hecho reflejado en la Figura \ref{figura:es/stanford_global_opinion}. Esto bien puede estar fundamentado en la falta de transparencia en la mayoría de los algoritmos de IA populares actualmente. La Figura \ref{figura:es/stanford_transparency} refleja este hecho. En ella se muestra el puntaje FMTI (Foundation Model Transparency Index) para algunos IA actuales de usos varios \cite{TMFIWebsite}. En términos generales, cuanto más alto es este valor, más transparente es el modelo. 

\begin{figure}[htbp]
\centerline{\includegraphics[width=\textwidth]{Imagenes/stanford_global_opinion.png}}
\caption{Opinión pública acerca de la integración de la IA en puestos de trabajo. Gráfico extraído del AI Index, de la Stanford University, Chapter 9 \cite{AIIndexStanfordWebsite}.}
\label{figura:es/stanford_global_opinion}
\end{figure}

\begin{figure}[htbp]
\centerline{\includegraphics[width=\textwidth]{Imagenes/stanford_transparency.png}}
\caption{Índices de transparencia FMTI, evaluados en algunos de los modelos de inteligencia artificial actuales. Gráfico extraído del AI Index, de la Stanford University, Chapter 3 \cite{AIIndexStanfordWebsite}.}
\label{figura:es/stanford_transparency}
\end{figure}

Por todo lo dicho en los párrafos precedentes, es pertinente brindar un breve marco teórico que sirva de sustento a la revisión sistemática presentada en este este trabajo. Se comienza con una introducción conceptual teórica acerca de la inteligencia artificial en el apartado \ref{section:es/nociones-ia}. En la sección \ref{section:es/nociones-xai}, se presentan los orígenes de IAX y el estado del arte actual. En la sección \ref{section:es/taxonomía-xai} se expone una posible taxonomía para todas las técnicas IAX existentes hasta la fecha. La sección \ref{section:es/uso-ia-medicina} reúne una serie de ejemplos de implementaciones de AI y IAX en medicina. Finalmente, en la sección \ref{section:es/lineamientos-tfe} se introducen las adopciones, criterios y lineamientos utilizados durante la revisión sistemática. Se añade además en el apéndice \ref{appendix:es/glosario} un glosario complementario que reúne términos de interés que brindan mayor sustento a los resultados de la revisión sistemática.

%TODO: Hablar de los trabajos de Montavon, Holzinger, Lipton, Guidoti, Samek y Roscher.
%TODO del TODO: Necesito repasar qué hicieron estos tipos...

%Temario posible para estado del arte o contexto preliminar
%- Nociones de Inteligencia Artificial
%	- Clasificaciones
%- Nociones de xAI
%	- Nomenclatura habitual (interpretability, trustworthy, explainability, causality, etc.)
%- Taxonomía de xAI
%- Usos de Inteligencia artificial en medicina
%	- Promesas o ventajas tentativas
%	- Diagnóstico diferencial
%	- Aprendizaje Federado
%- LLM en medicina (no sé bien dónde poner esto, pero siento que debería introducirlo)
%
%- Lineamientos y adopciones para este trabajo
%	- esclarecer porqué uso la sigla AI/xAI para referirme a AI y XAI en simultáneo o ambas por separado.
%	- Tengo que añadir una sección de conceptos importantes para informar la manera en que traduje cada término en inglés. ¿O debería dejarlos en inglés?
% Màs ideas en la parte de abajo...

% FUENTES: (¿Por qué anoté esto acá?)
% 1-Journal-Explainable AI Healthcare.pdf
% 4-Ayorinde-AI Renal Histopathology.pdf ???
% 5-Liu-AI in Medicine.pdf
% 6-Tosun-xAI Anatomic Pathology.pdf
% 7-Jarrahi-Workflow pathologists-IA.pdf
% 8-Kiehl-Computational Pathology.pdf !!!!
% 10-Verma-Rethinking Role AI.pdf










  %%%%%%%%%%%%%%%%%%%%%%%%%%%%%%%%%%%%%%%%%%%%%%%%%%%%%%
%   NOCIONES DE INTELIGENCIA ARTIFICIAL
%%%%%%%%%%%%%%%%%%%%%%%%%%%%%%%%%%%%%%%%%%%%%%%%%%%%%%
\subsection{Nociones de inteligencia artificial} \label{section:es/nociones-ia}

%- Nociones de Inteligencia Artificial
%	- Clasificaciones

% TODO: este estado del arte debería arrancar primero con las nociones de AI y luego la aparición de xAI. TAmbién se pueden comentar los grandes conjuntos de AI que se usan en la actualidad y ejemplos de investigaciones donde se los utiliza.

% "Red conceptual de conceptos importantes para el tema del artículo. La red muestra la integridad y unión entre todos los conceptos". Sugerencia de Pollo
% TODO: Tengo que ver cómo hago esto...

A lo largo de los años, definir la inteligencia artificial y trazar una línea que la diferencie de un algoritmo de procesamiento convencional fue una tarea compleja. Actualmente, la Real Academia Española define la inteligencia artificial como aquella \enquote{disciplina científica que se ocupa de crear programas informáticos que ejecutan operaciones comparables a las que realiza la mente humana, como el aprendizaje o el razonamiento lógico} \cite{defIA-RAE}. Por su parte, según la documentación de Google Cloud \enquote{la inteligencia artificial es un campo de la ciencia relacionado con la creación de computadoras y máquinas que pueden razonar, aprender y actuar de una manera que normalmente requeriría inteligencia humana o que involucra datos cuya escala excede lo que los humanos pueden analizar} \cite{defIA-Google}.

Pese a su popularidad reciente, sus orígenes se remontan hace más de 70 años, cuando el matemático e informático Alan Turing, considerado hoy como el padre de la inteligencia artificial, habló de ella en un artículo titulado \enquote{Computing Machinery and Intelligence}, publicado en 1950. En su trabajo, Turing postuló un análisis acerca de si las máquinas pueden pensar y las ambigüedades que acarrea esta pregunta. Para solventar esta disyuntiva, propuso lo que denominó como \enquote{el juego de la imitación} en el que dos participantes, que son una persona y una máquina, responden las preguntas de un interrogador. Este último se ubica en una sala separada y su objetivo es identificar, mediante texto, quién está respondiendo: si la máquina o si la persona. Si el interrogador no es capaz de diferenciar entre ambos, entonces se concluye que la máquina es inteligente. A este experimento actualmente se lo conoce como \enquote{prueba de Turing}, aunque no es el nombre original acuñado por el autor \cite{Turing1950}.

No obstante, el término \enquote{inteligencia artificial} fue acuñado por primera vez en el año 1956, por el informático John McCarthy en la Conferencia de Dartmouth. En dicha conferencia también se concibieron las siete cuestiones fundacionales de la IA, las cuales definieron los retos principales a abordar en el desarrollo de la inteligencia artificial, en términos de capacidad de procesamiento, lenguaje, capacidad de abstracción, eficiencia, entre otras \cite{NuriaOliver2020}.

Existen múltiples propuestas acerca de los tipos de inteligencia artificial existentes. En forma simplificada, en relación con los algoritmos de aprendizaje automático es posible decir que la inteligencia artificial es un superconjunto que los contiene, según se observa en la Figura \ref{figura:es/tipos-ai-ml}. Sin embargo, la clasificación propuesta por Arend Hintze es más abarcativa. Según el autor, la inteligencia artificial se puede subdividir en cuatro clases:

\begin{figure}[ht!]
    \centerline{\includegraphics[width=\textwidth]{Imagenes/Conceptos-IA-ML-DL.png}}
    \caption{Clasificación de alto nivel de inteligencia artificial en términos de algoritmos de aprendizaje automático \cite{tiposIA-ML-DL}.}
    \label{figura:es/tipos-ai-ml}
    \end{figure}

\begin{itemize}
    \item Tipo I - Máquinas reactivas: se trata de sistemas de IAs cuyo funcionamiento se limita a reaccionar a un estímulo y, por lo tanto, carece de memoria o de la capacidad de aprender datos nuevos. Se basan en reglas preprogramadas. El ejemplo más popular de este tipo de IA es Deep Blue, un sistema creado por IBM a finales de los años 90 para vencer al maestro ajedrecista Garry Kasparov \cite{tiposIA-Hintze}.
    \item Tipo II - Memoria limitada: son aquellos sistemas de IA que son capaces de tomar decisiones en base a datos del pasado. Se considera que la gran mayoría de los algoritmos de IA modernos pertenecen a este grupo. Un ejemplo de esta clase de sistemas son los modelos de aprendizaje profundo \cite{defIA-Google}.
    \item Tipo III - Teoría de la mente: las máquinas de este tipo son más avanzadas y no solo elaboran representaciones de su contexto, sino también sobre otros agentes o entidades de dicho contexto. Hinze destaca que en psicología esto se conoce como \enquote{teoría de la mente}, y se refiere a comprender que las personas, las criaturas y los objetos del mundo pueden tener pensamientos y/o emociones que condicionen su propio comportamiento \cite{tiposIA-Hintze}. Según Google, estas IAs aún son teóricas y están sujetas a investigación \cite{defIA-Google}.
    \item Tipo IV - Autoconocimiento: Se trata de sistemas que son capaces, no solo de respetar la teoría de la mente, sino que también son autoconcientes de su propia existencia y tiene las capacidades intelectuales y emocionales de un ser humano. La IA con autoconciencia no existe en la actualidad \cite{defIA-Google} \cite{tiposIA-Hintze}.
\end{itemize}

Actualmente, la IA se está implementando en una gran variedad de campos y disciplinas, por ejemplo en estadística, ingeniería de hardware y software, lingüística, neurociencia, filosofía y psicología. También se utiliza ampliamente en el ámbito empresarial para generar predicciones a partir de un conjunto de datos, procesar lenguaje natural, elaborar recomendaciones, entre otros usos \cite{defIA-Google}. A lo largo de este trabajo, se profundizará en aplicaciones en medicina, específicamente en los campos de la histología y la histopatología.



  %%%%%%%%%%%%%%%%%%%%%%%%%%%%%%%%%%%%%%%%%%%%%%%%%%%%%%
%   NOCIONES DE INTELIGENCIA ARTIFICIAL EXPLICABLE
%%%%%%%%%%%%%%%%%%%%%%%%%%%%%%%%%%%%%%%%%%%%%%%%%%%%%%
\subsection{Nociones de inteligencia artificial explicable} \label{section:es/nociones-xai}

%- Nociones de xAI
%	- Nomenclatura habitual (interpretability, trustworthy, explainability, causality, etc.)


La inteligencia artificial explicable, o IAX, es una colección de procesos, técnicas y métodos que permiten a los usuarios comprender y confiar en los resultados de los algoritmos de inteligencia artificial \cite{Giuste2023}. Las técnicas de IAX pretenden mejorar la transparencia de los modelos de IA, una característica de vital importancia en el campo de la medicina, dado que permite una mayor evidencia cualitativa de las predicciones elaboradas por el algoritmo.

El término \enquote{inteligencia artificial explicable} fue introducido por primera vez en el año 2004 \cite{Patricio2023}. Ocasionalmente también se le ha referido como \enquote{inteligencia artificial interpretable} o \enquote{inteligencia artificial de caja blanca}. Tras estos primeros acercamientos, continuaron años de profundo crecimiento en temas afines a big data, análisis de datos (data analytics), aprendizaje automático y aprendizaje profundo, que ocasionaron que en el año 2015 se planteara la necesidad del programa Explainable AI, impulsado por la Defense Advanced Research Projects Agency (DARPA). Este programa, de 48 meses de duración, comenzó efectivamente en el año 2017 y buscó garantizar a los usuarios un mejor entendimiento de las decisiones de los sistemas de inteligencia artificial \cite{Gunning2021}. Desde entonces, diversos autores se han esforzado por intentar estandarizar los fundamentos IAX.

Con el paso del tiempo se introdujo nomenclatura clave para caracterizar ciertas características de las técnicas IAX. Algunos de estos conceptos se utilizan a menudo en forma intercambiable, como si de sinónimos se tratase. Un ejemplo de esto son los calificativos \enquote{explicable}, \enquote{interpretable} y \enquote{transparente}. Esta situación es alarmante dado que no existe un contexto uniforme acerca de qué terminología utilizar. Graziani et al. analizan esta situación y demuestran las múltiples acepciones que tiene cada término en ámbitos distintos y hasta complementarios \cite{Graziani2022}. Dado este problema de definición según el dominio, algunos autores prefieren diferenciar el vocabulario. La Tabla \ref{tabla:es/terminos_xai} reúne algunos de los términos frecuentes en temas afines a IAX, traducidos al español. Los conceptos y sus definiciones fueron extraídos de la revisión sistemática de Salih et al. \cite{Salih2024}, el análisis de causabilidad de Holzinger y Müller \cite{Holzinger2021} y el estudio de Leventi-Peetz y Östreich acerca de la reproducibilidad en el aprendizaje automático \cite{LeventiPeetz2022}. En dicha tabla, la palabra \enquote{sistema} se refiere específicamente a sistemas de IA o de IAX. Se seleccionaron las palabras más aproximadas según el contexto para traducir los términos en inglés.

{
\begin{table}[ht!]
\scriptsize
\centering
\begin{tabular}{|| m{7.5em} | m{7.5em} | m{25em} ||} 
 \hline
  \textbf{Término propuesto en español} & \textbf{Término original en inglés} & \textbf{Definición} \\ [0.5ex] 
  \hline \hline
Explicabilidad & Explainability
 & Se refiere a la habilidad de comprender el comportamiento y la composición interna de un sistema, y poder explicar por qué realiza una determinada acción.
 % & Refer to the ability of understanding the internal mechanism and the behavior of a system and to explain why a specifc action was made (Salih et al. 2023b)
 \\ \hline
Interpretabilidad & Interpretability
 & Refleja cuán comprensible es la salida de un modelo o sistema, concebida desde una perspectiva humana.
 % & Reflects the extent to which degree that the model’s output is understandable from human prospective (Salih et al. 2023b)
 \\ \hline
Transparencia & Transparency
 & La cualidad de un modelo de ser intrínsecamente comprensible. Es la característica intrínseca a los modelos de caja blanca, y opuesta a los de caja negra.
 % & Opposite to black box and has the potential to be understandable by itself (Linardatos et al. 2020)
 \\ \hline
Transferibilidad & Transferability
 & Comprender un sistema de manera tal que pueda ser extendido o transferido a otro dominio o problema.
 % & Understanding an AI system in a way can be extended or transferred into another domain and problem (Arrieta et al. 2020)
 \\ \hline
Confiabilidad & Trustworthy
 & Característica que implica que un sistema sea transparente, seguro y confiable al mismo tiempo.
 % & That the system is transparent, safe and the output can be trusted (Arrieta et al. 2020)
 \\ \hline
Fidelidad & Fidelity
 & Grado de exactitud en que la explicación representa o captura el funcionamiento del sistema.
 % & To what extend does the explanation represent and capture the workings of the AI system? (Lopes et al. 2022)
 \\ \hline
Imparcialidad & Fairness
 & Implica que el sistema no exhiba prejuicios en contra de un grupo de individuos, en función de características inherentes a ellos.
 % & The AI system’s decisions do not exhibit prejudice against any group or individual based on inherent characteristics (Linardatos et al. 2020)
 \\ \hline
Responsabilidad & Accountability
 & Se refiere al aseguramiento de que un sistema es confiable y que funciona tal y como se lo expone.
 % & The assurance that the AI system can be trusted and works as was presented (Novelli et al. 2023)
 \\ \hline
Causalidad & Causality
 & Se define como la relación entre algo que sucede y la causa que lo genera \cite{defCausalityOxford}. Se la define también como la ley en virtud de la cual se producen efectos \cite{defCausalidadRAE}.
 % & Relying solely on statistical correlations can be very dangerous, especially in medicine, because correlation must not be confused with causality, which is completely missing in current AI.
 \\ \hline
Causabilidad & Causability
 & Medida del grado en que la explicación de una afirmación, dirigida a un experto humano, logra un nivel específico de comprensión causal con eficacia, eficiencia y satisfacción en un contexto específico de uso.
 % & Causability: Causability is the measurable extent to which an explanation of a statement to a human expert achieves a specified level of causal under- standing with effectiveness, efficiency, and satisfaction in a specified context of use. As causability is measured in terms of effectiveness, efficiency, and (human) satisfaction related to causal understanding and its transparency for an expert user, it refers to a human-understand- able model.
 \\ \hline
Reproducibilidad & Reproducibility
 & Se refiere a la habilidad de repetir resultados anteriores usando los mismos medios que se usaron en el ensayo original: el mismo software, los mismos datos de entrada, etc. Se considera la reproducibilidad como un requisito para establecer la causalidad en la interpretación de resultados.
 % Reproducibility refers to the ability to duplicate prior results using the same means as used in the original work, for example the same program code and raw data.
 \\ \hline
\end{tabular}
\caption{Definiciones para la terminología frecuente en temas afines a sistemas IA y IAX.}
\label{tabla:es/terminos_xai}
\end{table}
}

  %%%%%%%%%%%%%%%%%%%%%%%%%%%%%%%%%%%%%%%%%%%%%%%%%%%%%%
%   TAXONOMÍA DE INTELIGENCIA ARTIFICIAL EXPLICABLE
%%%%%%%%%%%%%%%%%%%%%%%%%%%%%%%%%%%%%%%%%%%%%%%%%%%%%%
\subsection{Taxonomía de inteligencia artificial explicable} \label{section:es/taxonomía-xai}

% Obtener contexto de los artículos que fui encontrando sobre Taxonomía de IAX

% Fuentes:
% - 106-4_Graziani_Taxonomy Interpretable AI
% - 72-Ortigossa-xAI Theory Applications
% - 60-Patrício-xDL Medical Image Survey
% - 108-Yang-Human Centric xAI Survey

La Real Academia Española define a la taxonomía como una \enquote{ciencia que trata de los principios, métodos y fines de la clasificación} \cite{defTaxonomiaRAE}. Contar con taxonomías es crucial para establecer los cimientos del estudio de una disciplina o tecnología, a la vez que permite administrar un lenguaje común entre profesionales. Patrício et al. proponen, como resultado de su revisión sistemática, una taxonomía de IAX, en la cual se la concibe desde múltiples perspectivas. La Figura \ref{figura:es/taxonomia_xai} muestra un diagrama adaptado y traducido del original propuesto por los autores. Se conciben cuatro formas de clasificar las técnicas de IAX: según su grado de reutilización, según el alcance de la explicación, según su origen y según la modalidad de la explicación. En esta sección se detallan cada una de estas cuatro concepciones y se brindan ejemplos.

\begin{figure}[h]
    \centering
    \includesvg[inkscapelatex=false, width=\textwidth]{Imagenes/taxo-iax.svg}
    \caption{Diagrama de taxonomía de IAX, adaptado del original propuesto en inglés por Patrício et al. \cite{Patricio2023}}
    \label{figura:es/taxonomia_xai}
\end{figure}


\subsubsection{Clasificación según grado de reutilización} \label{section:es/taxonomía-xai/reutilizacion}

Esta categoría se refiere a cuán aplicable es la técnica y su validez cuando se utiliza para explicar modelos de IA diferentes. Se distinguen en esta categoría dos grupos bien definidos:

\begin{itemize}
    \item Técnicas agnósticas del modelo: esta clase de métodos se pueden utilizar para explicar cualquier tipo de modelo de IA. Es decir, que no se limitan a una arquitectura ni topología específica. Ejemplos de estos son las técnicas LIME \cite{LIME2016} y el valor de SHAP \cite{SHAP2017}.
    \item Técnicas específicas de un modelo: este tipo de métodos son de uso restringido para una arquitectura específica de modelo de IA, dado que necesitan poder acceder a datos intrínsecos de este, como por ejemplo parámetros internos de una red neuronal. Un ejemplo de esta clase son las técnicas CAM \cite{CAM2023}, dado que solo son aplicables a redes neuronales con capas convolucionales en su arquitectura.
\end{itemize}

%Según su reusabilidad 
%Model-Agnostic versus Model-Specific. A distinguishing factor between interpretability approaches is their comprehensiveness regarding the models they can be applied to. Model-agnostic methods can be used to explain arbitrary models, not being limited to a specific model architecture. Conversely, model-specific methods are restricted to a specific model architecture, meaning that these methods require access to the model’s internal information.


\subsubsection{Clasificación según alcance de la explicación} \label{section:es/taxonomía-xai/alcance}

Esta clasificación se refiere al tipo de explicaciones provistas por el método IAX, es decir, qué tan abarcativo es el alcance de dicha explicación en relación a sus datos. Se distinguen dos grupos:

\begin{itemize}
    \item Técnicas de interpretabilidad global: son aquellas cuyas explicaciones presentan información del funcionamiento del modelo IA para todo el set de datos utilizado. Estos métodos permiten identificar qué patrones o características en los datos son las más influyentes en las predicciones del modelo, a la vez que revelan información acerca de cómo está aprendiendo el modelo. Un ejemplo de este tipo de técnicas es TCAV (Testing with Concept Activation Vectors) \cite{TCAV2018}.
    \item Técnicas de interpretabilidad local: son aquellas que se especializan en explicar predicciones individuales para ciertos datos de entrada. Permiten explicar cómo el modelo alcanza un resultado particular para una entrada específica. Un ejemplo de este tipo de métodos es LIME \cite{LIME2016} .
\end{itemize}

%¿De qué diablos depende la explicabilidad local y global? -- Del alcance de la explicación en relación a los datos.

%Según el alcance de la explicación
%Global Interpretability versus Local Interpretability. The type of explanations provided by XAI methods can be broadly divided into global and local according to whether the explanations provide insights about the model functioning for the general data distribution or for a specific data sample, respectively. Global interpretability methods explain which patterns in the data, i.e., class features, contributed the most to the model’s prediction. These explanations can reveal critical reasoning about what the model is learning. On the other hand, local interpretability methods seek to explain why a model performs a specific prediction for a single input.
%Local interpretability focuses on explaining individual predictions and helps show why the model reached a particular result. Global interpretability aims to understand the model's behavior across the entire dataset, showing its overall patterns and trends.

%Local Ejemplos de estos son las técnicas LIME \cite{LIME2016} y el valor de SHAP \cite{SHAP2017}.


\subsubsection{Clasificación según el origen de la explicación} \label{section:es/taxonomía-xai/naturaleza}

Esta clasificación se refiere a la naturaleza de la explicación, en términos de si es obtenida directamente del modelo IA o si se construye como resultado de una interacción con este. Se diferencian dos tipos, que son:

\begin{itemize}
    \item Explicaciones post-hoc: son aquellas que se obtienen por fuera del modelo de IA que se intenta interpretar. Generalmente, los métodos post-hoc se basan generar perturbaciones en los datos de entrada al modelo, una caja negra ya entrenada, con el objetivo de comprender el aporte parcial de determinadas características en la predicción del modelo. Un ejemplo de este tipo de explicaciones son las obtenidas por medio de LIME \cite{LIME2016}.
    \item Explicaciones intrínsecas: este tipo de explicaciones residen en la arquitectura interna del modelo. Es decir, que se obtienen a partir de modelos IA que son interpretables persé. A este tipo de explicaciones también se las conoce como ante-hoc \cite{Retzlaff2024}. Un ejemplo de este caso son los algoritmos de árboles de decisiones, los cuales son intrínsecamente interpretables.
\end{itemize}

% ¿Intrínseca es lo mismo que ante-hoc? Sí, y tengo una fuente que lo demuestra \cite{Retzlaff2024}.

%Según naturaleza de la explicación
%Post-hoc versus Intrinsic. This criterion distinguishes the methods with respect to whether the explanation mechanism lies in the internal architecture of the model (intrinsic) or if it is applied after the learning/development of the model (post-hoc). Post-hoc methods usually operate by per- turbing parts of the data so that they can understand the contribution of different features in the model prediction or by analytically determining the contribution of different features to the model prediction. On the other hand, intrinsic models, also known asin-model approaches orinherently in- terpretable models, are self-explainable since they are designed to produce human-understandable representations from the internal model features.

% Extracto de Retzlaff
%The main distinction which is central to this article is between post-hoc and ante-hoc xAI methods. They are distinguished based on whether a model is intrinsically explainable (ante-hoc), or whether explainability is achieved by xAI approaches that analyze the model after training (post-hoc) (Arrieta et al., 2020, Molnar, 2022).


\subsubsection{Clasificación según modalidad de explicación} \label{section:es/taxonomía-xai/modalidad}

%Modalidad de explicación. La modalidad de explicación se refiere al tipo de explicación que proporciona cada método de interpretación. Entre los métodos analizados, la explicación puede proporcionarse en forma de mapas de prominencia (Explicación por atribución de características), descripciones semánticas (Explicación por texto), ejemplos similares (Explicación por ejemplos) o utilizando conceptos de alto nivel (Explicación por conceptos)

Esta categoría se refiere a la forma en que se presenta una explicación al modelo. Patrício et al. distinguen cuatro formas fundamentales de presentar esta información, que son:

\begin{itemize}
    \item Explicación por atribución de características: son aquellas que permiten identificar la influencia de una determinada característica en una predicción del modelo IA. Se pueden subdividir en dos tipos:
    \begin{itemize}
        \item Métodos basados en perturbación (Perturbation-based): se fundamentan en generar una perturbación a los datos de entrada para estudiar cómo se ve afectada la predicción resultante. Un ejemplo de este tipo es LIME.
        \item Mapas de saliencia (Saliency maps): se basan en presentar mapas de calor en los que se representa la relevancia relativa de los píxeles de la imagen asociada a una predicción del modelo. Un ejemplo de esta clase es CAM \cite{CAM2023} y sus múltiples variantes.
    \end{itemize}
    \item Explicación por texto: se basan en descripciones semánticas para explicar las decisiones del modelo. Se pueden identificar dos posibles paradigmas:
    \begin{itemize}
        \item Descripción de imágenes (Image Captioning) con o sin explicaciones visuales añadidas: se refiere a la tarea de describir el contenido de una imagen en palabras. Esta operatoria es una combinación de conceptos de visión por computadora y procesamiento de lenguaje natural.
        \item Atribución de conceptos: la finalidad de esta subclase es aprender conceptos definidos por humanos a partir de activaciones internas en una red neuronal convolucional. Un ejemplo de esto es TCAV \cite{TCAV2018}.
    \end{itemize}
    \item Explicación por ejemplos: consiste en explicar una predicción específica mostrando un conjunto de ejemplos con características similares. Se subdividen en tres clases:
    \begin{itemize}
        \item Razonamiento basado en casos: son métodos de explicación que tienen como objetivo buscar en una base de datos entradas visualmente similares a imagen dada. Un ejemplo de este tipo es CBIR \cite{CBIR2011}.
        \item Explicaciones contrafácticas: Ejemplo de esta clase son las Counterfactual Generative Network (CGN) \cite{CGN2021} y las conditional Generative Adversarial Network (cGAN) \cite{cGAN2023}.
        \item Prototipos: consiste en que el modelo aprenda de prototipos durante su fase de entrenamiento, con el propósito de conseguir modelos de interpretabilidad intrínseca. Un ejemplo de este método es la arquitectura ProtoPNet \cite{ProtoPNet2019}.
    \end{itemize}
    \item Explicación por conceptos: este tipo de técnicas incorporan conceptos especificados por personas como una fase intermedia, condicionando así las predicciones finales. Ejemplo de este tipo son los modelos CBM \cite{CBM2021}.
\end{itemize}

%Según modalidad de la explicación
%Explanation Modality. Explanation modality refers to the type of explanation provided by each interpretability method. Among the reviewed methods, the explanation can be provided in the form of saliency maps (Explanation by Feature Attribution), semantic descriptions (Explanation by Text), similar examples (Explanation by Examples), or using high-level concepts (Explanation by Concepts). In Section 5, we used this categorization to discuss the reviewed methods.




% Extracto de Ortigossa
%3) ACCORDING TO THE EXPLANATION TASK Some XAI methods are designed to understand learning models’ structures and internal mechanisms, i.e., model explanation methods. Such a category of methods is typically found in Neural Network applications, in which information visualization is applied to generate visual representations of the internal patterns of neural units. However, contrarily to intuition, Poursabzi-Sangdeh et al. [108] indicated exposing the internal mechanisms of a learning model reduces the users’ ability to detect faulty behaviors for unusual instances. Amann et al. [16] claimed such an interpretability reduction might be related to the overhead induced by the large amount of information users are exposed to during the understanding process, even in transparent models. Note the findings of Poursabzi-Sangdeh et al. [108] do not invalidate model explanation methods, but rather alert developers to design tools that synthesize large amounts of information carefully. Model inspection is used when the explanation task is to verify the model’s sensitivity, i.e., the behavior of the learning algorithm or its predictions when the input data are varied through perturbations. On the other hand, prediction explanation methods display visual or textual elements that provide a qualitative/quantitative understanding of the relationship between the input variables and a prediction for clarifying the factors that influence the model’s final decision. According to Ribeiro et al. [97], prediction explanation methods promote trust between users and learning applications faithfully and intelligibly. The explanation of predictions does not require all the classifier’s internal logic to be unraveled. Moreover, such methods should explain individual predictions of a complex model, regardless of whether it is correct or not. Prediction explanation is one of the leading research areas in XAI, with multiple techniques devoted to identifying and quantifying the contribution of input elements to predictions of complex models [4], [33]. Explanations can be provided by a global method or a local attribution one that assigns some measure of importance to each input datum in both granularity, i.e., for a collection of instances or a set of input attributes of a specific data instance. Finally, the output of a learning model can be interpreted through evidence-based (or factual) explanations. In this context, contrastive and counterfactual methods seek justifications for why a decision was not different from that one predicted and how it can be modified, respectively [122].
  %%%%%%%%%%%%%%%%%%%%%%%%%%%%%%%%%%%%%%%%%%%%%%%%%%%%%%
%   NOCIONES DE INTELIGENCIA ARTIFICIAL
%%%%%%%%%%%%%%%%%%%%%%%%%%%%%%%%%%%%%%%%%%%%%%%%%%%%%%
\subsection{Usos de inteligencia artificial en medicina} \label{section:es/uso-ia-medicina}

%- Usos de Inteligencia artificial en medicina
%	- Promesas o ventajas tentativas
%           Computational pathology is poised to revolutionize
%           digital pathology by delivering meaningful automation of
%           anatomic pathology by augmenting pathologists without
%           replacing them.
%	- Diagnóstico diferencial (90-Lu-Trustworthy AI Tuberculosis)
%	- Aprendizaje Federado (112-Tariq-Trustworthy Federated Learning Review)
%   - LLM en medicina (90-Lu-Trustworthy AI Tuberculosis) (no sé bien dónde poner esto, pero siento que debería introducirlo)

% TODO: Acá se pueden comentar (objetivamente y citando fuentes) los beneficios de incorporar xAI en la pràctica médica. Tambièn se puede hacer un listado de controversias y principales dificultades (también citando fuentes)

% TODO: se puede hablar de las metodologías de trabajo propuestas para el uso de AI

La inteligencia artificial puede aplicarse a un sin fin de situaciones para resolver problemas tecnológicos o facilitar tareas que requieran un excesivo trabajo manual. Abdelsamea et al. reúnen en su revisión múltiples aplicaciones de estas tecnologías en histología, algunas de las cuales son: clasificación de tipos de cáncer de pulmón a partir de WSIs de histopatología, clasificación de cáncer de mama a partir de imágenes, clasificación de cáncer cervical, segmentación de cáncer multiorgánico, segmentación de cáncer cerebral, entre otros \cite{Abdelsamea2022}.

Por otro lado, en la comunidad de Kaggle es frecuente encontrar múltiples propuestas de algoritmos y modelos que se encargan clasificar o predecir tipos de cáncer \cite{KaggleSearch}. Esto demuestra un claro interés generalizado en investigar y desarrollar aún más estas tecnologías.

Fuera del ámbito de la histopatología, se destacan dos propuestas recientes. En primer lugar, Lu et al. proponen un flujo de trabajo en seis pasos que utiliza modelos de lenguaje preentrenados e incorpora un chequeo intermedio manual por parte de médicos, con el propósito de lograr explicaciones confiables basadas en texto, interpretables y robustas \cite{Lu2023}. Por último, Tariq et al. elaboran una revisión sistemática acerca del estado del arte del aprendizaje federado confiable, exponen sus ventajas, su taxonomía y un posible marco de trabajo de aplicación. Los autores sostienen que el uso de aprendizaje federado puede permitir el entrenamiento de modelos IA colaborativos entre dispositivos, manteniendo la seguridad de la información \cite{Tariq2024}.

  %%%%%%%%%%%%%%%%%%%%%%%%%%%%%%%%%%%%%%%%%%%%%%%%%%%%%%
%   Lineamientos y adopciones para este trabajo
%%%%%%%%%%%%%%%%%%%%%%%%%%%%%%%%%%%%%%%%%%%%%%%%%%%%%%
\subsection{Lineamientos y adopciones para el presente trabajo} \label{section:es/lineamientos-tfe}

%- Lineamientos y adopciones para este trabajo
%	- esclarecer porqué uso la sigla AI/xAI para referirme a AI y XAI en simultáneo o ambas por separado.
%	- Tengo que añadir una sección de conceptos importantes para informar la manera en que traduje cada término en inglés. ¿O debería dejarlos en inglés?
% Màs ideas en la parte de abajo...

% TODO: Justificar por qué este tema es importante. Por ejemplo: ¿hay pocos diagnósticos? ¿Se tarda mucho en diagnosticar? ¿Se suelen hacer mal? ¿Cuesta especializarse en determinados temas patológicos? Todo eso es importante porque le da solidez al tema.

Si bien a veces en documentación hispanohablante se utilizan las siglas AI y XAI, derivadas del inglés, para referirse a la inteligencia artificial y a la inteligencia artificial explicable, respectivamente, en este trabajo se opta por sus siglas traducidas al español. Así, a lo largo de todo este documento se hablará de IA e IAX para referirse a ambos términos. Ocasionalmente, se utilizará la sigla IA/IAX para referir a ambas en simultáneo. Esto particularmente es útil para describir conceptos que son aplicales a ambas tecnologías por igual. Sin embargo, otras siglas referidas a algoritmos concretos, técnicas específicas y conceptos médicos o informáticos se conservaron en inglés con el propósito de evitar ambigüedad por repetición de acrónimos.

Los conceptos técnicos y médicos más utilizados durante el desarrollo de la revisión sistemática se adjuntan en el glosario del Apéndice \ref{appendix:es/glosario}. En el caso de que se utilicen siglas o acrónimos en inglés para referirse a algoritmos, metodologías o técnicas específicas, se incluye también una posible traducción en dicho glosario.
  %\input{Secciones/es/02-6_Glosario_Complementario}
\newpage
%%%%%%%%%%%%%%%%%%%%%%%%%%%%%%%%%%%%%%%%%%%%%%%%%%%%%%
%   PLANIFICACIÓN DE LA REVISIÓN SISTEMÁTICA
%%%%%%%%%%%%%%%%%%%%%%%%%%%%%%%%%%%%%%%%%%%%%%%%%%%%%%
\section{Protocolo de revisión sistemática} \label{section:es/protocolo}

% Referencias de redacción:
% 11-Fasterholdt-AI medical imaging.pdf

% This section describes the research protocol used for this rapid review. Rapid Reviews (RR) are practice-oriented secondary studies, and their main goal is to provide evidence to support decision-making towards the solution, or at least attenuation, of issues practitioners face in practice \cite{Cartaxo2020}.
% For this Rapid Review, the PRISMA \cite{Tricco2019} methodology was used for reference. The research question for the case study of this work, the repositories used for the search, the search process and search string, the inclusion and exclusion criteria and the selection process are specified as follows.

En esta sección se detalla la planificación efectuada para la revisión sistemática, es decir: el protocolo de selección, las especificaciones de la búsqueda sistemática y el análisis de calidad de resultados. Dentro del protocolo de búsqueda se especifica a su vez las preguntas de investigación (PI) para el caso de estudio de este trabajo, las cadenas de búsqueda de cada repositorio involucrado, los criterios de inclusión y exclusión de las investigaciones halladas y el protocolo de selección definido.

Para este artículo se utilizó la metodología PRISMA de revisión como referencia \cite{Tricco2019}.

  %%%%%%%%%%%%%%%%%%%%%%%%%%%%%%%%%%%%%%%%%%%%%%%%%%%%%%
%   PREGUNTAS DE INVESTIGACIÓN
%%%%%%%%%%%%%%%%%%%%%%%%%%%%%%%%%%%%%%%%%%%%%%%%%%%%%%
\subsection{Preguntas de investigación}

La Tabla \ref{tabla:preguntas_investigacion} detalla las preguntas de investigación formuladas que guían la revisión sistemática. Cada pregunta pretende contemplar un aspecto diferente de la implementación de IA/IAX en histopatología. 

\begin{table}[ht!]
\small
\centering
\begin{tabular}{|| m{1em} m{14em} m{20em} ||} 
 \hline
 \textbf{ID} & \textbf{Pregunta de investigación} & \textbf{Objetivo particular} \\ [0.5ex] 
 \hline\hline
 PI1 & ¿Cuáles son los factores principales que limitan la implementación de IA/IAX en histopatología? & Nombrar y clasificar los principales obstáculos para la implementación de IA/IAX en la histopatología en cinco categorías: Dificultades técnicas, Transparencia del proceso, Acoplamiento al flujo de trabajo, Costo de implementación, Aspectos regulatorios.
 \\ 
 \hline
 PI2 & ¿Qué postura adoptan los autores para el uso de IA/IAX? & Determinar si los autores presentan las soluciones IA/IAX como sustituto, herramienta o asistente para los patólogos. \\
 \hline
 PI3 & ¿Cuáles son los algoritmos utilizados en implementaciones de IA/IAX? & Confeccionar una lista de los algoritmos utilizados para implementar IA/IAX y su frecuencia de uso. \\
 \hline
 PI4 & ¿Qué aportes se realizaron para la integración de IA/IAX al flujo de trabajo de los patólogos? & Determinar principales aportes para integrar herramientas IA/IAX al flujo de trabajo de los patólogos. \\
 \hline
% PI5 & ¿Qué criterios de diseño utilizan para los sistemas IA/IAX en histopatología? & Determinar si los sistemas IA/IAX son concebidos como soluciones end-to-end o expert-in-the-loop, si se plantean soluciones que integren dos o más modelos IA/IAX diferente, si se propone una arquitectura de sistema multiagente, etc. \\
% \hline
 PI5 & ¿Qué aspectos se declaran pendientes de investigación en los artículos analizados? & Determinar aspectos que aún se encuentran pendientes de investigación con respecto al uso de IA/IAX en histopatología. \\
 \hline
\end{tabular}
\caption{Preguntas de investigación postuladas para la revisión sistemática.}
\label{tabla:preguntas_investigacion}
\end{table}

  %%%%%%%%%%%%%%%%%%%%%%%%%%%%%%%%%%%%%%%%%%%%%%%%%%%%%%
%   CADENAS DE BÚSQUEDA
%%%%%%%%%%%%%%%%%%%%%%%%%%%%%%%%%%%%%%%%%%%%%%%%%%%%%%
\subsection{Cadenas de búsqueda} \label{section:es/cadena_busqueda}


Se realizó una búsqueda automática en las bibliotecas digitales de ACM, IEEE Xplore, PubMed y Springer.

% For the construction of the search string, the main terms “explainable artificial intelligence”, “histopathology”, “histology” and “anatomic pathology” were considered, including their alternative terms. The resulting search string is presented in Figure \ref{figura:en/cadena_generica}.
Para la confección de la cadena de búsqueda se utilizaron los términos principales \enquote{explainable artificial intelligence}, \enquote{histopathology}, \enquote{histology} y \enquote{anatomic pathology}, incluyendo sus variantes y sinónimos. La cadena de búsqueda resultante se presenta en la Figura \ref{figura:es/cadena_generica}.

\begin{figure}[thp] % the figure provides the caption
\centering          % which should be centered
\begin{tabular}{c}  % the tabular makes the listing as small as possible and centers it
\begin{lstlisting}
#FULL TEXT
  "xai" OR
  "explainable artificial intelligence" OR
  "explainable ai" OR
  "interpretable ai" OR
  "interpretable artificial intelligence" OR
  "white box" OR
  "human ai"
#AND
#FULL TEXT
  "histopathology" OR
  "histopathological" OR
  "histology" OR
  "histological" OR
  "anatomic pathology" OR
  "anatomical pathology"
\end{lstlisting}
\end{tabular}
\caption{Cadena de búsqueda genérica.}
\label{figura:es/cadena_generica}
\end{figure}
  %%%%%%%%%%%%%%%%%%%%%%%%%%%%%%%%%%%%%%%%%%%%%%%%%%%%%%
%   CRITERIOS DE INCLUSIÓN/EXCLUSIÓN
%%%%%%%%%%%%%%%%%%%%%%%%%%%%%%%%%%%%%%%%%%%%%%%%%%%%%%
\subsection{Criterios de inclusión y exclusión} \label{section:es/criterios_inc_exc}

% Since 2015, it has been observed an increment of xAI techniques applied to AI clinical decision support systems \cite{Giuste2023}. Based on these dates, the inclusion criteria was defined as follows: articles published in English between January 2015 and June 2023, in journals or conferences that underwent peer review, and whose main focus is the implementation of IA/IAX in histology or histopathology. Duplicate articles, surveys, systematic reviews and other rapid reviews were excluded for this study.

Se observa un incremento considerable en la investigación de técnicas IA/IAX aplicadas a sistemas de decisión clínica desde el año 2015 en adelante \cite{Giuste2023}. En base a este dato, se definieron los criterios de inclusión, por los cuales se consideran sólo artículos escritos en inglés, publicados en journals, conferencias o congresos con revisión de pares entre enero de 2015 y agosto de 2024, cuyo enfoque principal esté relacionado a las implementaciones de IAX en histología o histopatología. Se excluyen de la revisión los artículos duplicados o incompletos, que no tengan revisión de pares o cuyo acceso no esté disponible. Se excluyen además los artículos cuyo puntaje de calidad sea estrictamente menor a tres. El método de evaluación de calidad se detalla en la Sección \ref{section:es/calidad}.

% TODO: Retocar formato de esta tabla
\begin{table}[h]
\small
\centering
\begin{tabular}{|| m{17em} m{17em} ||} 
 \hline
 \textbf{Criterios de inclusión} & \textbf{Criterios de exclusión} \\ [0.5ex] 
 \hline\hline
 Artículos publicados en inglés. & Artículos que estén redactados en un idioma que no sea inglés. \\ 
 \hline
 Artículos publicados entre enero de 2015 y agosto de 2024. & Artículos publicados antes de enero de 2015. \\
 \hline
 Artículos publicados en journals, conferences o congresos con revisión de pares. & Artículos que no tengan revisión de pares. \\
 \hline
 Artículos cuyo enfoque principal esté relacionado a las implementaciones de IAX en histología o histopatología. & Artículos no disponibles o incompletos. \\
 \hline
 Artículos con puntaje de calidad total mayor o igual a tres. & Artículos con puntaje de calidad total menor a tres. \\
 \hline
  & Artículos duplicados. \\ [1ex]
 \hline
\end{tabular}
\caption{Criterios de inclusión y exclusión}
\label{tabla:criterios_inc_exc}
\end{table}


  %%%%%%%%%%%%%%%%%%%%%%%%%%%%%%%%%%%%%%%%%%%%%%%%%%%%%%
%   MATRIZ DE EVALUACIÓN DE CALIDAD
%%%%%%%%%%%%%%%%%%%%%%%%%%%%%%%%%%%%%%%%%%%%%%%%%%%%%%
\subsection{Matriz de evaluación de calidad} \label{section:es/calidad}

La Tabla \ref{tabla:es/calidad} muestra la matriz de evaluación que se confeccionó con el propósito de calificar la calidad de los artículos encontrados en la revisión sistemática. Se definieron tres atributos de calidad (AC). En función del grado de cumplimiento de cada atributo, se asigna un determinado puntaje al artículo. Se seleccionan como estudios primarios aquellos artículos cuyo puntaje total sea de al menos tres puntos de calidad.

\begin{table}[h]
\small
\centering
\begin{tabular}{|| m{2em} | m{12em} | m{2.5em} | m{14em} ||} 
 \hline
  	& \textbf{Método de evaluación} & \textbf{Punt.} & \textbf{Descripción} \\ [0.5ex] 
 \hline\hline
 \textbf{AC1} 
 & El artículo trata sobre IAX o sobre la interpretabilidad de los resultados de IA.
 & 2
 & El artículo trata principalmente sobre IAX. \\
 \cline{3-4} 
 &
 & 1
 & El artículo trata parcialmente temas afines a IAX. \\ 
 \cline{3-4} 
 &
 & 0
 & El artículo no está especializado ni hay una mención fuerte de IAX en él. \\ 
 \hline

 \textbf{AC2}
 & El artículo está completamente focalizado en histología o histopatología.
 & 2
 & El artículo trata principalmente de histología e histopatología. \\
 \cline{3-4} 
 &
 & 1
 & El artículo trata parcialmente la histología o histopatología. \\ 
 \cline{3-4} 
 &
 & 0
 & El artículo no se relaciona directamente a la histología e histopatología. \\ 
 \hline

 \textbf{AC3} 
 & Cantidad de preguntas de investigación que responden.
 & 2
 & El artículo contiene la respuesta a más de una pregunta de investigación. \\
 \cline{3-4} 
 &
 & 1
 & El artículo contiene la respuesta a una única pregunta de investigación. \\ 
 \cline{3-4} 
 &
 & 0
 & El artículo no responde ninguna pregunta de investigación. \\
 \hline
\end{tabular}
\caption{Matriz de evaluación de calidad de artículos.}
\label{tabla:es/calidad}
\end{table}

  %%%%%%%%%%%%%%%%%%%%%%%%%%%%%%%%%%%%%%%%%%%%%%%%%%%%%%
%   PROTOCOLO DE SELECCIÓN
%%%%%%%%%%%%%%%%%%%%%%%%%%%%%%%%%%%%%%%%%%%%%%%%%%%%%%
\subsection{Proceso de selección y extracción de datos} \label{section:es/proceso_seleccion}

%The study selection process consisted of ten steps, which were executed sequentially. Details about the process are described in Figure \ref{figura:en/data_extraction}. This process allowed the selection of the primary studies that were analyzed to answer the research question. The PRISMA checklist used in this study is presented in Appendix \ref{appendix:en/prisma_checklist}.
El proceso de selección consistió en once etapas secuenciales. Los detalles del proceso se muestran en la Figura \ref{figura:es/data_extraction}. Como resultado de este proceso se obtuvieron estudios primarios que brindan respuestas a las preguntas de investigación. La lista de verificación PRISMA utilizada en este artículo se presenta en el Apéndice \ref{appendix:es/prisma_checklist}.

La cadena de búsqueda se aplicó en cada biblioteca digital con los ajustes necesarios dependiendo de las particularidades de cada una de ellas. 

\begin{figure}[h]
    \centering
    \includesvg[inkscapelatex=false, width=\textwidth]{Imagenes/TFE_Steps_es.svg}
    \caption{Diagrama PRISMA en el que se detalla el proceso de búsqueda y selección de artículos.}
    \label{figura:es/data_extraction}
\end{figure}

%For this rapid review, one researcher read the titles, abstracts, and full texts to assess their inclusion. Of a total of 777 articles found, 11 primary studies were analyzed. The list of the selected studies is presented in Appendix \ref{appendix:en/estudios_primarios}.
De un total de 1463 artículos encontrados, se seleccionaron en total 62 estudios primarios. La Tabla \ref{tabla:es/matriz_calidad_ep} sintetiza los valores de calidad obtenidos para cada uno de ellos, tanto para atributos particulares como puntaje total.

\begin{table}[ht!]
\scriptsize
\centering
\begin{tabular}{|| m{.30\textwidth}|c|ccc|c ||}
 \hline
Autores del Estudio Primario	& Año	& AC1	& AC2 & AC3 & Calidad total \\ [0.5ex]
 \hline \hline

Shaban-Nejad et al. \cite{ShabanNejad2021}  & 2021 &	2	&	0	&	2	&	4 \\ \hline
Ayorinde et al. \cite{Ayorinde2022}         & 2022 &	0	&	2	&	2	&	4 \\ \hline
Tosun et al. \cite{Tosun2020}               & 2020 &	2	&	2	&	2	&	6 \\ \hline
Jarrahi et al. \cite{Jarrahi2022}           & 2022 &	2	&	2	&	2	&	6 \\ \hline
Kiehl et al. \cite{Kiehl2022}               & 2022 &	0	&	2	&	2	&	4 \\ \hline
Sabol et al. \cite{Sabol2019}               & 2019 &	2	&	2	&	2	&	6 \\ \hline
Verma et al. \cite{Verma2023}               & 2023 &	1	&	1	&	2	&	4 \\ \hline
Tran et al. \cite{Tran2021}                 & 2021 &	1	&	1	&	2	&	4 \\ \hline
Zhou et al. \cite{Zhou2021}                 & 2021 &	1	&	1	&	1	&	3 \\ \hline
Hauser et al. \cite{Hauser2022}             & 2022 &	2	&	1	&	2	&	5 \\ \hline
dos-Santos et al. \cite{dosSantos2022}      & 2022 &	0	&	2	&	1	&	3 \\ \hline
Wang et al. \cite{Wang2021}                 & 2021 &	0	&	2	&	2	&	4 \\ \hline
Rösler et al. \cite{Roesler2023}            & 2023 &	1	&	1	&	2	&	4 \\ \hline
Zehra et al. \cite{Zehra2023}               & 2023 &	0	&	2	&	2	&	4 \\ \hline
Guleria et al. \cite{Guleria2021}           & 2021 &	2	&	2	&	2	&	6 \\ \hline
Finzel et al. \cite{Finzel2022}             & 2022 &	2	&	0	&	2	&	4 \\ \hline
Teng et al. \cite{Teng2022}                 & 2022 &	2	&	1	&	2	&	5 \\ \hline
Mohammadi et al. \cite{Mohammadi2022}       & 2022 &	2	&	2	&	2	&	6 \\ \hline
Holzinger et al. \cite{Holzinger2019}       & 2019 &	2	&	0	&	2	&	4 \\ \hline
Yin et al. \cite{Yin2020}                   & 2020 &	2	&	2	&	2	&	6 \\ \hline
Sabol et al. \cite{Sabol2020}               & 2020 &	2	&	2	&	2	&	6 \\ \hline
Cai et al. \cite{Cai2019}                   & 2019 &	1	&	2	&	2	&	5 \\ \hline
Shawi et al. \cite{Shawi2022}               & 2022 &	2	&	2	&	2	&	6 \\ \hline
Tjoa y Guan \cite{Tjoa&Guan2021}            & 2021 &	2	&	1	&	2	&	5 \\ \hline
Roscher et al. \cite{Roscher2020}           & 2020 &	1	&	1	&	2	&	4 \\ \hline
Schuhmacher et al. \cite{Schuhmacher2022}   & 2022 &	2	&	2	&	2	&	6 \\ \hline
Nazar et al. \cite{Nazar2021}               & 2021 &	2	&	0	&	2	&	4 \\ \hline
Palatnik de Sousa et al.
\cite{PalatnikdeSousa2019}                  & 2019 &	2	&	2	&	2	&	6 \\ \hline
Gu et al. \cite{Gu2023NaviPath}             & 2023 &	1	&	2	&	2	&	5 \\ \hline
Gu et al. \cite{Gu2021Lessons}              & 2021 &	1	&	2	&	2	&	5 \\ \hline
Sauter et al. \cite{Sauter2022}             & 2022 &	2	&	2	&	2	&	6 \\ \hline
Gu et al. \cite{Gu2023XPath}                & 2023 &	2	&	2	&	2	&	6 \\ \hline
Tschandl et al. \cite{Tschandl2020}         & 2020 &	2	&	0	&	2	&	4 \\ \hline
Wenzel y Wiegand \cite{Wenzel&Wiegand2020}  & 2020 &	1	&	0	&	2	&	3 \\ \hline
Ullah et al. \cite{Ullah2024}               & 2024 &	1	&	0	&	2	&	3 \\ \hline
Palkar et al. \cite{Palkar2024}             & 2024 &	2	&	0	&	2	&	4 \\ \hline
Pahud de Mortanges et al. 
\cite{PahuddeMortanges2024}   				& 2024 &	2	&	0	&	2	&	4 \\ \hline
Bouderhem et al. \cite{Bouderhem2024}       & 2024 &	1	&	0	&	2	&	3 \\ \hline
Finzel et al. \cite{Finzel2024}             & 2024 &	2	&	0	&	2	&	4 \\ \hline
Shovon et al. \cite{Shovon2023}             & 2023 &	1	&	2	&	2	&	5 \\ \hline
Liang y Meng et al. \cite{Liang&Meng2023}    & 2023 &	1	&	2	&	2	&	5 \\ \hline
Nasir et al. \cite{Nasir2024}               & 2024 &	2	&	0	&	2	&	4 \\ \hline
Aziz et al. \cite{Aziz2023}   				& 2023 &	1	&	2	&	2	&	5 \\ \hline
Lee et al. \cite{Lee2024}   				& 2024 &	1	&	2	&	2	&	5 \\ \hline
Tabatabaei et al. \cite{Tabatabaei2023}  	& 2023 &	0	&	2	&	1	&	3 \\ \hline
Klauschen et al. \cite{Klauschen2024}  		& 2024 &	2	&	1	&	2	&	5 \\ \hline
Shahamatdar et al. \cite{Shahamatdar2024}   & 2024 &	2	&	2	&	2	&	6 \\ \hline
Sloboda et al. \cite{Sloboda2024}   		& 2024 &	0	&	2	&	1	&	3 \\ \hline
Praetorius et al. \cite{Praetorius2023}   	& 2023 &	1	&	2	&	2	&	5 \\ \hline
Basaad et al. \cite{Basaad2024}   			& 2024 &	2	&	2	&	2	&	6 \\ \hline
Gallo et al. \cite{Gallo2023}               & 2023 &	1	&	2	&	2	&	5 \\ \hline
Alsubai et al. \cite{Alsubai2024}   		& 2024 &	1	&	2	&	2	&	5 \\ \hline
Civit-Masot et al. \cite{CivitMasot2024}    & 2024 &	2	&	2	&	2	&	6 \\ \hline
Amato et al. \cite{Amato2024}               & 2024 &	1	&	2	&	1	&	4 \\ \hline
Vanea et al. \cite{Vanea2024}               & 2024 &	0	&	2	&	2	&	4 \\ \hline
Di Giammarco et al. \cite{DiGiammarco2024}  & 2024 &	1	&	2	&	2	&	5 \\ \hline
Dy et al. \cite{Dy2024}   					& 2024 &	0	&	2	&	1	&	3 \\ \hline
Alabi et al. \cite{Alabi2023}               & 2023 &	1	&	2	&	1	&	4 \\ \hline
Bellantuono et al. \cite{Bellantuono2023}   & 2023 &	1	&	2	&	2	&	5 \\ \hline
Dolezal et al. \cite{Dolezal2024}   		& 2024 &	0	&	2	&	2	&	4 \\ \hline
Dörrich et al. \cite{Doerrich2023}   		& 2023 &	2	&	2	&	2	&	6 \\ \hline
Vanitha et al. \cite{Vanitha2024}   		& 2024 &	1	&	2	&	2	&	5 \\ \hline

\end{tabular}
\caption{Matriz de calidad de los estudios primarios seleccionados.}
\label{tabla:es/matriz_calidad_ep}
\end{table}



\newpage
%%%%%%%%%%%%%%%%%%%%%%%%%%%%%%%%%%%%%%%%%%%%%%%%%%%%%%
%   ANÁLISIS DE RESULTADOS
%%%%%%%%%%%%%%%%%%%%%%%%%%%%%%%%%%%%%%%%%%%%%%%%%%%%%%
\section{Análisis de resultados} \label{section:es/resultados}

En esta sección se analizan respuestas a las preguntas de investigación (PIs) en función del análisis realizado sobre los estudios primarios.

  \subsection{Análisis estadísticos de estudios seleccionados} \label{appendix:es/análisis_biométrico}

La Figura \ref{figura:es/analisis_año_ep} contiene un histograma estadístico en el que se informa la cantidad de estudios primarios agrupados por año de publicación. Se observa en ella un incremento notable en la cantidad de publicaciones en los últimos cinco años. La curva trazada en el gráfico representa la línea de tendencia, la cual nos permite ilustrar que el crecimiento anual es aproximadamente exponencial.

\begin{figure}[htbp]
\centerline{\includegraphics[width=\textwidth]{Imagenes/analisis_año_ep.png}}
\caption{Histograma de cantidad de estudios primarios agrupados por año de publicación. Se observa una tendencia de crecimiento aproximamente exponencial en la cantidad de publicaciones.}
\label{figura:es/analisis_año_ep}
\end{figure}

  %%%%%%%%%%%%%%%%%%%%%%%%%%%%%%%%%%%%%%%%%%%%%%%%%%%%%%
%   PI1 - ANÁLISIS DE RESULTADOS
%%%%%%%%%%%%%%%%%%%%%%%%%%%%%%%%%%%%%%%%%%%%%%%%%%%%%%
\subsection{PI1 - ¿Cuáles son los factores principales que limitan la implementación de IA/IAX en histopatología?} \label{section:es/resultados-pi1}

Se identificaron cinco categorías de dificultades para implementar IA/IAX en histopatología, a saber: dificultades técnicas, transparencia del proceso, acoplamiento al flujo de trabajo, costos de implementación y aspectos regulatorios.
    %%%%%%%%%%%%%%%%%%%%%%%%%%%%%%%%%%%%%%%%%%%%%%%%%%%%%%
%   PI1 - ANÁLISIS DE RESULTADOS
%%%%%%%%%%%%%%%%%%%%%%%%%%%%%%%%%%%%%%%%%%%%%%%%%%%%%%
\subsubsection{Dificultades técnicas}

%In supervised AI systems, it is crucial to rely on datasets whose samples are classified by an expert with the intention of obtaining consistent predictions. However, many authors declare there are operative difficulties when classifying.
En sistemas supervisados de IA es crucial que las muestras de los datasets sean clasificadas por un experto para que las predicciones del sistema sean consistentes. Sin embargo, algunos autores declaran que la clasificación está sujeta a problemas operativos.

%Ayorinde et al. \cite{Ayorinde2022} analyze that a reliable categorization can be problematic for some elements of slide assessment, especially in determination of arteriosclerosis. Arteriosclerosis is recognized by the narrowing of the blood vessel lumen, although this is not always the case: an apparently narrow lumen may arise not only from disease but also from oblique sectioning. Thus, when processing blood vessel slides, the similarity between them becomes a critical issue for AI implementation due to the possibility of arriving at an erroneous diagnosis.
Ayorinde et al. \cite{Ayorinde2022} analiza que disponer de una clasificación confiable puede ser problemático para la evaluación de ciertos elementos en las muestras, principalmente en la determinación de arterioesclerosis. La arterioesclerosis se puede reconocer por el angostamiento del lumen en el vaso sanguíneo, aunque es posible confundir esto con un seccionamiento oblicuo. Es decir, la similitud entre ambos tipos de muestra es un problema crítico para la implementación de IA debido a la probabilidad de arribar a un diagnóstico erróneo.

%Tran et al. \cite{Tran2021} remark that this is also limiting in oncology, and that it is currently unclear how DL systems would deal with this inter- and intra-laboratory variability. They also add that, in many DL systems, the predictions are simply the best guess with the highest probability. In critical circumstances, overconfident predictions, e.g. predicting cancer primary site with only 40\% certainty, can result in inaccurate diagnosis or cancer management decisions.
Tran et al. \cite{Tran2021} destacan que esto también es limitante en oncología dado que no es precisa la forma en que los sistemas de aprendizaje profundo lidiarían con la variabilidad inter e intralaboratorio. Añaden que en muchos de estos sistemas las predicciones son inherentemente suposiciones con una alta probabilidad de ser ciertas. En circunstancias críticas una estimación arriesgada, como determinar cáncer con un 40\% de certeza, puede provocar diagnósticos y tratamientos erróneos.

%Along these lines, Mohammadi et al. \cite{Mohammadi2022}, mention that a fully supervised learning for whole slide image–based diagnostic tasks in histopathology is problematic due to the requirement for costly and time-consuming manual annotation by experts, and propose the use of weakly supervised learning methods in order to reduce costs at scale. When it comes to processing, Mohammadi et al. comment that training AI models on gigapixel size whole-slide images is highly expensive and that they usually resort to patching approaches.
Mohammadi et al. \cite{Mohammadi2022} mencionan que el uso de algoritmos basados en técnicas de aprendizaje fuertemente supervisado es problemático aplicado a tareas de diagnóstico basadas WSI. Esto se debe a que la clasificación de cada corte histológico demanda mucho tiempo y esfuerzo a los expertos. Para atender esta problemática, los autores proponen el uso de métodos de aprendizaje semi-supervisado. En cuanto al procesamiento, mencionan que entrenar un modelo de IA/IAX con imágenes de gigapíxeles de tamaño es muy costoso y suelen recurrir a métodos de parcheo.

%Dos-Santos et al. \cite{dosSantos2022} state that the lack of diversity in datasets is another difficulty that AI systems face. Moreover, the datasets must be in some way linked to clinical patient data to allow the validation of external algorithms, in addition to morphological validations performed by pathologists.
Dos-Santos et al. \cite{dosSantos2022} declaran que la falta de diversidad en los sets de datos es otra problemática subyacente a los sistemas de IA/IAX. Por otro lado, observan que los conjuntos de datos deben estar vinculados de alguna manera a los datos clínicos de los pacientes para permitir la validación de algoritmos externos, además de ciertas validaciones morfológicas realizadas por los patólogos.


%An important aspect to consider is the data privacy of the patients. On this, Rösler et al. \cite{Roesler2023} say that data in medicine must be securely stored, transferred, and protected from unauthorized access.
No obstante, la privacidad de los datos de los pacientes, en casos donde estos sean necesarios, es otro aspecto importante a considerar en la práctica. En este sentido, Rösler et al. \cite{Roesler2023} observan que los datos en medicina deben almacenarse de manera segura y ser transferido y protegidos de accesos no autorizados.

%On the other hand, Holzinger et al. \cite{Holzinger2019}, reflect on the fact that oftentimes the datasets are not “big” enough. They observe that there is an inherent tension between ML performance (predictive accuracy) and explainability. Often the best-performing AI models are the least transparent, and the ones providing a clear explanation are less accurate.
Por otro lado, Holzinger et al. \cite{Holzinger2019} reflexionan sobre el hecho de que algunos sets de datos no son suficientemente abundantes. Existe una relación de compromiso entre el desempeño de los modelos de aprendizaje automático, medida en términos de la exactitud predictiva, y su explicabilidad. Con frecuencia los modelos IA/IAX más performantes son los menos transparentes, mientras que los más explicados son más inexactos.

%% A escribir...
Civit-Masot et al. \cite{CivitMasot2024} comentan que unos pocos sistemas de diagnóstico asistido logran una exactitud del 100\%, y que esto se debe a que generalmente prueban su desempeño con un subconjunto de las muestras que se utilizaron durante el entrenamiento. Esto puede ocasionar errores de clasificación cuando se evalúan muestras tomadas con diferentes dispositivos de escaneo. Añaden, además, que en redes neuronales los pesos y parámetros internos son completamente incomprensibles y no permiten interpretar cómo el algoritmo alcanza un resultado dado.

Zhou et al. \cite{Zhou2021} postulan una serie de problemáticas subyacentes a los datos necesarios para implementar técnicas de aprendizaje profundo. Observan que algunas tareas requieren diferentes etiquetados de imágenes médicas, lo que es costoso en términos de tiempo, y que la diferencia de experiencia entre el personal ocasiona disparidad en el detalle de las etiquetas. Las muestras, por su parte, presentan disparidad entre sí, dado que los hospitales no suelen utilizar el mismo equipamiento ni configuración para obtenerlas. Por otro lado, la ocurrencia de enfermedades se rige por una distribución sesgada de cola larga o cola de Pareto. Esto significa que, aunque se dispone de un gran volumen de datos para un número pequeño de enfermedades, la mayoría de las enfermedades son poco frecuentes en clínicas. Dado que la inteligencia artificial requiere un volumen elevado de datos confiables, estos factores pueden inducir sesgos en las predicciones de un modelo.

Respecto a la clasificación de imágenes de biopsia, Guleria et al \cite{Guleria2021} observan que el reconocimiento basado en aprendizaje profundo requiere una basta cantidad de muestras etiquetadas que le permiten aprender a detectar las características anormales de los tejidos. Esta tarea de etiquetado suele ser tediosa y sostienen que es un obstáculo que dificulta a los investigadores desarrollar sus modelos de IA.

Por otra parte, Tjoa y Guan \cite{Tjoa&Guan2021} destacan que las verdades fundamentales, ground truths como las llaman en inglés, provistas por los profesionales de áreas médicas no siempre son completamente correctas. Esto puede desencadenar predicciones erradas a causa de los datos incorrectos de aprendizaje.

Finzel et al. \cite{Finzel2022} comentan las ventajas del uso de redes neuronales gráficas (GNN) en aplicaciones médicas y biológicas, pero destacan que, a pesar de que simplifican la necesidad de seleccionar características o definir funciones matemáticas específicas para ciertas distribuciones de datos, aún es difícil para un usuario interpretar por qué una GNN arriba a un determinado resultado. Debido a esto se han estado investigando técnicas para aumentar su interpretabilidad, ya sea usando estrategias generativas, backpropagation, métodos basados en perturbación, entre otros. No obstante, los autores observan que existe la necesidad de evaluar la validez conceptual de una GNN.

Nazar et al. \cite{Nazar2021} en su revisión sistemática exponen una tabla con los desafíos que afronta la inteligencia artificial explicable. A nivel técnico, destacan la dificultad en aportar interpretabilidad a los algoritmos de aprendizaje profundo, que son de caja negra, y problemas relacionados al desempeño del sistema. Los autores observan que el rendimiento del algoritmo IAX condiciona el desempeño total del sistema. De esto se desprende la necesidad de obtener la mejor técnica de explicabilidad posible para aumentar la transparencia del modelo.

Palatnik de Sousa et al \cite{PalatnikdeSousa2019} encuentran ventajas en utilizar LIME como técnica de explicabilidad, dado que al ser agnóstico del modelo podría ser más fácil comparar resultados con estudios futuros, lo cual no ocurre con técnicas de gradiente o saliencia. Advierten, además, que los mapas de calor de saliencia han demostrado explicaciones no confiables en ciertas condiciones.

Gu et al. \cite{Gu2021Lessons} advierten que la mayor dificultad para la patología digital radica en que los datos histológicos tienden a tener una alta varianza entre pacientes. Por lo tanto, un modelo pre-entrenado a menudo tiene dificultades para generalizar resultados cuando se aplica con un nuevo conjunto de pacientes. Estos problemas de generalización deterioran el rendimiento del sistema y se acentúan ante cambios en el dominio, como por ejemplo cambios en el procedimiento con el que se obtienen datos en los centros médicos \cite{Gu2023NaviPath}.

Gu et al. \cite{Gu2021Lessons} observan, además, que los datos patológicos del día a día están desbalanceados por naturaleza. Las áreas de metástasis a menudo ocupan porciones muy pequeñas de toda el corte WSI. Debido a esto, habría más anotaciones negativas que positivas. Este desbalance genera sesgos en las predicciones del modelo. Lian y Meng \cite{Liang&Meng2023} observan un ejemplo concreto de datos desbalanceados en el set BreaKHis \cite{BreaKHisDataset}, el cual contiene una cantidad significativamente mayor de muestras malignas que benignas. Ante este problema, Civit-Masot et al. \cite{CivitMasot2024} utilizan técnicas de aumento de datos como preprocesamiento para mejorar la distribución de muestras por categoría.

Sauter et al. \cite{Sauter2022} mencionan una serie de factores relacionados con los datos de entrenamiento que limitan la precisión en los algoritmos de clasificación. Algunos de estos son los desbalances de cantidad de muestras por clase, mal etiquetado, cantidad de datos disponibles y sesgos latentes en las muestras. En particular, los sesgos de medición, de muestreo y de correlación de clase, entre otros, con frecuencia ocasionan límites teórico-prácticos a la calidad de las predicciones del modelo.

Respecto a la clasificación de datos, Bellantuono et al. \cite{Bellantuono2023} proponen un modelo explicable para el diagnóstico de cáncer de tiroides y exponen sus limitaciones en casos con características anómalas. Encuentran casos donde un análisis histológico arroja que el tejido está sano, mientras que una prueba inmunohistoquímica revela la presencia de mutaciones que resultan en picos correspondientes a carotenoides. Los autores afirman que no existe un consenso médico sobre este respecto, puesto que no se acepta universalmente que los tejidos de estas características progresen a cáncer. Esto implica ambigüedad y posibles errores en la asignación formal de tales muestras histológicas.

Kiehl et al. \cite{Kiehl2022} objetan que un problema inmediato en los sistemas de IA es la carencia de datos suficientes de entrenamiento que representen la extensa variabilidad existente en las clínicas; es decir, la variabilidad inherente entre cada institución, entre regiones, diferentes etnias, entre otros factores. Contar con estos conjuntos de datos es un prerrequisito para la patología computacional de grado clínico. Los autores marcan la diferencia entre el grado clínico y el grado de investigación, siendo la primera más restrictiva.

Kiehl et al. observan que la baja disponibilidad de datos de entrenamiento etiquetados por patólogos también es un problema, aunque esto se puede mejorar adoptando métodos automatizados que vuelvan más eficiente el proceso de anotación \cite{Kiehl2022}.

Ullah et al. \cite{Ullah2024} advierten que hay ciertas preocupaciones en cuanto a combinar herramientas de diagnóstico asistido por computadora (CAD) con modelos de lenguaje extenso (LLM). Los LLMs pueden generar respuestas coherentes, pero carecen de verdadero entendimiento acerca de conceptos médicos. Las respuestas de este tipo de modelos están basadas en patrones estadísticos aprendidos durante el entrenamiento, lo cual podría no abarcar diagnósticos médicos complejos o intrincados. Por otro lado, el rendimiento de los LLMs es altamente dependiente de la calidad y la diversidad de los datos de entrenamiento. La presencia de sesgos e imprecisiones pueden pasar inadvertidas durante el entrenamiento y deteriorar el modelo, perpetuando resultados sesgados y dispares.

Bouderhem et al. \cite{Bouderhem2024} sostienen que las investigaciones de IA pueden verse restringidas debido a que los datos podrían no ser precisos o contener errores o diagnósticos equivocados. Los desarrolladores, en este caso, deben diseñar algoritmos que tomen en consideración una amplia gama de situaciones para todos los grupos de la población de datos. Los autores declaran que un algoritmo de IA sesgado inherentemente causa discriminación y predicciones erradas.

Otro requisito crucial para los desarrolladores es implementar indicadores de rendimiento que permitan medir el éxito de la IA,  que permitan a los médicos detectar posibles errores y sesgos. No considerar este factor puede desembocar en mala praxis \cite{Bouderhem2024}.

Holzinger et al. \cite{Holzinger2019} declaran que el principal problema de los sistemas de aprendizaje automático radica en que trabajan en forma estadística, lo que conlleva serias limitaciones en su desempeño. Tales sistemas no son capaces de comprender el contexto y, por ende, no pueden razonar sobre intervenciones y retrospecciones. Los autores sostienen que este problema se puede mitigar con la dirección de un modelo humano, similar a otros utilizados en investigaciones de causalidad, que permiten identificar por qué se arriba a un resultado concreto.

    %%%%%%%%%%%%%%%%%%%%%%%%%%%%%%%%%%%%%%%%%%%%%%%%%%%%%%
%   PI1 - ANÁLISIS DE RESULTADOS
%%%%%%%%%%%%%%%%%%%%%%%%%%%%%%%%%%%%%%%%%%%%%%%%%%%%%%
\subsubsection{Transparencia del proceso}

%Explainability is considered one of the prerequisites for AI in medicine. In this sense, xAI is seen as a step towards the realization of the FATE principles in AI \cite{ShabanNejad2021}.
Se considera a la explicabilidad como uno de los prerrequisitos del uso de IA en medicina. En este sentido, IAX es visto como un paso hacia la realización de los principios FATE en IA \cite{ShabanNejad2021}.

%Holzinger et al. \cite{Holzinger2019} propose that it is necessary to understand the causality of representations. This is why they differentiate the words “explainability” and “causality”. Explainability highlights decision-relevant parts of the used representations of the algorithms and active parts in the algorithmic model and it does not refer to an explicit human model. Causability is defined as the extent to which an explanation of a statement to a human expert achieves a specified level of causal understanding in a specified context of use. Establishing causability as a solid scientific field becomes imperative in this sense.
Holzinger et al. \cite{Holzinger2019} proponen que es necesario entender la causalidad de los resultados, diferenciando entre explicabilidad y causalidad. La explicabilidad destaca partes relevantes para la decisión de las representaciones utilizadas de los algoritmos y no se refiere a un modelo humano explícito. Por su parte, la causalidad se define como el grado en que la explicación de una afirmación logra un nivel específico de comprensión causal para un experto en un contexto específico. En este sentido, se vuelve imperativo establecer la causalidad como un campo científico sólido.

%A escribir
Teng et al. \cite{Teng2022} observan que una gran parte de las técnicas de interpretabilidad propuestas hasta el momento están diseñadas como marcos de trabajo genéricos y que no son específicos para tratar con campos sensibles del dominio médico. Si bien los modelos interpretables de aprendizaje profundo han tenido buenos resultados, destacan que aún persiste la carencia de conceptos estándar o métricas de evaluación definidas. Fundamentan su postura basado en que generalmente la interpretabilidad se evalúa en forma intuitiva observando imágenes, pero en el ámbito médico este hecho podría acarrear un impacto negativo. Por lo tanto, para lograr modelos más confiables y robustos es necesario contar con métodos cuantitativos de análisis de intepretabilidad que permitan evaluar su efectividad real en la práctica clínica.

Sabol et al. \cite{Sabol2020} sostienen que los modelos predictivos deberían ser precisos y responsables; es decir, que deberían declarar la incertidumbre en sus predicciones e indicar que para casos complejos sea necesario constatar con la inspección de un experto. Es por eso que otro enfoque de interpretabilidad en los algoritmos de aprendizaje automático es medir la incertidumbre predictiva, es decir, cuán incierta es la predicción para un ejemplo particular de muestra.

Respecto a los modelos interpretables mediante técnicas post-hoc (LRP, Grad-CAM, entre otros), Schuhmacher et al. \cite{Schuhmacher2022} observan que, aunque muestran una salida interpretable, los enfoques existentes carecen de una definición clara de lo que constituye una interpretación o explicación válida. Destacan además que, pese al esfuerzo de varios autores por proponer definiciones de interpretabilidad y explicabilidad, aún no se logró un consenso uniforme. Por su parte, Nazar et al. \cite{Nazar2021} arriban a conclusiones similares.

Gallo et al. \cite{Gallo2023} concluyen que es complejo evaluar la calidad de las explicaciones generadas por un modelo IA/IAX. Observan que una buena explicación debería cumplir varios requisitos, como por ejemplo ser interpretable y fiel al modelo. Debido a la subjetividad subyacente en las explicaciones, a menudo se consulta con expertos del dominio para evaluar su efectividad a través de cuestionarios. No obstante, en este sentido la calidad se ve afectada por la experiencia de los usuarios. Los autores advierten además que la forma en que se presentan los resultados del sistema puede afectar la utilidad y confiabilidad percibida del mismo.

Pese a estas dificultades, Gallo et al. mencionan que existen métodos automáticos para medir la solidez de una explicación en un sistema de inteligencia artificial. Algunos de ellos se basan en insertar perturbaciones y evaluar la degradación de su desempeño, ya sea a través de la Sensitivity-n, el área sobre la curva perturbada (AOPC), entre otros métodos.

Nasir et al. \cite{Nasir2024} destacan que la falta de interpretabilidad en modelos de inteligencia artificial obstaculiza la identificación y mitigación de sesgos. Dichos sesgos pueden ocasionar graves consecuencias en el ámbito médico: un diagnóstico erróneo, tratamiento deficiente y potencial daño a la salud de los pacientes. La inexplicabilidad y falta de transparencia de los sistemas de IA también provocan dilemas éticos y deterioran su confiabilidad. Los autores citan un caso de estudio de aplicabilidad de GPT-4 como un chatbot médico en el que se concluye que, si bien tiene beneficios al reducir la sobrecarga laboral de los médicos y proveer un servicio de atención las 24 horas a los pacientes, acarrea también serias limitaciones, como la incapacidad de comprender el contexto completo, la falta de criterio clínico y la dependencia con los datos de entrenamiento. Esto involucra múltiples riesgos de aspecto ético y legal, razón por la cual debería implementarse cautelosamente con el fin de complementar al experto, en lugar de reemplazarlo.

Ullah et al. \cite{Ullah2024} advierten que los modelos de lenguaje extenso (LLMs) tienen naturaleza de caja negra. Por lo tanto carecen de explicabilidad y acarrean los mismos problemas y dificultades que otros algoritmos de inteligencia artificial. El modelo no provee razones transparentes para sus decisiones. Esto dificulta que los profesionales del ámbito médico validen y confíen sus recomendaciones, e invalida que estos sistemas puedan utilizarse en escenario donde la toma de decisiones es crítica. Los autores sostienen que los avances recientes en algoritmos evolutivos y genéticos, en términos de verificación de rendimiento, ofrecen posibles soluciones a esta preocupación.

Finzel et al. \cite{Finzel2024} observan que los modelos de aprendizaje profundo interpretables pueden no ser explicables debido a las relaciones complejas entre conceptos y jerarquías conceptuales. A su vez, proponen que estructurar las razones de una determinada clasificación a través de una interacción basada en el diálogo entre el usuario y el sistema podría ayudar a navegar las diferentes rutas de decisión y estrategias del modelo.

Bouderhem et al. \cite{Bouderhem2024} declaran que la explicabilidad es un gran desafío. Los sistemas de IA son criticados por su opacidad dado que los observadores e investigadores no conocen cómo alcanzan ciertas decisiones. La confianza pública en salud digital podría verse amenazada si las personas no confían en sus servicios. Es crucial comprender claramente cómo funcionan los sistemas de IA. Para esto, es preciso determinar sus limitaciones para que los pacientes puedan tomar decisiones y den su consentimiento informado, incluso en emergencias médicas. Los autores sostienen que proporcionar métodos que vuelvan explicable a la IA aportará transparencia y confiabilidad.

    %%%%%%%%%%%%%%%%%%%%%%%%%%%%%%%%%%%%%%%%%%%%%%%%%%%%%%
%   PI1 - ANÁLISIS DE RESULTADOS
%%%%%%%%%%%%%%%%%%%%%%%%%%%%%%%%%%%%%%%%%%%%%%%%%%%%%%
\subsubsection{Acoplamiento al flujo de trabajo}

%Ayorinde et al. \cite{Ayorinde2022} postulate that it is not necessary to achieve a perfect AI system for it to be used in the hospitals. In fact, it is more probable that the most useful tools are a combination between AI models and a set of work rules that favor the effective human supervision.
Ayorinde et al. \cite{Ayorinde2022} postulan que no es necesario lograr un sistema IA perfecto para que pueda ser utilizado en las clínicas. Sostienen, en cambio, que es probable que las herramientas más útiles sean una combinación entre modelos de IA y conjuntos de reglas de trabajo que favorezcan una supervisión humana efectiva.

%According to Tosun et al. \cite{Tosun2020} the main function of xAI in pathology is to promote safety, reliability, and accountability in addressing issues with bias, transparency, safety, and causality. The authors find, however, that there is not yet a consensus on how pathologists should supervise or work with computational pathology systems. As patient safety in pathology is the result of a complex interaction between pathologists, other physicians, laboratory personnel and computational pathology applications, they conclude that xAI must help the pathologist be more precise and efficient in their work.
Según Tosun et al. \cite{Tosun2020}, la principal función de IAX en patología es promover la seguridad y confiabilidad al abordar cuestiones con sesgos, transparencia y causalidad. Los autores observan, no obstante, que no hay un consenso respecto a cómo deberían supervisar los patólogos los sistemas de patología computacional, o sobre cómo trabajar con ellos. En estos términos, el bienestar del paciente es el resultado de una interacción compleja entre patólogos, médicos, personal de laboratorio y las aplicaciones computacionales. Por esta razón, los autores concluyen que las técnicas IAX deben asistir a los patólogos a que sean más precisos y eficientes en su trabajo.

%Jaharri et al. \cite{Jarrahi2022} claim that the IA/IAX systems employed in medicine present interoperability challenges, given that these systems often show an amount of information or in formats that appear indecipherable to physicians. These authors offer that an expert-in-the-loop AI work system could clarify a mutual workflow between humans and machines.
Jarrahi et al. \cite{Jarrahi2022} postulan que los sistemas IA/IAX utilizados en medicina poseen problemas de interoperabilidad, dado que a menudo presentan demasiada información en formatos indescifrables para los médicos. Los autores de este estudio proponen que un sistema de IA intervenido por expertos, también llamado \enquote{experto en el bucle} o \enquote{expert-in-the-loop}, podría clarificar un flujo de trabajo mutuo entre personas y computadoras.

%As for Verma et al. \cite{Verma2023}, they suggest that AI can be included in collective decision-making processes in oncology either as a tool or as a member, each of these alternatives generating different sets of ethical, societal and technological issues. In an interview they conducted for their article \cite{Verma2023} with seven physicians working at the Lausanne University Hospital (CHUV), in relation to AI systems autonomy, they detected a consensus on how trust cannot be put in something that is not trustful or bypasses the doctors. One of the experts that took part in this interview added that an AI model cannot be trusted if it is not able to choose which treatment modality works best for a particular patient.
Por su parte, Verma et al. \cite{Verma2023} sugieren que la inteligencia artificial puede incluirse en procesos colectivos de toma de decisiones, ya sea como miembro o como herramienta. Realizaron una entrevista a siete médicos del Lausanne University Hospital (CHUV) en la que se les preguntó cuestiones éticas, sociales y tecnológicas sobre la inteligencia artificial. Respecto a la autonomía de los sistemas IA, los autores observan un consenso en que no se puede recurrir a algo que no es confiable e intentar sobrepasar a los doctores. Uno de los expertos entrevistados añade además que no se puede confiar en un sistema si este no es capaz de elegir la modalidad más adecuada de tratamiento para un paciente en concreto.

Nazar et al. \cite{Nazar2021} observan que la aplicabilidad de IAX en diferentes dominios aún requiere esfuerzos de investigación y estudiarse en profundidad. Por otro lado, concluyen que es preciso obtener explicaciones a las decisiones de los sistemas de IA, y que el usuario final debería poder comprenderlas.

Gu et al. \cite{Gu2021Lessons} encuentran que para el sistema es complejo detectar lesiones pequeñas de tejido en cortes WSI, y advierten que, según consultas con profesionales, es mucho más valiosos encontrar estas áreas pequeñas con inteligencia artificial, ya que las lesiones más grandes se pueden localizar rápidamente sin asistencia.

Por otro lado, Gu et al. \cite{Gu2023XPath} comentan que es difícil convencer a los patólogos sobre transformar el diagnóstico manual en prácticas asistidas por inteligencia artificial. Los autores creen que la causa de esto subyace en que, a pesar de que se investigan formas de mejorar el desempeño de los sistemas de IA, hay una notoria carencia de entendimiento acerca de cómo los doctores se podrían beneficiar de la inteligencia artificial y usarla en diagnósticos. Y si bien las investigaciones sobre IAX tienen por objetivo principal explicar los hallazgos de los modelos, más que optimizar su rendimiento, argumentan que no es suficiente para asistir a los patólogos contar con algoritmos optimizados o aplicar IAX. Una integración deficiente en el flujo de trabajo médico podría suponer una carga mayor para los patólogos, factor que desincentiva el uso de sistemas de inteligencia artificial en la práctica.

Dolezal et al. \cite{Dolezal2024} observan que es crucial que los sistemas de aprendizaje profundo sean transparentes e interpretables para poder asistir de manera efectiva a la toma de decisiones clínicas, y afirman que el software que integra explicabilidad con cuantificación de la incertidumbre presenta una ventaja notable para ser adoptada como herramienta clínica.

Kiehl et al. \cite{Kiehl2022} destacan que adoptar patología computacional acarrea riesgos de sesgos por automatización, en el que el usuario puede confiar ciegamente en el resultado del sistema, sin hacer validaciones suficientes sobre la calidad del mismo. Este riesgo también es percibido por Ullah et al. \cite{Ullah2024}. Estos últimos observan que la incorporación de LLMs en diagnósticos médicos puede deteriorar la autonomía de los profesionales y poner en riesgo el pensamiento crítico y el razonamiento clínico independiente. Ullah et al. advierten también que, aunque ChatGPT se muestra prometedor en varias aplicaciones, es necesario aplicar validaciones rigurosas, estudios a largo plazo y pruebas en la práctica cotidiana para evaluar su desempeño, confiabilidad e impacto en el resultado de los pacientes. 

Finzel et al. \cite{Finzel2024} analizan que los métodos basados en resaltar áreas relevantes de píxeles en las imágenes de entrada, aplicados a redes convolucionales, pueden ser interpretables por expertos en el dominio, pero el significado de este resaltado podría ser ambiguo y fuertemente dependiente de si la etiqueta de verdad fundamental coincide con el resultado de la red neuronal. Añaden que es necesario brindar algún significado a las áreas de píxeles importantes, si los destinatarios de la explicación son novatos, y mencionan investigaciones recientes de explicaciones basadas en conceptos.

Bouderhem et al. \cite{Bouderhem2024} declaran que es necesario ver más allá de la publicidad exagerada del uso de la IA y evaluar las ventajas y desventajas de su uso en el ámbito sanitario. La inteligencia artificial en medicina posee nuevos desafíos que es preciso afrontar, como la presencia de sesgos o la responsabilidad subyacente en situaciones en que los reportes médicos de los pacientes fueran vulnerados o robados.

Respecto a los sistemas de asistencia de decisiones clínicas (CDSS, del inglés clinical decision support system), Cai et al. \cite{Cai2019} destacan que algunos trabajos demuestran que pueden ser difíciles de implementar con éxito en la práctica, citando como causa raíz la falta de consideración de lineamientos de Interacción Humano-Computadora (HCI, de Human computer interaction). Los usuarios pueden resistirse a adoptar una herramienta si no comprenden sus capacidades o su utilidad a comparación de las prácticas preexistentes. A este factor, se añade también la aversión algorítmica como desafío subyacente a esta clase de sistemas.

Cai et al. también encuentran que algunos autores descubrieron que las prácticas existentes favorecen la socialización entre médicos, y si la tecnología de diagnóstico asistido carece de esta capacidad de razonar decisiones, puede ser difícil integrarla al esquema de trabajo actual.

Klauschen et al. \cite{Klauschen2024} advierten que existe un obstáculo interdisciplinario, puesto que los expertos en aprendizaje automático y los patólogos necesitan aprender a interactuar entre sí y encontrar un lenguaje común. Observan que la currícula académica actual raras veces incorporan los cursos necesarios para que los estudiantes puedan cooperar con miembros de otras disciplinas. Los autores sugieren además que es necesario contar con planes de estudio académicos que sean novedosos, en niveles que van desde especializaciones menores hasta educación de posgrado.

    %%%%%%%%%%%%%%%%%%%%%%%%%%%%%%%%%%%%%%%%%%%%%%%%%%%%%%
%   PI1 - ANÁLISIS DE RESULTADOS
%%%%%%%%%%%%%%%%%%%%%%%%%%%%%%%%%%%%%%%%%%%%%%%%%%%%%%
\subsubsection{Costos de implementación}

%Zehra et al. \cite{Zehra2023} analyze the implementation of AI in low- and middle-income countries (LMICs). They remark that, while AI systems there face similar technical challenges and issues to those of more developed countries, there are further difficulties in LMICs. The authors perceive that when labs struggle due to financial constraints to hire trained histopathologists, and when there is scarcity of trained laboratory technologists even for conventional histopathology, it would be extremely difficult to obtain funds and manpower for implementing AI and digital pathology.
Zehra et al. \cite{Zehra2023} analizan la implementación de inteligencia artificial en países de bajos y medianos ingresos (LMIC). Destacan que, a pesar de que los desafíos técnicos que deben enfrentar los sistemas IA/IAX son los similares en la mayoría de los países más desarrollados, los LMIC deben enfrentar dificultades adicionales. Los autores perciben que es extremadamente difícil obtener fondos para implementar IA y patología digital, dado que los laboratorios no solo se enfrentan a restricciones financieras para contratar histopatólogos, sino también a la escasez de estos profesionales.
%In the face of this, they propose a series of low-cost alternatives to venture into AI systems. For instance, low resource organizations can access available open-source whole-slide image archives provided by the Cancer Genome Atlas \cite{CancerGenomeAtlasWebsite}, the Cancer Imaging Archive \cite{CancerImagingArchiveWebsite} and the Digital Pathology Association’s Whole-Slide Imaging Repository \cite{DAPAWebsite}, among others. Another alternative would be when facing internet glitches and the downloading of these images provided by these repositories. As they are large in size, many of the professionals in LMICs may find it difficult to obtain the data for the AI training. To solve this issue, Pathologists can photograph a region of interest for a particular pathology by using the data of their own patients. After taking the photograph, these images can be classified and uploaded in AI systems either open-source or commercially available, although the latter are usually expensive.
Para atender este problema, los autores proponen una serie de alternativas de bajo costo para incursionar en los sistemas de IA. Por ejemplo, las organizaciones de bajos recursos pueden acceder a archivos WSI libres (open-source) provistos por el Cancer Genome Atlas \cite{CancerGenomeAtlasWebsite}, Cancer Imaging Archive \cite{CancerImagingArchiveWebsite} y el repositorio WSI de la Digital Pathology Association \cite{DAPAWebsite}, entre otros.

Ante fallas en la conexión a internet, los autores sugieren descargar las imágenes de estos repositorios. Debido a que las WSI suelen ser inmensas en tamaño, muchos profesionales en LMIC podrían tener dificultades para obtener datos de entrenamiento para los modelos de IA. Para resolver este problema, los autores recomiendan que los patólogos fotografíen la región de interés de una patología en concreta usando los datos de sus propios pacientes. Estas fotografías pueden posteriormente clasificarse y subirse a sistemas IA libres o comerciales, teniendo en cuenta que esas últimas suelen ser costosas.

Shawi et al. \cite{Shawi2022} comentan que uno de los principales obstáculos para los algoritmos de aprendizaje profundo es que requieren una gran cantidad de datos etiquetados para afinar su arquitectura y parámetros internos, que son costosos y difíciles de obtener en la práctica médica. Además, objetan que desarrollar una red neuronal con sus hiperparámetros bien sintonizados es una tarea desafiante y que insume mucho tiempo.

    %%%%%%%%%%%%%%%%%%%%%%%%%%%%%%%%%%%%%%%%%%%%%%%%%%%%%%
%   PI1 - ANÁLISIS DE RESULTADOS
%%%%%%%%%%%%%%%%%%%%%%%%%%%%%%%%%%%%%%%%%%%%%%%%%%%%%%
\subsubsection{Aspectos regulatorios}

%In relation to regulations, Tosun et al. \cite{Tosun2020} reveal that a proposed regulation before the European Union would prohibit “automatic processing” unless people are safeguarded. They foresee that future laws may further restrict AI use in professional practices, which represents a huge challenge to industry.
En términos regulatorios, Tosun et al. \cite{Tosun2020} revelan que una regulación propuesta ante la Unión Europea podría prohibir el procesamiento automático a menos que se tenga en cuenta la protección de los datos personales. Prevén además que en un futuro se implementen leyes que restrinjan aún más el uso de IA en prácticas profesionales.

%Dos-Santos et al. \cite{dosSantos2022} affirm that AI algorithms must work under regulatory standards for testing and usage in medical facilities. According to Tosun et al. \cite{Tosun2020}, this is already a known necessity in radiology.
Dos-Santos et al. \cite{dosSantos2022} afirman que los algoritmos de IA deben funcionar regidos por estándares regulatorios para pruebas y uso en instalaciones médicas. De acuerdo con Tosun et al. \cite{Tosun2020} esto es una necesidad conocida en radiología.

%Thus, due to data privacy, Rösler et al. \cite{Roesler2023} observe that professionals must work under an ethical approval and informed consent of the patient prior to the use and evaluation of patient-related data. In addition to this, if possible, anonymized raw data have to be included from the start of the training process.
Respecto a la privacidad de la información, Rösler et al. \cite{Roesler2023} observan que los profesionales deben contar con el consentimiento informado del paciente antes de hacer uso de sus datos.

%A escribir
Zhou et al. \cite{Zhou2021} observan que la falta de estándares en los protocolos de adquisición de datos, en términos del equipamiento y la configuración utilizados, provoca deriva en el desempeño del sistema. Destacan también que se carece de estándares que establezcan la forma de etiquetar las imágenes médicas, susceptibles de ser utilizadas para entrenar un modelo.

Hauser et al. \cite{Hauser2022} mencionan que la Unión Europea apela a la transparencia de los modelos de inteligencia artificial como un requisito legal. No obstante esto, mencionan que en la literatura médica aún está vigente el debate respecto a la necesidad y la utilidad de implementar IAX.

Por su parte, Tjoa y Guan \cite{Tjoa&Guan2021} observan que puede ser necesario alejar el estudio de la interpretabilidad de los estudios centrados en algoritmos. Objetan, además, que el establecimiento de requisitos estándar en el desarrollo de modelos podría frenar los procesos de investigación propiamente dichos, pero consideran que puede ser una manera eficiente de concertar acuerdos que velen por evitar daños.

Nazar et al. \cite{Nazar2021} observan que la naturaleza de caja negra de los algoritmos de aprendizaje profundo puede acarrear cuestiones legales y éticas que reduzcan la confianza de los usuarios, principalmente por el uso de información sensible de naturaleza médica.

Wenzel y Wiegand \cite{Wenzel&Wiegand2020} comentan que la definición de estándares en sistemas de inteligencia artificial dentro del área médica es compleja debido a los múltiples casos de aplicación, que van desde detectar nódulos en muestras pulmonares, incluyendo interpretación de datos de pacientes, hasta sistemas basados en diálogo que formulan preguntas sistemáticas sobre síntomas y sugieren procedimientos. Tomando este contexto como punto de partida, los autores encuentran que diversos organismos comenzaron a trabajar en la definición de estándares, entre los que se encuentran: el plan federal del U.S. National Institute of Standards and Technology (NIST) en el año 2019; un consorcio dirigido por el Chinese Electronics Standard Institute en 2018; la publicación “Ethics Guidelines for Trustworthy Artificial Intelligence” por parte de la Unión Europea en 2019; y los grupos de trabajo del Deutches Institut für Normung (DIN) en el año 2019.

Según Ullah et al. \cite{Ullah2024}, la integración de LLMs en el diagnóstico médico conlleva cuestiones éticas a considerar. Es preciso priorizar la seguridad y privacidad de los datos sensibles de los pacientes para evitar ocasionarles daños de ningún tipo y las implicancias legales que pueda acarrear. Los autores postulan que los organismos regulatorios y las sociedades profesionales deberían colaborar para establecer estándares y marcos de trabajo que aseguren implementaciones de aplicaciones basadas en LLM que sean éticas y seguras.

Bouderhem et al. \cite{Bouderhem2024} advierten que el marco de trabajo regulatorio actual podría no ser suficiente o apropiado para la naturaleza de la IA. Es necesario contar con más colaboración y diálogo entre las partes interesadas (gobiernos, entes reguladores, desarrolladores, profesionales de la salud y pacientes) para establecer estándares claros y lineamientos para el uso de IA en el ámbito médico. Citan el uso de ChatGPT y la privacidad de los datos como dos escenarios donde la carencia de regulación es notable.
  %%%%%%%%%%%%%%%%%%%%%%%%%%%%%%%%%%%%%%%%%%%%%%%%%%%%%%
%   PI2 - ANÁLISIS DE RESULTADOS
%%%%%%%%%%%%%%%%%%%%%%%%%%%%%%%%%%%%%%%%%%%%%%%%%%%%%%
\subsection{PI2 - ¿Qué postura adoptan los autores para el uso de IA/IAX?} \label{section:es/resultados-pi2}

Se realizó una búsqueda de frases y expresiones claves para identificar la postura de los diversos autores respecto al uso de sistemas IA/IAX. Se confeccionaron dos grupos de posturas: colaborativo y no colaborativo. Se considera \enquote{colaborativo} a todas aquellas expresiones, frases o párrafos que transparenten aceptación, muestren intencionalidad de integrar la tecnología al flujo de trabajo actual de los patólogos y profesionales del área médica, que presenten a la inteligencia artificial como una herramienta o asistente, menciones paradigmas human-in-the-loop, etc. Por otro lado, se considera \enquote{no colaborativo} a todo aquello que implique un enfoque competitivo, de superioridad, rivalidad entre tecnología y humanidad, que planteen la inteligencia artificial como un sustituto al patólogo, etc.

Es importante destacar que no solo se incluyeron posturas de los autores de los estudios primarios, sino también de aquellos a quienes hayan citado en el cuerpo de cada artículo. La síntesis de estos datos permitió obtener el histograma de la Figura \ref{figura:es/rpi2}, en le que se observa una clara tendencia a considerar los sistemas IA/IAX como un complemento para la práctica diaria de los patólogos y el personal médico. Los datos en crudo con los que se confeccionó este gráfico se pueden consultar en el Apéndice \ref{appendix:es/rpi2_datos}.

%- No colaborativo --> competitivo, enfoque de superioridad, rivalidad, reemplazar al personal médico.
%- Colaborativo --> integración entre profesionales, enfoque de IA como herramienta o asistente, paradigmas human-in-the-loop
%En la tabla XXX se eliminaron las referencias bibliograficas del material original para que no se confundan con las referencias de este trabajo. ¿Debería aclarar esto?

\begin{figure}[htbp]
    \centerline{\includegraphics[width=\textwidth]{Imagenes/rpi2.png}}
    \caption{Histograma de posturas acerca del uso de la IA y su integración con los patólogos.}
    \label{figura:es/rpi2}
\end{figure}


  %%%%%%%%%%%%%%%%%%%%%%%%%%%%%%%%%%%%%%%%%%%%%%%%%%%%%%
%   PI3 - ANÁLISIS DE RESULTADOS
%%%%%%%%%%%%%%%%%%%%%%%%%%%%%%%%%%%%%%%%%%%%%%%%%%%%%%
\subsection{PI3 - ¿Cuáles son los algoritmos utilizados en implementaciones de IA/IAX?} \label{section:es/resultados-pi3}

Para dar respuesta a esta pregunta de investigación se extrajeron los algoritmos, técnicas y herramientas mencionados en cada uno de los estudios primarios. Con estos datos, se confeccionaron sendos histogramas para reflejar la frecuencia de uso de cada método, los cuales se visualizan en las Figuras \ref{figura:es/rpi3_ai} y \ref{figura:es/rpi3_xai}. Los datos que se utilizaron en la confección de estos gráficos se adjuntan en el Apéndice \ref{appendix:es/rpi3_datos}.

\begin{figure}[htbp]
\centerline{\includegraphics[width=\textwidth]{Imagenes/rpi3_ai.png}}
\caption{Histograma de uso de algoritmos de inteligencia artificial y marcos de trabajo reportados. En el eje de abscisas, las siglas se corresponden con las nomenclaturas en inglés, a saber: Convolutional Neural Network (CNN); Dense Convolutional Network (DenseNet); Extreme Inception (Xception); Feature Pyramid Network (FPN); Generative Adversarial Networks (GANs); Graph Neural Network (GNN); Inductive Logic Programming (ILP); Region-based Convolutional Neural Network (R-CNN); Residual Neural Network (ResNet); SegNet; Squeeze-and-Excitation network; y Visual Geometry Group network (VGG).}
\label{figura:es/rpi3_ai}
\end{figure}

\begin{figure}[htbp]
\centerline{\includegraphics[width=\textwidth]{Imagenes/rpi3_xai.png}}
\caption{Histograma de uso de algoritmos y técnicas de explicabilidad de inteligencia artificial.}
\label{figura:es/rpi3_xai}
\end{figure}

Se observa en la Figura \ref{figura:es/rpi3_ai} una tendencia destacable a utilizar redes neuronales convolucionales. Esto puede deberse a su buen desempeño en el análisis de imágenes, superior al 90\% de exactitud en las predicciones. Dada la naturaleza de caja negra de estas redes, que dificulta su aplicabilidad en el ámbito médico, se observan otros tipos de propuestas, como por ejemplo el uso de técnicas de lógica difusa. En particular, Sabol et al. \cite{Sabol2020} observan que X-CFCMC, del inglés eXplainable Cumulative Fuzzy Class Membership Criterion, permite obtener sistemas más transparentes y confiables en sus predicciones, conclusión obtenida a partir de contrastar su desempeño contra una CNN convencional y consultar con patólogos de edades variadas.

% cuya arquitectura se compone de una ResNet, un FPN y dos subredes FCN.
Otro hecho interesante radica en la cantidad de frameworks o sistemas propuestos en los estudios primarios. Se destacan entre ellos RetinaNet \cite{Lee2024}, Brea-Net \cite{Liang&Meng2023}, GLoRIA y ChatCAD \cite{PahuddeMortanges2024}, MesoNet \cite{Tran2021}, UV-Net \cite{Dy2024}, CELNet y CLAM \cite{Mohammadi2022} y SpRAy \cite{Sauter2022}. Esta tendencia indica que el uso de soluciones mixtas y la integración de múltiples agentes de inteligencia artificial presentan un desempeño favorable.

En términos de técnicas de explicabilidad, la Figura \ref{figura:es/rpi3_xai} demuestra que las más populares son LIME, SHAP y CAM, en sus múltiples variaciones. Se observa también un interés notable por los algoritmos de explicabilidad en redes neuronales gráficas, pese al uso reducido reportado en la Figura \ref{figura:es/rpi3_ai}. Se destaca GraphLIME como un caso particular de LIME aplicado a este tipo de redes.

La media del histograma de la Figura \ref{figura:es/rpi3_xai} demuestra un claro interés por aumentar la explicabilidad de la inteligencia artificial. Sin embargo, la explicabilidad concebida a partir de la interpretación de sesgos se ha explorado menos en relación con las demás perspectivas. A este respecto, Sauter et al. \cite{Sauter2022} destacan los métodos InsideBias y REvealing VIsual biaSEs (REVISE).
  %%%%%%%%%%%%%%%%%%%%%%%%%%%%%%%%%%%%%%%%%%%%%%%%%%%%%%
%   PI4 - ANÁLISIS DE RESULTADOS
%%%%%%%%%%%%%%%%%%%%%%%%%%%%%%%%%%%%%%%%%%%%%%%%%%%%%%
\subsection{PI4 - ¿Qué aportes se realizaron para la integración de IA/IAX al flujo de trabajo de los patólogos?} \label{section:es/resultados-pi4}

% NPC = nasopharyngeal cancer
Alabi et al. \cite{Alabi2023} concluyen que el aprendizaje automático es capaz de estimar el pronóstico de cáncer nasofaríngeo. Dado que estos algoritmos pueden examinar relaciones complejas entre variables, potencialmente pueden analizar cómo respondieron los pacientes a su tratamiento y predecir el resultado de un nuevo paciente en condiciones similares. Los autores sostienen que estos algoritmos, combinados con técnicas de explicabilidad, proporcionan un potencial prometedor para estratificar la probabilidad de supervivencia en pacientes con cáncer nasofaríngeo.

Dy et al. \cite{Dy2024} convocaron a 90 patólogos especialistas con experiencia en puntuación del índice de proliferación (PI) del Ki-67, un marcador de proliferación que puede ser un factor de pronóstico en tumores neuroendocrinos \cite{defKi67}. Estos especialistas se seleccionaron sin hacer distinción de género, edad o situación laboral. Los autores evaluaron la concordancia de puntuación con y sin IA. Para llevar a cabo este estudio utilizaron UV-Net, una herramienta de aprendizaje profundo desarrollada por la Toronto Metropolitan University que es capaz de diferencia Ki-67 positivo o Ki-67 negativo en secciones de tejido teñidas con inmunohistoquímica. Observan que la mayoría de los convocados consideraron las sugerencias de esta herramienta, la encontraron apropiada y estuvieron de acuerdo en que puede mejorar la precisión en la evaluación de Ki-67. Varios de los encuestados coincidieron en que usarían esta herramienta en la práctica y se manifestaron de acuerdo con implementar asistencia por IA en evaluaciones de Ki-67 en la próxima década.

Bouderhem et al. \cite{Bouderhem2024} advierten que para que los sistemas de IA puedan integrarse correctamente en la atención médica es preciso llevar a cabo una revolución digital, con campañas de concientización y programas educacionales que mitiguen el temor del personal médico de ser reemplazados, y que convenzan a los pacientes de que esta tecnología no les produciría daños. Los autores sostienen que esto es un proceso necesario para construir la confianza pública con los sistemas de inteligencia artificial.

Por otro lado, los autores enumeran una serie de posibles soluciones para regular correctamente el uso de IA en medicina, a saber:

\begin{itemize}
  \item Establecer normas y estándares en el marco de la Organización Mundial de la Salud (OMS).
  \item Fortalecer la supervisión regulatoria.
  \item Promover responsabilidad y transparencia.
  \item Fomentar la autorregulación de la industria.
  \item Fomentar la cooperación internacional.
  \item Fortalecer la ética del uso de datos personales del ámbito médico.
  \item Establecer una \enquote{cultura de IA} que involucre a todas las partes interesadas del ámbito médico.
\end{itemize}

En términos de GNNs, si bien existen métricas y algoritmos para medir la eficiencia de una explicación, Finzel et al. \cite{Finzel2022} creen que las explicaciones verbales pueden aportar información sobre la toma de decisiones de una GNN de manera más humana y natural, y hacen fuerte énfasis en simular las expectativas de los usuarios. Comentan que eligieron este enfoque para permitir una entrada human-in-the-loop en un procedimiento post-hoc, en vez de requerir que las personas enmascaren el gráfico de entrada extraído de acuerdo con conceptos específicos del dominio.

Mohammadi et al. \cite{Mohammadi2022} proponen un modelo interpretable basado en aprendizaje semi-supervisado, a partir de WSI, para diagnosticar cáncer de endometrio. Tras haberlo probado, consultaron con patólogos expertos y obtuvieron observaciones acerca de los mapas de saliencia y la visibilidad de características. Detectaron que los mapas de saliencia mostraban que la estructura epitelial es altamente prominente para la clase maligna, tanto en muestras benignas y malignas. Esto podría prestar confusión en muestras benignas.

Teng et al. \cite{Teng2022} observan que es necesario que el personal médico participe del proceso de diseño de los modelos de IA, no solo para aportar conocimientos y experiencia, sino también para comprender cómo toma decisiones el sistema. Sostienen que este enfoque incrementa notoriamente la confianza de médicos y pacientes, y promueve la implementación de sistemas de diagnóstico asistido en la práctica médica.

Sabol et al. \cite{Sabol2020} desarrollaron una interfaz que provee explicaciones visuales y semánticas extraídas de CFCMC (Cumulative Fuzzy Class Membership Criterion) al que llamaron X-CFCMC (explainable CFCMC), un modelo clasificador basado en lógica difusa. Para medir su desempeño y aceptación, diseñaron una interfaz de usuario que permite a los patólogos examinar partes arbitrarias del corte WSI haciendo clic en el área deseada. La interfaz muestra sus predicciones, y luego el patólogo puede tomar una decisión final seleccionando sendos botones en la pantalla. Sus hallazgos demuestran que los patólogos consideran X-CFCMC más preciso, más riguroso, completo y más confiable que una CNN convencional interpretable. Los autores creen que su sistema propuesto puede contribuir al uso de la IA, principalmente a mejorar la usabilidad y aceptación en la práctica médica.

Cai et al. \cite{Cai2019} condujeron una entrevista semi-estructurada de tres fases con un total de 21 patólogos. En la primera fase, les preguntaron qué tipo de información necesitarían saber acerca de un asistente IA antes de utilizarlo. También se les pidió que describieran cómo se habían incorporado previamente a una tecnología existente o una prueba de diagnóstico en su práctica actual. En la segunda fase, los autores buscaron comprender estas necesidades mientras usaban un asistente IA. Finalmente, en la tercera fase se les preguntó qué información adicional consideraban necesaria conocer para trabajar con la IA de manera eficaz.

Como resultado de la entrevista, algunos de los hallazgos de Cai et al. fueron:

\begin{itemize}
    \item Puede ser útil determinar qué tipos de métricas de rendimiento están acostumbrados a ver los usuarios, con el fin de que las partes interesadas y usuarios finales estén mejor preparados para comprender las medidas empíricas del rendimiento de un asistente IA.
    \item El deseo más común de los entrevistados era conocer las limitaciones y fortalezas del sistema para tenerlas en cuenta durante la toma de decisiones. Varios patólogos asumieron que la IA tendría dificultades con los mismos casos especiales con los que ellos mismos luchan, aunque se dieron más crédito por considerarse capaces de manejar adecuadamente estos casos.
    \item Los participantes manifestaron desear que el algoritmo tuviera estilos de diagnóstico similares a los suyos; que pueda ser, por ejemplo, más liberal o más conservador, al asignar grados de cáncer de mayor gravedad.
    \item Dada la subjetividad inherente a la clasificación del cáncer, los participantes deseaban saber qué fuentes médicas utilizaba el algoritmo. Mientras que los patólogos suelen conocer la experiencia y los antecedentes de sus colegas, el conocimiento clínico de una IA es ciertamente opaco.
    \item Algunos patólogos imaginaron ensamblar un conjunto de casos de verdad fundamental y comparar sus diagnósticos con los de la IA en una fase de calibración humano-IA.
\end{itemize}

Shawi et al. \cite{Shawi2022} encontraron que los médicos no confían en las predicciones de un modelo de caja negra, e incluso prefieren disponer de modelos de caja blanca, aunque tengan un rendimiento menor.

Ayorinde et al. \cite{Ayorinde2022} plantean un paradigma colaborativo en el que los equipos de desarrollo mantengan un diálogo continuo y fluido con los médicos, con el propósito de difundir las ventajas y limitaciones de los diseños elegidos, conocer los problemas comunes de la IA respecto a las fases de entrenamiento y despliegue, y aumentar la confianza en estos sistemas.

% digital scans -- lo traduje como muestras digitales
Gu et al. \cite{Gu2023NaviPath} detectaron que a los patólogos les insume más tiempo examinar una muestra digital que a través de microscopios. La dificultad radica en la navegación de los cortes histológicos digitales porque tienen resoluciones extremadamente altas del orden de \((10^6)^2\) píxeles, mientras que los monitores de escritorio alcanzan \(8.3\times10^6\) píxeles si son 4K UHD. Para atender este problema, los autores consultaron con seis patólogos profesionales de dos centros médicos diferentes y con la información recolectada desarrollaron NaviPath: un sistema de navegación colaborativo humano-IA. Los autores, además, aseguran que este sistema cierra la brecha entre la IA y los patólogos porque integra conocimientos del dominio médico, lo cual puede mejorar la integración al flujo de trabajo cotidiano de los patólogos.

Wenzel y Wiegand \cite{Wenzel&Wiegand2020} observaron que los patólogos prefieren procesos de toma de decisión que sean transparentes y manifiestan incomodidad con los métodos de caja negra. Wenzel y Wiegand advierten que, pese a los avances en técnicas IAX, se debe validar que la explicación es plausible y que sin un chequeo de validez el modelo en su totalidad no es confiable. Para llevar a cabo esta validación, sostienen que los expertos del dominio deben verificar que los objetivos del sistema sean relevantes en la práctica, además de la validación técnica convencional.

Los hallazgos de Gallo et al. \cite{Gallo2023} sugieren que una red neuronal entrenada para detectar carcinoma de próstata en biopsias de núcleo, utiliza características morfológicas similares a las de los patólogos, con aumentos medianos (100x–200x) y bajos (20x–50x) en las muestras.

Dolezal et al. \cite{Dolezal2024} observan que cuando sea crean herramientas analíticas de computación para uso clínico, es importante considerar que los usuarios finales podrían no disponer de conocimiento computacional suficiente. En este sentido, incorporar una interfaz de usuario (GUI) puede hacer el sistema más accesible y facilitar el despliegue de herramientas de aprendizaje profundo en el ámbito médico.

Vanitha et al \cite{Vanitha2024} postulan que el uso de Grad-CAM como técnica de explicabilidad es un paso crucial para adoptar la inteligencia artificial en la práctica clínica. Sostienen que la visualización a través de mapas de calor permite a los médicos verificar visualmente el fundamento de las predicciones de las CNN y facilitan un entendimiento más profundo de su funcionamiento, así como también aumenta la confianza en el sistema. Tales explicaciones también tienen provecho en ámbitos educativos, donde los médicos profesionales pueden observar cómo los modelos disciernen matices sutiles en las imágenes histopatológicas que pueden pasarse por alto en otros exámenes.

Dörrich et al. \cite{Doerrich2023} también argumentan a favor de las técnicas CAM, ya que encontraron que son útiles para estudiar y comparar el comportamiento entre dos redes neuronales. Comentan, además, que se ha demostrado que presentar Grad-CAM como información adicional junto con WSI puede mejorar la precisión de la clasificación de los patólogos.

Otros autores también perciben los beneficios de utilizar Grad-CAM. Shovon et al. \cite{Shovon2023} utilizan Grad-CAM para explicar un modelo de clasificación de cáncer de mama, logrando así un sistema más transparente para los usuarios. Por su parte, Liang y Meng \cite{Liang&Meng2023} lo utilizan para aportar explicabilidad a Brea-Net, otra red de clasificación de cáncer de pecho, especializada para sets de datos desbalanceados. Por último, Praetorius et al. \cite{Praetorius2023} utilizan Grad-CAM en red IMFSegNet, red capaz de cuantificar la distribución espacial de grasa intramuscular en secciones histológicas, para dilucidar visualmente el proceso de toma de decisiones del modelo.

Basaad et al. \cite{Basaad2024} proponen BERTGNN, un sistema que GNN con LLMs y es capaz de predecir cáncer de mama metástico. La combinación de ambas tecnologías permite procesar la semántica de reportes histopatológicos y descubrir patrones o dependencias cruciales para predecir esta enfermedad. Disponer de esta información, facilitaría a los patólogos a tomar decisiones y mejorar los resultados de los pacientes. Cabe destacar que los autores declaran que su aporte es solo el paso inicial y que es imperativo confirmar su eficacia y confiabilidad.

Palkar et al. \cite{Palkar2024} sostienen que la interpretabilidad sirve como principio rector para la implementación de la inteligencia artificial en el ámbito sanitario, pese a que aún falta mucho por investigar de IAX en este campo. Siguiendo este lineamiento, los autores destacan que no solo refuerza la confianza, sino también alienta a los usuarios a valerse de los beneficios de esta tecnología y a permanecer atentos a sesgos potenciales y consideraciones éticas de su uso.

Nasir et al. \cite{Nasir2024} presentan un nuevo marco de trabajo ético constituido por seis pilares: sensibilidad, evaluación, enfoque en el usuario, responsabilidad, beneficencia y seguridad. La incorporación de estos pilares sienta la base para sistemas de IA holísticos y seguros. Cada pilar, a su vez, se contextualiza dentro de un panorama ético de la IA. Este marco está diseñado para regular el impacto de estas tecnologías en la vida de las personas, garantizando el beneficio social, la protección de los derechos humanos y el respeto por la privacidad y la autonomía de las personas.

Aziz et al. \cite{Aziz2023} presentan un modelo llamado IVNet (ImageNet-VGG16) para diagnóstico en tiempo real de cáncer de mama en entornos hospitalarios. Utilizando transferencia de aprendizaje, el modelo analiza imágenes histopatológicas e identifica las células afectadas. Los autores incorporan al sistema una interfaz de usuario para monitoreo en tiempo real, lo que permite a los médicos planificar el tratamiento y realizar pronósticos.

Sloboda et al. \cite{Sloboda2024} proponen la arquitectura xAI-CycleGAN, la cual está basada en CycleGAN y que pretende mejorar la tasa de convergencia y la calidad de la imagen en tareas de transformación de imagen a imagen sin supervisión.

Alsubai et al. \cite{Alsubai2024} presentan el modelo Inception-ResNetV2 que permite clasificar cáncer de colon y pulmón en muestras histopatológicas de cáncer. Los autores utilizan el método SHAP de explicabilidad, que ilustra el aporte individual de cada característica en las predicciones del modelo. De esta forma, aumentan la transparencia y fomentan una comprensión más profunda de los procesos de toma de decisiones del sistema.

Civit-Masot et al. \cite{CivitMasot2024} proponen una red convolucional para clasificar muestras de cáncer cervical. Incorporan técnicas IAX que presentan un reporte detallado que ayuda al personal médico en tareas de diagnóstico, el cual incluye una representación gráfica de mapas de calor que muestra las zonas de la imagen que condicionaron el resultado del modelo y además un reporte en texto que ilustra el grado de certeza en la predicción provista por el sistema. Los autores sostienen que este generador de reportes es esencial para que los patólogos puedan aseverar la validez de los resultados del modelo.

Ullah et al. \cite{Ullah2024} concluyen que es crucial que los desarrolladores de IA, los profesionales del área médica y los expertos del dominio colaboren entre sí para integrar exitosamente los algoritmos de LLM en las tareas de diagnóstico, ya sea participando en el ajuste de los conjuntos de datos, en los mecanismos de validación o ambos. Este esfuerzo conjunto puede mejorar el entrenamiento del modelo y asegurar que se alinee con el flujo de trabajo clínico actual.

Finzel et al. \cite{Finzel2024} proponen un marco de trabajo bidireccional, en el que las personas reciben las explicaciones de una decisión de clasificación, aplicable a varios tipos de modelo. Estas explicaciones se manifiestan en forma unimodal o multimodal. El profesional, ya sea experto o novato, puede explorar y comprender el modelo, su clasificación para ejemplos puntuales y las explicaciones dadas en la interfaz, y ajustar o afinar al sistema. Los autores destacan que esta retroalimentación correctiva es benéfica para mejorar el desempeño del modelo en circuitos casos, pero advierten que también podría dañarlo si la decisión correctiva está sesgada, es incierta o contradice otras partes del modelo, necesarias para detectar otras subclases de muestras.

Vanea et al. \cite{Vanea2024} presentan HAPPY (del inglés Histology Analysis Pipeline.PY), un método para cuantificar células y tejidos microanatómicos a través de cortes histológicos WSI de placenta teñidos con hematoxilina y eosina (H\&E), técnica de tinción ampliamente utilizada en muestras de núcleos celulares \cite{usoH&E}. Los autores sostienen que esta herramienta puede facilitar estudios morfo-métricos a gran escala sobre la histología de la placenta, una tarea que actualmente es manual y requiere mucha mano de obra experta.

Pahud de Mortanges et al. \cite{PahuddeMortanges2024} proponen un orquestador IAX cuyo propósito es ayudar a los clínicos con la sinopsis multimodal y longitudinal de los datos, las predicciones de IA y la explicabilidad correspondiente. Aunque los autores no proporcionan una implementación completa del orquestador IAX, describen cómo podría lograrse a raíz de los desarrollos actuales LLM, y sugieren propiedades, funcionalidades y métricas deseables.

Tabatabaei et al. \cite{Tabatabaei2023} comentan que CBMIR (del inglés Content-Based Medical Image Retrieval) favorece la confianza de los patólogos en los resultados de una predicción, ya que no solo tendrán a disposición una segunda opinión, sino también podrán buscar patrones en tejidos anteriores. Dada la naturaleza explicable de CBMIR, los médicos pueden comprender la forma en que el sistema arribó a un diagnóstico o recomendación determinados. Los autores destacan que, a diferencia de los modelos de clasificación habituales, CBMIR está centrado en el patólogo.

Gu et al. \cite{Gu2021Lessons}, a través de sesiones de trabajo con ocho patólogos de un centro médico, consiguen reunir seis aprendizajes fundamentales para integrar la IA en la práctica clínica.

\begin{itemize}
    \item El sistema IA debe proporcionar junto con sus resultados detalles específicos del caso de estudio particular, para que el médico pueda evidenciar qué factores lo condujeron a elaborar dicho resultado.
    \item Dado que el diagnóstico suele acarrear más de una tarea, el sistema IA necesita constantemente tomar en consideración la nueva información que pueda extraer de las entradas del usuario.
    \item El tiempo de las tareas médicas suele ser crítico. Los beneficios potenciales de utilizar IA deben sopesarse junto a la cantidad de esfuerzo adicional que pueda demandar y la información proporcionada.
    \item La IA debe ayudar al personal médico a limitarse a pequeñas regiones de un gran espacio de tareas, así como a filtrar información en regiones específicas.
    \item En casos en que los médicos puedan proveer etiquetas durante su flujo de trabajo, el sistema IA debería proveer retroalimentación explícita de cómo mejora el modelo en consecuencia, puesto que motivaría al personal a realimentar más y mejor al sistema.
    \item En tareas médicas de alto riesgo, la IA debe proporcionar información que permita validar su nivel de confianza.
\end{itemize}

A raíz de los aprendizajes adquiridos en \cite{Gu2021Lessons}, Gu et al. proponen xPath \cite{Gu2023XPath}, una herramienta de diagnóstico colaborativa humano-IA que pretende asistir a los patólogos e integrarse a su flujo de trabajo mediante tres características fundamentales:

\begin{itemize}
    \item Reporta sobre múltiples criterios patológicos calculados mediante IA.
    \item Presenta evidencia trazable de cada reporte, volviendo al sistema más explicable y confiable.
    \item Permite a los patólogos realizar diagnósticos en un flujo de trabajo similar al de su práctica diaria.
\end{itemize}

Tschandl et al. \cite{Tschandl2020} manifiestan que sus hallazgos sugieren riesgos indeseados. Observan que si un grupo de evaluadores logra la confianza necesaria para valerse de la asistencia de la IA, también son vulnerables a desempeñarse por debajo de su desempeño esperado ante fallas en la IA.

Klauschen et al. \cite{Klauschen2024} argumentan que los sistemas IA/IAX pueden servir para asistir en la toma de decisiones de las personas, para construir un sistema autónomo de toma de decisión, o bien para extraer información científica. En el primer caso, los autores sostienen que incorporar soluciones basadas en redes neuronales, combinadas con mapas de atención o LRP, son la mejor opción para consultas en tiempo real. En el segundo caso, Klauschen et al. argumentan que dada la creciente escasez mundial de patólogos las herramientas de diagnóstico autónomo pueden ser un gran aporte a la comunidad, pero es complejo una aprobación regulatoria en este contexto. Además, esta clase de sistemas debe demostrar robustez y precisión excepcional.

En el tercer caso, Klauschen et al. sostienen que las técnicas IAX proporcionan una extensión poderosa a la bioinformática tradicional. Dado que los modelos de IA pueden aprender relaciones complejas, no lineales, entre múltiples variables, pueden combinarse con IAX para elaborar hipótesis dirigidas por datos.

Amato et al. \cite{Amato2024} proponen una metodología de computación granular para clasificar imágenes histopatológicas, que se basa en aprendizaje profundo y utiliza un mapa autoorganizado para generar una estructura granular con los datos de aprendizaje. Los autores creen que su propuesta mejora las técnicas existentes creando una cadena de procesamiento que puede visualizarse con facilidad, y sostienen que esto puede ayudar tanto a los usuarios como a los desarrolladores de sistemas.


  %%%%%%%%%%%%%%%%%%%%%%%%%%%%%%%%%%%%%%%%%%%%%%%%%%%%%%
%   PI5 - ANÁLISIS DE RESULTADOS
%%%%%%%%%%%%%%%%%%%%%%%%%%%%%%%%%%%%%%%%%%%%%%%%%%%%%%
\subsection{PI5 - ¿Qué aspectos se declaran pendientes de investigación en los artículos analizados?} \label{section:es/resultados-pi5}

Finzel et al. \cite{Finzel2022} observan una carencia de explicadores de redes gráficas que generen reportes verbales similares al lenguaje natural. La gran mayoría de explicadores buscan la subestructura con mayor relevancia según pérdidas, métricas y suposiciones de diseño.

Yin et al. \cite{Yin2020} sostienen que para alcanzar una integración eficaz con los patólogos en la toma de decisiones diarias, además de lograr sistemas interpretables se debería tomar en cuenta la causabilidad, concepto propuesto por Holzinger et al. \cite{Holzinger2019}, la cual se mide en términos de efectividad, eficiencia, satisfacción relacionada con la comprensión causal y su transparencia para un usuario. Tratándose de cuestiones de calidad de la explicación, incorporar este factor permite a un patólogo experto analizar la causalidad de una enfermedad concreta. Yin et al. declaran además que, si bien su modelo propuesto es interpretable, incorporar causabilidad es una meta futura que hará que su sistema sea útil y apto para el uso de los patólogos.

Sabol et al. \cite{Sabol2020} observan que su clasificador explicable de cáncer colorrectal debe someterse a pruebas clínicas. Destacan que en la práctica médica los datos suelen estar desbalanceados, ser heterogéneos e incluso inexactos, y declaran que sus próximos pasos son evaluar cómo responde el modelo con datos imperfectos.

Shawi et al. \cite{Shawi2022} observan varios puntos de mejora para su clasificador interpretable de cáncer de mama. Sugieren, por un lado, que un posible próximo paso es extender su trabajo y considerar técnicas diversas de interpretabilidad que extraigan conceptos relevantes de un conjunto de datos. Por otro lado, otra posible dirección es expandir la aplicación de su propuesta a otros problemas de la histopatología y desarrollar herramientas que permitan a los patólogos a diagnosticar más rápido y mejor.

Roscher et al. \cite{Roscher2020} advierten que la gran mayoría de publicaciones que usan aprendizaje automático no se ocupa de aspectos de interpretabilidad o explicabilidad. Los autores manifiestan interés en que la encuesta que llevaron a cabo proporcione ideas y metodologías acerca de cómo extraer información relevante sobre un objeto de estudio puntual. Sostienen además que la inferencia causal puede desempeñar un rol crucial en explicabilidad, pero requiere más investigación para adoptar su uso.

Nazar et al. \cite{Nazar2021} observan que, dado que IAX es un campo de investigación relativamente reciente, aún quedan desafíos pendientes por resolver en el ámbito médico. Las dificultades observadas en su revisión incluyen mejorar y agregar nuevos
enfoques de explicabilidad en ciertos dominios, y aumentar la interpretabilidad y la precisión del modelo. Los autores sostienen que se le debe dar prioridad a estos puntos en el futuro.

Palatnik de Sousa et al. \cite{PalatnikdeSousa2019} declaran pendiente realizar estudios futuros con la colaboración de patólogos expertos, centrados en evaluar más en profundidad las explicaciones de IA y aplicar su propuesta agnóstica del modelo a otros conjuntos de imágenes de diferentes dominios médicos.

Tschandl et al. \cite{Tschandl2020} concluyen que las herramientas de refinamiento centradas en las personas mejoran la experiencia de usuario de CBIR (del inglés content-based image retrieval) en patología. Fundamentados en esto, sostienen que deberían estudiarse más variedad de diseños y combinaciones de colaboración humano-IA y que está pendiente el evaluar su impacto en la prestación de servicios de salud.
 
Vanitha et al. \cite{Vanitha2024} declaran que su arquitectura propuesta, basada en la combinación de MobileNet, Xception y Grad-CAM, es prometedora. Sugiere como futuras investigaciones explorar combinar este marco de trabajo a otros tipos de cáncer o a tareas de diagnóstico más complejas para incrementar su rango de aplicaciones. La búsqueda de estos avances podría allanar el camino para obtener procesos más automatizados, precisos y confiables, que mejoren el bienestar de los pacientes en oncología.

Dörrich et al. \cite{Doerrich2023} declaran que el futuro explorarán una estrategia alternativa para anotar los núcleos celulares con un mayor número de clases distintas. Las investigaciones futuras también pueden contribuir a identificar otras características útiles para reconocer subtipos de cáncer. Los autores sugieren, además, que se podrían aplicar técnicas adicionales de IA explicable para estudiar a fondo las características discriminantes de cada clase.

Según Klauschen et al. \cite{Klauschen2024} es necesario tomar acciones con respecto a la privacidad de los datos, tanto durante el entrenamiento como el despliegue del modelo. Destacan que el aprendizaje federado se ha mostrado prometedor en este sentido.

Basaad et al. \cite{Basaad2024} postulan que su sistema propuesto es un paso inicial en la detección de cáncer de mama metástico y que es preciso hacer validaciones con datos clínicos. Sostienen que las investigaciones futuras que puedan realizarse serán esenciales para evaluar la respuesta del sistema en poblaciones diversas de pacientes y su integración a la práctica clínica.

Palkar et al. \cite{Palkar2024} identifican puntos de mejora y posibles investigaciones futuras relacionadas a su modelo interpretable pronosticador de glioma. Por un lado, sostienen que se puede buscar colaboración internacional para combinar datos de distintos países y crear una iniciativa global de investigación de glioma. Implementar un sistema en la nube puede favorecer esta iniciativa y aumentar la escalabilidad. Los autores añaden que se llevarán a cabo validaciones rigurosas con pruebas clínicas y que los planes de estudio adecuados reducirán la brecha entre informáticos y médicos expertos. Por último, destacan que las consideraciones éticas seguirán siendo primordiales, tanto en sentido de la privacidad del paciente como de la seguridad de los datos.

Civit-Masot et al. \cite{CivitMasot2024} observan que es pertinente consolidar los resultados de su modelo de clasificación de cáncer cervical usando un volumen de datos mayor, con imágenes de diferentes hospitales, para evitar el uso de técnicas de aumento de datos. Por otro lado, gracias a los tiempos de ejecución alcanzados, los autores sostienen que su propuesta da pie a una posible rama de investigación como continuación de su trabajo, cuyo enfoque se base en integrar el clasificador en sistemas embebidos.

Sabol et al. \cite{Sabol2019} observan que en futuras investigaciones se concentrarán en evaluar cómo la semántica de la información afecta al proceso de toma de decisión de los doctores utilizando diferentes métodos de aprendizaje automático interpretable. 

Finzel et al. \cite{Finzel2024} comentan que un desafío pendiente es desarrollar un marco de trabajo de explicaciones integrales, que sea dinámico, y que permita a los usuarios cambiar entre explicaciones locales o globales, seleccionar diferentes modalidades de explicación acorde al dominio, entre otras características. Observan además una brecha entre la investigación basada en aplicaciones y las centradas en el humano (human-centered) y añaden que falta evidencia empírica que demuestre la utilidad real de los modelos explicativos, tanto para expertos como novatos. Los autores creen que proporcionar marcos de trabajo centrados en las personas allanará el camino para integrar estas tecnologías y fomentará la confianza y comprensión de los usuarios.

Pahud de Mortanges et al. \cite{PahuddeMortanges2024} destacan que aún existen desafíos no resueltos que impiden aprovechar al máximo el potencial de IAX. Algunos de ellos incluyen la falta de estudios que traten sobre enriquecer sistemas IAX con otros tipos de datos clínicos, lo que se conoce como IAX multimodal, o usar conjuntos de datos longitudinales. Postulan que el desarrollo de IAX multimodal y longitudinal es crucial y necesario en muchos flujos de trabajo clínicos.

Respecto a su orquestador IAX propuesto, Pahud de Mortanges et al. \cite{PahuddeMortanges2024} destacan que, debido a las responsabilidades atribuidas a dicho sistema en la coordinación de sistemas IA/IAX específicos, aún es necesario abordar varios desafíos para garantizar su confiabilidad y la seguridad de los datos.

Cai et al. \cite{Cai2019} destacan que existen claras oportunidades para diseñar y probar recursos basados en los hallazgos de su estudio, por ejemplo acerca de cómo los asistentes de IA pueden dar forma a las prácticas laborales o ayudar a las personas a desarrollar estrategias más precisas, cómo inculcar modelos mentales más prácticos, o cuáles son los efectos en la confianza del usuario. Advierten, sin embargo, que su estudio se focalizó en recolectar las impresiones iniciales de los usuarios, y que la relación de uno con la herramienta puede evolucionar con el tiempo.

Gu et al. \cite{Gu2023XPath} comentan que su propuesta xPath tiene puntos de mejora, y observan que los trabajos futuros deberían considerar usar imágenes de múltiples centros de salud, convocar participantes con experiencias variadas, y conducir comparaciones con estudios a largo plazo, constatando el desempeño de xPath con el microscopio óptico. Los autores sostienen que recopilar toda esta información les permitirá validar el desempeño de su propuesta de una forma más integral.

%Por último, a través de los hallazgos de Gu et al. \cite{Gu2021Lessons} y Gu et al. \cite{Gu2023XPath}, Gu et al. 
Por último, a través de los hallazgos presentados en \cite{Gu2021Lessons} y \cite{Gu2023XPath}, Gu et al. \cite{Gu2023NaviPath} observan que, si bien los doctores prefieren los diseños simples, estos generalmente acarrean una pérdida en la comprensión de información y podrían no ser suficientemente informativos para su flujo de trabajo y la toma de decisiones. Los autores sugieren que las próximas investigaciones de HCI estudien qué información se debe conservar y qué se debe descartar mediante estudios empíricos.
  %\input{Secciones/es/04-6_PI6}
\newpage
%%%%%%%%%%%%%%%%%%%%%%%%%%%%%%%%%%%%%%%%%%%%%%%%%%%%%%
%   CONCLUSIONES
%%%%%%%%%%%%%%%%%%%%%%%%%%%%%%%%%%%%%%%%%%%%%%%%%%%%%%
\section{Conclusiones} \label{section:es/conclusiones}

% Though the implementation of IA/IAX seems promising, this article brings to light the challenges that must be solved so that an effective implementation of these technologies may take place in histopathology. These difficulties are not limited to technical ones only, but they also cover other equally important perspectives.
%The coupling workflow is the main factor that must be considered when designing an AI system. It is necessary to establish a list of requirements with specialists and pathologists with the purpose of understanding their needs and finding the best way they can benefit themselves from these technologies. IA/IAX systems must be conceived as assistants to pathologists, and allow for sufficient transparency so their results can be supervised. 
%It is possible that, to achieve an adequate integration with the workflow, it is vital to incorporate the expert-in-the-loop approach. Another possibility is to consider IA/IAX algorithms like components of a more complex system that implements pieces of traditional programming and results expressed in natural language. With the recent advent of Large Language Models (LLM), the aforementioned bonding between pathologists and xAI systems could be enhanced.
%IA/IAX systems are highly sensitive to the quality of their data. Adequately classified samples are crucial for the success of the AI training. Due to this task being exhaustive, it may be accelerated by resorting to weakly supervised learning methods. However, to obtain a precise classification in oncology and renal histopathology sometimes is complex. In the face of this difficulty, designing AI systems from an approach centered in causability instead of explainability might contribute significantly to the professionals. 
%There is still a necessity for a set of regulations that guarantee a standard in relation to the testing and the use of AI systems in histopathology, in the case that these systems require the utilization of patient-related data for training.
%The cost of implementation is also a limiting factor in some organizations. Against this, open-source alternatives might become the starting point for the implementation of AI-based solutions.
%As for the next steps of this article, this work will be expanded to create a systematic mapping study of literature. It is expected for this to be the foundations of a future investigation.

Pese a que la implementación de IA/IAX es prometedora, en este trabajo se evidencian los desafíos que se deben resolver para una implementación efectiva de estas tecnologías en histopatología. Estas dificultades no se limitan a lo técnico, sino también a aspectos regulatorios y de integración.

La integración al flujo de trabajo es el principal factor que se debe considerar durante el diseño de un sistema de IA. Si bien se han demostrado avances en la recolección de requisitos con especialistas y patólogos, tarea que permite comprender sus necesidades y cómo podrían beneficiarse de la IA, la falta de lenguaje común entre los desarrolladores y el personal médico sigue siendo una brecha limitante. Aspirar a disponer de capacitaciones o niveles de estudio integrales que favorezcan la comunicación entre ambos tipos de profesionales puede ser útil para reducir esta brecha.

La tendencia actual, según los datos reportados en PI3, es favorecer un ámbito colaborativo entre la IA y los profesionales del área médica. Pese al esfuerzo por parte de algunos autores de reemplazar al patólogo por sistemas de inteligencia artificial, esta tarea es riesgosa y poco aconsejable debido a los sesgos que pueden cometerse y los daños potenciales que pueden sufrir los pacientes. Por lo tanto, es primordial que los sistemas IA/IAX sean concebidos como asistentes supervisables, explicables y transparentes, para que los patólogos puedan beneficiarse de su uso en la práctica diaria. No obstante, se ha reportado el riesgo de que el personal puede confiar en exceso en estas tecnologías. Teniendo este factor en cuenta, es imprescindible llevar adelante estudios exhaustivos a mediano plazo que permitan evaluar que el uso de IA en tareas de diagnóstico no sea contraproducente.

%Es probable que para lograr una integración adecuada con el flujo de trabajo sea preciso incorporar el enfoque expert-in-the-loop. Otra posibilidad es plantear los IA/IAX algorithms como componentes de un sistema màs complejo que implemente pieces of traditional programming y resultados expresados in natural language.

Los sistemas IA/IAX son altamente sensibles a la calidad de sus datos. Si bien las técnicas de aumento de datos y el aprendizaje semi-supervisado han demostrado buenos resultados, es aconsejable contar con conjuntos de datos recolectados de distintos centros médicos y digitalizados con equipamiento diferente para que las predicciones sean lo más fidedignas posible. Una posible solución arquitectónica a esta necesidad es implementar aprendizaje federado o implementaciones en la nube. Independientemente de cuál sea la implementación final, se debe velar en todo momento por garantizar la privacidad de los pacientes y la seguridad de sus datos y los del sistema.

Ninguno de los estudios primarios encontrados inspecciona el impacto ambiental de entrenar herramientas de diagnóstico basadas en IA, orientadas a un uso masivo en la práctica diaria. Esta cuestión debería ser explorada para poder aspirar a una arquitectura y orquestación eficientes.

El costo de implementación es un factor limitante en algunas organizaciones. Frente a esto, las alternativas de software libre pueden ser el punto de partida para implementar sistemas basados en IA de uso masivo.

Dado el crecimiento de las herramientas de lenguaje natural, es preciso contar con reglamentación actualizada que garantice la seguridad de los pacientes antes el uso y estudio de estas tecnologías en tareas de diagnóstico.

Se aconseja explorar más en profundidad el potencial de soluciones basadas en lógica difusa, debido a su naturaleza interpretable.

Pese a las propuestas y esfuerzo de varios autores, aún hay una notable carencia de definiciones consensuadas y un marco teórico uniforme para las técnicas IAX. Esto se manifiesta sobre todo en el hecho de que las palabras \enquote{interpretabilidad} y \enquote{explicabilidad} suelen usarse indistintamente en las investigaciones como si fueran sinónimos, característica con la que algunos autores no están de acuerdo.

%TODO: Temas interesantes a plantear
% - La cuestión reglamentaria
% - La falta de definiciones concensuadas y marco teórico uniforme.
% - Ningún artículo plantea el impacto ambiental que ocasiona el entrenamiento de algoritmos de IA. El aprendizaje federado podría ser una manera eficiente de perjudicar lo menos posible??
% - Pese a los esfuerzo por parte de algunos autores de reemplazar al patólogo por una inteligencia artificial, esta tarea es riesgosa y poco aconsejable, dados los sesgos que pueden cometerse y los daños potenciales que pueden sufrir los pacientes. La forma más acopable al flujo de trabajo es conseguir un asistente tecnológico que aporte verdadero valor a la práctica diaria.
% - No todos las muestras por región tiene los mismos patrones de tejido, lo que podría ocasionar en limitaciones de aprendizaje. Una buena forma de mitigar este problema es utilizar un sistema inteligente o multiagente que elabore conclusiones con la muestra y con los datos del paciente. (¿No había un artículo que planteaba esto? ¿Era estudio primario?). Esto podría ser un paso futuro de investigación.
% - El potencial de lógica difusa como sistema interpretable y extractor de conclusiones (ver TP de SMA).




\newpage
\section{Apéndices}






%%%%%%%%% SECTION %%%%%%%%%
% \subsection{Ambigüedad de la terminología IAX, reportada por Graziani et al.} \label{appendix:es/graziani-terminologia-ambigua}

% % Se cita en la Figura \ref{figura:es/xai-terminologia-ambigua} el diagrama construído por Graziani et al. en el que se demuestra las múltiples acepciones que tiene un mismo vocablo según el dominio \cite{Graziani2022}.

% \begin{figure}[h]
%   \centerline{\includegraphics[width=\textwidth]{Imagenes/XAI-terminologia-ambigua.png}}
%   \caption{Diagrama propuesto por Graziani et al., que ilustra la ambigüedad inherente a los términos de XAI según el dominio \cite{Graziani2022}.}
%   \label{figura:es/xai-terminologia-ambigua}
%   \end{figure}










  %%%%%%%%% SECTION %%%%%%%%%
\subsection{Glosario de términos} \label{appendix:es/glosario}



\subsubsection{Términos específicos de informática}

\textbf{Aprendizaje automático (Machine Learning)}: \textit{abrev. ML}. \enquote{Es una rama de la inteligencia artificial que estudia como dotar a las máquinas de capacidad de aprendizaje, basándose en algoritmos capaces de identificar patrones en grandes bases de datos y aprender de ellos} \cite{tiposIA-ML-DL}.  

\textbf{Aprendizaje profundo (Deep Learning)}: \textit{abrev. DL}. Es un subconjunto del aprendizaje automático que usa redes neuronales de múltiples capas para simular la toma de decisiones que realiza el cerebro humano \cite{defDeepLearning}.

\textbf{Set o conjunto de datos (Dataset)}: En el ámbito de IA, se le denomina así al conjunto de datos que se utiliza durante el entrenamiento y pruebas de un modelo de IA.

\textbf{Aumento de datos (Data augmentation)}: Son un conjunto de técnicas y procesos que permiten generar artificialmente nuevos datos a partir de otros existentes \cite{defDataAugmentation}. Aplicadas a imágenes, se las suele clasificar en dos tipos: transformaciones geométricas y transformaciones fotométricas. Las primeras consisten en modificar el espacio y disposición de la imagen, mientras que las segundas implican modificar su distribución de color \cite{tiposDataAugmentation}.

\textbf{Aprendizaje supervisado (Supervised learning)}: Es una forma de entrenamiento de IA que utiliza datos muy estructurados y completamente etiquetados para elaborar resultados consistentes \cite{defDeepLearning}.

\textbf{Aprendizaje semi-supervisado (Semi-supervised o weakly supervised learning)}: Es un conjunto de técnicas que modifican o complementan el aprendizaje supervisado incorporando datos no etiquetados \cite{defSemiSupervised}.

\textbf{Red neuronal artificial (Artificial Neural Network)}: Una red neuronal artificial, también llamada solo red neuronal, es una estructura de nodos, llamadas neuronas, interconectados entre sí, cada uno con un valor, denominado peso. Esta estructura consta como mínimo de una serie de capas de entrada, al menos una capa intermedia, que se la conoce como capa oculta, y una última capa de neuronas de salida. A través de estas interconexiones, y mediante cálculos matemáticos, es posible convertir datos de entrada en otros de salida. Los pesos se ajustan en forma automática durante el entrenamiento de la red \cite{defRedNeuronal}.

\textbf{Red neuronal convolucional (Convolutional Neural Network)}: \textit{abrev. CNN}. Son un tipo de red neuronal que se suele utilizar en reconocimiento de imagen y visión computarizada. Estas redes utilizan operaciones matriciales de convolución para identificar patrones en una imagen \cite{defCNN}.

\textbf{Mapas de saliencia (Saliency maps)}: El concepto de mapa de saliencia, en esencia, se refiere al conjunto de características de un objeto que captan la atención de las personas. Estas características pueden estar relacionadas con el color, la forma, tamaño, iluminación, brillo y profundidad del objeto \cite{defMapaSaliencia}. Es un concepto que se utiliza como base para la extracción de características e interpretabilidad de redes neuronales convolucionales \cite{usosMapaSaliencia}.

\textbf{Modelos de lenguaje extensos (Large-Language modelos)}: \textit{abrev. LLM}. \enquote{Es un modelo estadístico de lenguaje entrenado con una gran cantidad de datos que puede utilizarse para generar y traducir texto y otros tipos de contenido, así como para llevar a cabo otras tareas de procesamiento del lenguaje natural (PLN)} \cite{defLLM-Google}.

\textbf{Procesamiento de lenguaje natural (Natural Language Processing)}: \textit{abrev. NLP}. Es un subcampo de la informática y la inteligencia artificial que permite a las computadoras y dispositivos digitales reconocer, comprender y generar texto y voz. Para lograr este cometido, se recurren a modelos estadísticos y modelos de aprendizaje automático \cite{defProcLengNatu}.

\textbf{Lógica difusa (Fuzzy Logic)}: Es un modelo de razonamiento lógico ambiguo, impreciso y multivariable, en el que los valores de verdad se interpretan en función de grados o niveles de verdad \cite{defFuzzyLogic}. Este concepto tiene aplicación en informática como parte del diseño de sistemas de control y predicción, definiendo funciones de pertenencia que convierte un conjunto de datos de entrada en otros de salida \cite{usosFuzzyLogic}.

\textbf{Integración Humano-Computadora (Human-Computer Integration)}: \textit{abrev. HCI}. Se refiere al conjunto de técnicas de diseño e implementación de sistemas informáticos con los que los usuarios pueden interactuar \cite{defHCI}. Para que exista una integración exitosa con el usuario, se debe velar por lograr la funcionalidad deseada y maximizar la usabilidad del sistema.

\textbf{Sistema de IA intervenido por expertos (expert-in-the-loop o human-in-the-loop)}: \enquote{Es un enfoque colaborativo que integra la aportación humana y la experiencia en el ciclo de vida de los sistemas de aprendizaje automático e inteligencia artificial. Los seres humanos participan activamente en el entrenamiento, la evaluación o el funcionamiento de los modelos de aprendizaje automático ofreciendo directrices, comentarios y anotaciones valiosos} \cite{defHumanInTheLoop}.



\subsubsection{Términos específicos de medicina e histología}

\textbf{Herramientas de diagnóstico asistido por computadora (Computer Aided Diagnosis)}: \textit{abrev. CAD}. Son sistemas que pretenden asistir a los médicos de distintas disciplinas proveyéndoles de una \enquote{segunda perspectiva} de diagnóstico \cite{defCAD}.

\textbf{Imágenes de cortes histológicos completos (Whole Slide Image)}: \textit{abrev. WSI}. Se trata de una imagen digitalizada de un corte histológico microscópico, de suma utilidad en histoquímica y citoquímica. Se ha demostrado que su uso en diagnóstico patológico no es inferior al diagnóstico basado en microscopios \cite{defWSI}.


  %%%%%%%%% SECTION %%%%%%%%%
\subsection{PRISMA: Lista de verificación} \label{appendix:es/prisma_checklist}

La lista de verificación usada para la revisión sistemática se presenta en la Tabla \ref{tabla:es/prisma_checklist}. La tabla está basada en la traducción al español, publicada por Page et al. \cite{PRISMAespañol}.

{
\scriptsize
\begin{longtable}[h]{M{1.8cm}|M{0.7cm}|M{8.7cm}|M{2cm}}

\hline
    \textbf{Sección o Tema} & \textbf{Item} & \textbf{Ítem de la lista de verificación} & \textbf{Localización en la publicación} \\
    \hline \hline
    \endhead
\hline
    \caption{Lista de verificación de metodología PRISMA. \label{tabla:es/prisma_checklist}}
    \endfirstfoot
\hline
    \caption[]{(Continuación) Lista de verificación de metodología PRISMA.}
    \endfoot
\multicolumn{4}{l}{\textbf{Título}} \\
\hline
Título & 1 & Identifique la publicación como una revisión sistemática. & Título \\
\hline
\hline
\multicolumn{4}{l}{\textbf{Resumen}} \\
\hline
Resumen estructurado & 2 &  Vea la lista de verificación para resúmenes estructurados de la declaración PRISMA 2020. & Introducción \\
\hline
\hline
\multicolumn{4}{l}{\textbf{Introducción}} \\
\hline
Justificación & 3 &Describa la justificación de la revisión en el contexto del conocimiento existente.& Sección \ref{section:es/introduccion}\\
\hline
Objetivos & 4 & Proporcione una declaración explícita de los objetivos o las preguntas que aborda la revisión. & Sección \ref{section:es/introduccion}\\
\hline
\hline
\multicolumn{4}{l}{\textbf{Métodos}} \\
\hline
Criterios de elegibilidad &5&  Especifique los criterios de inclusión y exclusión de la revisión y cómo se agruparon los estudios para la síntesis. & Sección \ref{section:es/criterios_inc_exc} \\
\hline
Fuentes de información &6&  Especifique todas las bases de datos, registros, sitios web, organizaciones, listas de referencias y otros recursos de búsqueda o consulta para identificar los estudios. Especifique la fecha en la que cada recurso se buscó o consultó por última vez. & Sección \ref{section:es/cadena_busqueda} \\
\hline
Estrategia de búsqueda & 7&  Presente las estrategias de búsqueda completas de todas las bases de datos, registros y sitios web, incluyendo cualquier filtro y los límites utilizados. & Sección \ref{section:es/cadena_busqueda} \\
\hline
Proceso de selección de los estudios &8 & Especifique los métodos utilizados para decidir si un estudio cumple con los criterios de inclusión de la revisión, incluyendo cuántos autores de la revisión cribaron cada registro y cada publicación recuperada, si trabajaron de manera independiente y, si procede, los detalles de las herramientas de automatización utilizadas en el proceso. & Sección \ref{section:es/proceso_seleccion} \\
\hline
Proceso de extracción de los datos & 9 & Indique los métodos utilizados para extraer los datos de los informes o publicaciones, incluyendo cuántos revisores recopilaron datos de cada publicación, si trabajaron de manera independiente, los procesos para obtener o confirmar los datos por parte de los investigadores del estudio y, si procede, los detalles de las herramientas de automatización utilizadas en el proceso. & Sección \ref{section:es/proceso_seleccion} \\
\hline
Lista de los datos &10a & Enumere y defina todos los desenlaces para los que se buscaron los datos. Especifique si se buscaron todos los resultados compatibles con cada dominio del desenlace (por ejemplo, para todas las escalas de medida, puntos temporales, análisis) y, de no ser así, los métodos utilizados para decidir los resultados que se debían recoger. & Sección \ref{section:es/proceso_seleccion} \\
\cline{2-4} 
 &10b & Enumere y defina todas las demás variables para las que se buscaron datos (por ejemplo, características de los participantes y de la intervención, fuentes de financiación). Describa todos los supuestos formulados sobre cualquier información ausente (missing) o incierta. & Sección \ref{section:es/proceso_seleccion} \\
\hline
Evaluación del riesgo de sesgo de los estudios individuales & 11 & Especifique los métodos utilizados para evaluar el riesgo de sesgo de los estudios incluidos, incluyendo detalles de las herramientas utilizadas, cuántos autores de la revisión evaluaron cada estudio y si trabajaron de manera independiente y, si procede, los detalles de las herramientas de automatización utilizadas en el proceso. & Sección \ref{section:es/calidad} \\
\hline
Medidas del efecto & 12 & Especifique, para cada desenlace, las medidas del efecto (por ejemplo, razón de riesgos, diferencia de medias) utilizadas en la síntesis o presentación de los resultados. & No aplica \\
\hline
Métodos de síntesis &13a & Describa el proceso utilizado para decidir qué estudios eran elegibles para cada síntesis (por ejemplo, tabulando las características de los estudios de intervención y comparándolas con los grupos previstos para cada síntesis (ítem 5). &  Sección \ref{section:es/proceso_seleccion} \\
\cline{2-4} 
 &13b & Describa cualquier método requerido para preparar los datos para su presentación o síntesis, tales como el manejo de los datos perdidos en los estadísticos de resumen o las conversiones de datos. &  Sección \ref{section:es/proceso_seleccion} \\
\cline{2-4} 
 & 13c & Describa los métodos utilizados para tabular o presentar visualmente los resultados de los estudios individuales y su síntesis. &  Secciones \ref{section:es/proceso_seleccion} y \ref{section:es/resultados} \\
\cline{2-4} 
 & 13d & Describa los métodos utilizados para sintetizar los resultados y justifique sus elecciones. Si se ha realizado un metanálisis, describa los modelos, los métodos para identificar la presencia y el alcance de la heterogeneidad estadística, y los programas informáticos utilizados. & Sección \ref{section:es/proceso_seleccion} \\
\cline{2-4} 
 & 13e & Describa los métodos utilizados para explorar las posibles causas de heterogeneidad entre los resultados de los estudios (por ejemplo, análisis de subgrupos, metarregresión). & Secciones \ref{section:es/cadena_busqueda} y \ref{section:es/proceso_seleccion} \\
\cline{2-4} 
 & 13f & Describa los análisis de sensibilidad que se hayan realizado para evaluar la robustez de los resultados de la síntesis. & No aplica \\
\hline
Evaluación del sesgo en la publicación & 14 & Describa los métodos utilizados para evaluar el riesgo de sesgo debido a resultados faltantes en una síntesis (derivados de los sesgos en las publicaciones). & Sección \ref{section:es/calidad} \\
\hline
Evaluación de la certeza de la evidencia & 15 & Describa los métodos utilizados para evaluar la certeza (o confianza) en el cuerpo de la evidencia para cada desenlace. & No aplica \\
\hline
Selección de los estudios & 16a & Describa los resultados de los procesos de búsqueda y selección, desde el número de registros identificados en la búsqueda hasta el número de estudios incluidos en la revisión, idealmente utilizando un diagrama de flujo. & Sección \ref{section:es/resultados} y Figura \ref{figura:es/data_extraction} \\
\cline{2-4} 
 & 16b & Cite los estudios que aparentemente cumplían con los criterios de inclusión, pero que fueron excluidos, y explique por qué fueron excluidos. & Sección \ref{section:es/resultados} y Figura \ref{figura:es/data_extraction} \\
\hline
Características de los estudios & 17 & Cite cada estudio incluido y presente sus características. & Sección \ref{section:es/resultados} y Tabla \ref{tabla:es/calidad} \\
\hline
Riesgo de sesgo de los estudios individuales & 18 & Presente las evaluaciones del riesgo de sesgo para cada uno de los estudios incluidos. & Sección \ref{section:es/calidad} \\
\hline
Resultados de los estudios individuales & 19 & Presente, para todos los desenlaces y para cada estudio: a) los estadísticos de resumen para cada grupo (si procede) y b) la estimación del efecto y suprecisión (por ejemplo, intervalo de credibilidad o de confianza), idealmente utilizando tablas estructuradas o gráficos. & Sección \ref{section:es/resultados} \\
\hline
Resultados de la síntesis & 20a &  Para cada síntesis, resuma brevemente las características y el riesgo de sesgo entre los estudios contribuyentes. & Sección \ref{section:es/resultados} \\
\cline{2-4} 
 & 20b & Presente los resultados de todas las síntesis estadísticas realizadas. Si se ha realizado un metanálisis, presente para cada uno de ellos el estimador de resumen y su precisión (por ejemplo, intervalo de credibilidad o de confianza) y las medidas de heterogeneidad estadística. Si se comparan grupos, describa la dirección del efecto. & No aplica \\
\cline{2-4} 
 & 20c & Presente los resultados de todas las investigaciones sobre las posibles causas de heterogeneidad entre los resultados de los estudios. & No aplica \\
\cline{2-4} 
 & 20d & Presente los resultados de todos los análisis de sensibilidad realizados para evaluar la robustez de los resultados sintetizados. & No incluido \\
\hline
Sesgos en la publicación & 21 & Presente las evaluaciones del riesgo de sesgo debido a resultados faltantes (derivados de los sesgos de en las publicaciones) para cada síntesis evaluada. & No incluido \\
\hline
Certeza de la evidencia & 22 & Presente las evaluaciones de la certeza (o confianza) en el cuerpo de la evidencia para cada desenlace evaluado. & Sección \ref{section:es/resultados} \\
\hline
\hline
\multicolumn{4}{l}{\textbf{Discusión}} \\
\hline
Discusión & 23a & Proporcione una interpretación general de los resultados en el contexto de otras evidencias. & Sección \ref{section:es/conclusiones} \\
\cline{2-4} 
 & 23b & Argumente las limitaciones de la evidencia incluida en la revisión. & Sección \ref{section:es/conclusiones} \\
\cline{2-4} 
 & 23c & Argumente las limitaciones de los procesos de revisión utilizados. & No incluido \\
\cline{2-4} 
 & 23d & Argumente las implicaciones de los resultados para la práctica, las políticas y las futuras investigaciones. & Sección \ref{section:es/conclusiones} \\
\hline
\hline
\multicolumn{4}{l}{\textbf{Otra información}} \\
\hline
Registro y protocolo & 24a & Proporcione la información del registro de la revisión, incluyendo el nombre y el número de registro, o declare que la revisión no ha sido registrada. & Revisión no registrada \\
\cline{2-4} 
 & 24b & Indique dónde se puede acceder al protocolo, o declare que no se ha redactado ningún protocolo. & Sección \ref{section:es/protocolo} \\
\cline{2-4} 
 & 24c & Describa y explique cualquier enmienda a la información proporcionada en el registro o en el protocolo. & No aplica \\
\hline
Financiamiento & 25 & Describa las fuentes de apoyo financiero o no financiero para la revisión y el papel de los financiadores o patrocinadores en la revisión. & No aplica \\
\hline
Conflicto de intereses & 26 & Declare los conflictos de intereses de los autores de la revisión. & No declara \\
\hline
Disponibilidad de datos, códigos y otros materiales & 27 & Especifique qué elementos de los que se indican a continuación están disponibles al público y dónde se pueden encontrar: plantillas de formularios de extracción de datos, datos extraídos de los estudios incluidos, datos utilizados para todos los análisis, código de análisis, cualquier otro material utilizado en la revisión. & Sección \ref{section:es/resultados}, Apéndices y Referencias

\end{longtable}
}
  %%%%%%%%% SECTION %%%%%%%%%
\subsection{Anexo PI2: Datos de recuento} \label{appendix:es/rpi2_datos}

Los datos utilizados para la síntesis de resultados a la Pregunta de Investigación 2 se presentan en la Tabla \ref{tabla:es/rpi2_datos}

%\begin{longtable}{|| m{8em} m{5em} m{19em} ||}
{
\scriptsize
\begin{longtable}{|| p{.20\textwidth} p{.16\textwidth} p{.54\textwidth} ||}

\hline
    \textbf{Autores del Estudio Primario}		& \textbf{Postura reportada}	& \textbf{Cita textual que lo fundamenta} \\ [0.5ex]
    \hline \hline
    \endhead
% \hline \hline 
    \caption{Posturas reportadas por los autores de los estudios primarios. \label{tabla:es/rpi2_datos}}
    \endfirstfoot
% \hline \hline 
    \caption[]{(Continuación) Posturas reportadas por los autores de los estudios primarios.}
    \endfoot
Verma et al. 						& Colaborativo		& We asked the interviewees about their perceptions regarding the current role and impact of AI in cancer care, and how it might evolve in the future. In particular, how AI-powered technologies will shape their work practices and their relationship with patients \cite{Verma2023}. \\
 \hline
Di Giammarco et al.					& Colaborativo		& The use of three distinguished CAMs, i.e. Grad-CAM, Score-CAM, and FastScore- CAM, united to the index similarity; improves the reliability and the trustworthiness of AI in healthcare. This indicates that while deep learning prediction does not replace human decisions, it does aid in the consultation process during the diagnostic procedure \cite{DiGiammarco2024}. \\
 \hline
Bouderhem et al.					& Colaborativo		& It is necessary to look beyond the hype and assess the pros and cons of AI in healthcare today \cite{Bouderhem2024}. \\
 \hline
Pahud de Mortanges et al.			& Colaborativo		& LLM-based technologies could enable a bidirectional “dialogue” between users and (X)AI systems. In the more or less distant future, such systems may serve as a virtual assistant capable of working as a counselor in clinical scenarios \cite{PahuddeMortanges2024}. \\
 \hline
Lee et al.							& Colaborativo		& Adopting digital pathology can enable pathologists to improve their workflow efficiency, accessibility, and quantitative analysis, advancing biomedical knowledge and research \cite{Lee2024}. \\
 \hline
Lee et al.							& No colaborativo	& (...) we use metrics to compare the performances of explainable AI methods instead of relying on human evaluations \cite{Lee2024}. \\
 \hline
Wang et al.							& No colaborativo	& This results in overworked pathologists, which can lead to higher chances of deficiencies in their routine work and dys- functions of the pathology laboratories with more laboratory errors. (...) It is thus imperative to develop reliable tools for pathological image analysis and CRC detection that can improve clinical efficiency and efficacy without unin- tended human bias during diagnosis \cite{Wang2021}. \\
 \hline
Rösler et al.						& Colaborativo		& As populations age and cancer become more prevalent, more trained personnel are required to care for cancer patients. AI can potentially help to train these experts, although this aspect is still an emerging feld in hematology and oncology \cite{Roesler2023}. \\
 \hline
Zehra et al.						& Colaborativo		& Digital pathology and AI can only be added to routine histopathology workflow. Conventional histopathology cannot be substituted \cite{Zehra2023}. \\
 \hline
Guleria et al.						& No colaborativo	& We suggest deploying deep-learning-based image recognition models, which ofer the potential to improve accuracy of pCLE and histopathology and to recognize patterns that may have eluded human visual analysis. (...) Our work indirectly suggests that these models may achieve similar accuracy as human interpreters, paving the way frst for more direct comparisons of humans and deep learning models and then for increased clinical application of these technologies \cite{Guleria2021}. \\
 \hline
Finzel et al.						& Colaborativo		& They (GNN) provide the means toward transparency and, ideally, could be used in combination with symbolic approaches to generate verbal or language-like explana- tions for increased comprehensibility and better control of the overall system to satisfy the validation and improvement requirements of human-centric AI \cite{Finzel2022}. \\
 \hline
Mohammadi et al.					& Colaborativo		& Modern artificial intelligence (AI)-based systems have the potential to handle vast amounts of data, far in excess of what humans are capable of, and as such have huge potential to assist pathologists in their diagnostic work and allay these pressures \cite{Mohammadi2022}. \\
 \hline
Holzinger et al.					& Colaborativo		& Such (ML) systems are not able to understandthe context, hence cannot reason about interventions and retrospection. However, such approaches needs the guidance of ahuman model similar to the ones used in causality research to answer the question “Why?” \cite{Holzinger2019}. \\
 \hline
Yin et al.							& Colaborativo		& Although the inclusion of pathologists (i.e., human-in-the-loop) in the model development process is very important, there is a need to go beyond interpretable machine learning. To reach a level supporting the pathologists in their daily decision making, another factor that should be taken into account is causability [33], which is measured in terms of effectiveness, efficiency, satisfaction related to causal understanding and its transparency for a user. In other words, it refers to a human understandable model \cite{Yin2020}. \\
 \hline
Sabol et al.						& Colaborativo		& A long list of machine learning approaches to image classification and whole-slide segmentation has been developed to support pathologist in interpreting histopathological images \cite{Sabol2020}. \\
 \hline
Cai et al.							& Colaborativo		& (...) we examine what types of information end-users desire to know about an AI Assistant during onboarding, and relate these needs to the existing medical practices of competence articulation and the seeking of input and second opinions from colleagues \cite{Cai2019}. \\
 \hline
Cai et al.							& No colaborativo	& Within this space, a significant portion of recent research has focused on demonstrating that these models can rival the accuracy of medical experts \cite{Cai2019}. \\
 \hline
Tjoa y Guan							& Colaborativo		& For now, if “interpretable” algorithms are deployed in medical practices, human supervision is still necessary. Interpretability information should be considered nothing more than complementary support for the medical practices before there is a robust way to handle interpretability \cite{Tjoa&Guan2021}. \\
 \hline
Ayorinde et al.						& Colaborativo		& To effectively appraise research and new market tools, clinicians need an awareness of the common problems that arise in AI training and deployment. In parallel, teams developing AI image classiffiers need continual dialogue with clinicians that is centered on the trade-offs that their chosen designs entail; to secure trust, and facilitate a smooth translation to the clinic \cite{Ayorinde2022}. \\
 \hline
Palatnik de Sousa et al.			& Colaborativo		& (...) A pathologist using this XAI methodology as a decision support tool could then apply this approach on every patch where deemed necessary \cite{PalatnikdeSousa2019}. \\
 \hline
Gu et al.							& Colaborativo		& In this work, instead of employing AI to replace pathologists, we adapt AI closely to doctors’ domain knowledge of navigation, enabling them to work collaboratively with AI. Our validation study shows that our human + AI approach is recognized to have a better workflow integration and can help pathologists achieve higher precision and recall on average compared to start-of-the-art AI \cite{Gu2023NaviPath}. \\
 \hline
Gu et al.							& Colaborativo		& (Pathologists) examine its results and evidence accordingly in an explainable manner, and examine the evidence to update the suggested diagnosis \cite{Gu2023XPath}. \\
 \hline
Tschandl et al.						& Colaborativo		& AI-based triage and decision support could assist readers in managing workloads and expanding their performance \cite{Tschandl2020}. \\
 \hline
Tschandl et al.						& No colaborativo	& Most research to date has been predicated on head-to-head comparisons of the diagnostic accuracy of AI-based systems with that of humans \cite{Tschandl2020}. \\
 \hline
Gallo et al.						& No colaborativo	& (...) the authors demonstrate an AI solution that performs at a level equal to pathologists. In some cases, such as detecting Gleason pattern four types of prostate cancer, it even surpasses the pathologists’ detection rate \cite{Gallo2023}. \\
 \hline
Dolezal et al.						& Colaborativo		& Explaining how a model has reached its decision and the level of certainty associated with a prediction can help build clinician trust, which may help foster greater adoption of these tools into clinical practice. Soft- ware that seamlessly integrates explainability and uncertainty quantifcation presents a signifcant advantage in promoting the potential clinical utility of these deep learning tools \cite{Dolezal2024}. \\
 \hline
Vanitha et al.						& Colaborativo		& This step forward (Grad-CAM) is pivotal for the adoption of AI in routine clinical practices, ensuring that AI-sup- ported diagnostics are both interpretable and verifable by expert clinicians \cite{Vanitha2024}. \\
 \hline
Dörrich et al.						& Colaborativo		& Explainable AI techniques help both developers and physicians to better understand AI algorithms, their abilities, and their limitations \cite{Doerrich2023}. \\
 \hline
Tosun et al.						& Colaborativo		& It is critical to emphasize that the purpose of the xAI is not to make a diagnosis independent from the pathologist, but to assist the pathologist in being more accurate and efficient \cite{Tosun2020}. \\
 \hline
Jarrahi et al.						& Colaborativo		& The interoperability challenges lead to accountability issues in high-stake decisions in pathology as human experts (i.e., pathologists) are deemed irreplaceable and must actively participate in decision making. As others noted, historically “human–machine collaborations have performed better than either one alone” in these contexts, and such a partnership requires opening the black box of AI and making results transparent and explainable for different stakeholders \cite{Jarrahi2022}. \\
 \hline
Shahamatdar et al.					& No colaborativo	& DNNs rival expert pathologists at detecting malignant skin lesions, diagnosing diabetic retinopathy and detecting breast cancer. In each of these cases, DNNs learned to solve straightforward but time-consuming tasks that are already within the expert physician’s repertoire. However, there is now a growing number of reports that DNNs can also learn to solve tasks posed on histopathological images that are difficult or impossible for pathologists to do by visual analysis alone \cite{Shahamatdar2024}. \\
 \hline
Praetorius et al.					& No colaborativo	& We envisage IMFSegNet to be a step towards making conventional ordinal-scaled and subjective characterization by a pathologist obsolete in the future. In fact, while artificial intelligence may take over the routine work, the future role of pathologists may be more supervising the automated process \cite{Praetorius2023}. \\
 \hline
Kiehl et al.						& Colaborativo		& Human pathologists are also needed to detect rare pathologies that the algorithms have not been trained on. For the near future, it seems more likely that pathologists using AI will replace those not using it, creating a combination of human and machine intelligence (augmented intelligence). AI-based diagnostic support systems can help avoid decision fatigue and enable the pathologist to do more work. For the longer term, however, emerging technologies will bring significant changes in workflow. These changes will enable machines to perform diagnostic tasks in ways that are very different from how human pathologists operate \cite{Kiehl2022}. \\
 \hline
Kiehl et al.						& No colaborativo	& Will AI replace pathologists? There are some concerns in the specialty about the possible elimination of jobs. The threat of automation may discourage potential trainees from choosing the specialty, thus increasing the growing shortage of pathologists \cite{Kiehl2022}.  \\
 \hline
Ullah et al.						& Colaborativo		& The introduction of LLMs in diagnostic medicine may lead to concerns regarding the professional autonomy and decision-making abilities of healthcare professionals. There is a risk of over-reliance on the model’s suggestions, potentially diminishing critical thinking and independent clinical judgment. Care must be taken to ensure that LLMs serve as a valuable tool to augment the expertise of healthcare professionals rather than replacing their crucial role in the diagnostic process \cite{Ullah2024}. \\
 \hline
Nasir et al.						& Colaborativo		& Risks involve potential misinformation, over-reliance on AI, ethical and legal concerns, and the absence of emotional support. It should be used cautiously to complement human expertise rather than replace it entirely \cite{Nasir2024}. \\
 \hline
Finzel et al.						& Colaborativo		& We believe that providing more human-centered and integrative explanation frameworks will pave the way to beneficial AI transparency and human understanding of and trust in AI \cite{Finzel2024}. \\
\hline

\end{longtable}
}
  %%%%%%%%% SECTION %%%%%%%%%
\subsection{Anexo PI3: Datos de recuento} \label{appendix:es/rpi3_datos}

Los datos utilizados para la síntesis de resultados a la pregunta de investigación 3 se presentan en la Tabla \ref{tabla:es/rpi3_datos}.

{
\scriptsize
\begin{longtable}{|| p{.07\textwidth} p{.35\textwidth} p{.15\textwidth} p{.23\textwidth} ||}

\hline
    \textbf{Dominio}		&		\textbf{Grupo}		&		\textbf{Tipo}		&		\textbf{Artículo en que aparece} \\ [0.5ex]
    \hline \hline
    \endhead
% \hline \hline 
    \caption{Algoritmos, técnicas y métodos reportados por los autores de los estudios primarios. \label{tabla:es/rpi3_datos}}
    \endfirstfoot
% \hline \hline 
    \caption[]{(Continuación) Algoritmos, técnicas y métodos reportados por los autores de los estudios primarios.}
    \endfoot

IA	&	Convolutional Neural Network (CNN)	&	No especifica	&	Shaban-Nejad et al. \cite{ShabanNejad2021} \\ \hline
IA	&	Convolutional Neural Network (CNN)	&	3D CNN	&	Shaban-Nejad et al. \cite{ShabanNejad2021} \\ \hline
IA	&	Decision Trees	&	XGBoost	&	Alabi et al. \cite{Alabi2023} \\ \hline
IAX	&	Local Interpretable Model-Agnostic Explanations (LIME)	&	LIME	&	Alabi et al. \cite{Alabi2023} \\ \hline
IAX	&	SHapley Additive exPlanation (SHAP)	&	SHAP	&	Alabi et al. \cite{Alabi2023} \\ \hline
IAX	&	Class Activation Mapping (CAM)	&	CAM	&	Di Giammarco et al. \cite{DiGiammarco2024} \\ \hline
IAX	&	Class Activation Mapping (CAM)	&	Grad-CAM	&	Di Giammarco et al. \cite{DiGiammarco2024} \\ \hline
IAX	&	Class Activation Mapping (CAM)	&	Score-CAM	&	Di Giammarco et al. \cite{DiGiammarco2024} \\ \hline
IAX	&	Class Activation Mapping (CAM)	&	FastScore-CAM	&	Di Giammarco et al. \cite{DiGiammarco2024} \\ \hline
IA	&	Convolutional Neural Network (CNN)	&	MobileNet	&	Di Giammarco et al. \cite{DiGiammarco2024} \\ \hline
IA	&	Convolutional Neural Network (CNN)	&	Inception-v3	&	Di Giammarco et al. \cite{DiGiammarco2024} \\ \hline
IA	&	EfficientNet	&	EfficientNet	&	Di Giammarco et al. \cite{DiGiammarco2024} \\ \hline
IA	&	Framework	&	ChatCAD	&	Pahud de Mortanges et al. \cite{PahuddeMortanges2024} \\ \hline
IA	&	Framework	&	GLoRIA	&	Pahud de Mortanges et al. \cite{PahuddeMortanges2024} \\ \hline
IA	&	Feature Pyramid Network (FPN)	&	FPN	&	Lee et al. \cite{Lee2024} \\ \hline
IA	&	Feature Pyramid Network (FPN)	&	SN-FPN	&	Lee et al. \cite{Lee2024} \\ \hline
IA	&	Region-based Convolutional Neural Network (R-CNN)	&	R-CNN	&	Lee et al. \cite{Lee2024} \\ \hline
IA	&	Framework	&	RetinaNet	&	Lee et al. \cite{Lee2024} \\ \hline
IA	&	Region-based Convolutional Neural Network (R-CNN)	&	Cascade R-CNN	&	Lee et al. \cite{Lee2024} \\ \hline
IA	&	Region-based Convolutional Neural Network (R-CNN)	&	Faster-RCNN	&	Lee et al. \cite{Lee2024} \\ \hline
IA	&	Feature Pyramid Network (FPN)	&	NAS-FPN	&	Lee et al. \cite{Lee2024} \\ \hline
IA	&	Convolutional Neural Network (CNN)	&	HRNet	&	Lee et al. \cite{Lee2024} \\ \hline
IA	&	Framework	&	PANet	&	Lee et al. \cite{Lee2024} \\ \hline
IA	&	Feature Pyramid Network (FPN)	&	A2-FPN	&	Lee et al. \cite{Lee2024} \\ \hline
IA	&	Feature Pyramid Network (FPN)	&	MA-FPN	&	Lee et al. \cite{Lee2024} \\ \hline
IA	&	Feature Pyramid Network (FPN)	&	PAC-Net	&	Lee et al. \cite{Lee2024} \\ \hline
IA	&	Convolutional Neural Network (CNN)	&	DeepLabV3Plus	&	Lee et al. \cite{Lee2024} \\ \hline
IA	&	Framework	&	UNet++	&	Lee et al. \cite{Lee2024} \\ \hline
IA	&	Framework	&	LinkNet	&	Lee et al. \cite{Lee2024} \\ \hline
IA	&	Convolutional Neural Network (CNN)	&	MA-Net	&	Lee et al. \cite{Lee2024} \\ \hline
IA	&	Framework	&	PAN	&	Lee et al. \cite{Lee2024} \\ \hline
IA	&	Convolutional Neural Network (CNN)	&	PSPNet	&	Lee et al. \cite{Lee2024} \\ \hline
IA	&	Feature Pyramid Network (FPN)	&	FPN	&	Lee et al. \cite{Lee2024} \\ \hline
IA	&	Residual Neural Network (ResNet)	&	ResNet50	&	Lee et al. \cite{Lee2024} \\ \hline
IA	&	Dense Convolutional Network (DenseNet)	&	DenseNet121	&	Tabatabaei et al. \cite{Tabatabaei2023} \\ \hline
IA	&	Framework	&	DenseRes-Unet	&	Tabatabaei et al. \cite{Tabatabaei2023} \\ \hline
IA	&	Generative Adversarial Networks (GANs)	&	AnoGAN	&	Tabatabaei et al. \cite{Tabatabaei2023} \\ \hline
IA	&	Convolutional Neural Network (CNN)	&	IRRCNN	&	Tabatabaei et al. \cite{Tabatabaei2023} \\ \hline
IA	&	Framework	&	MesoNet	&	Tran et al. \cite{Tran2021} \\ \hline
IA	&	Framework	&	PathME	&	Tran et al. \cite{Tran2021} \\ \hline
IAX	&	SHapley Additive exPlanation (SHAP)	&	SHAP	&	Tran et al. \cite{Tran2021} \\ \hline
IA	&	Framework	&	PAGE-Net	&	Tran et al. \cite{Tran2021} \\ \hline
IA	&	Framework	&	Image2TMB	&	Tran et al. \cite{Tran2021} \\ \hline
IA	&	Otras redes neuronales artificiales	&	HE2RNA	&	Tran et al. \cite{Tran2021} \\ \hline
IA	&	Dense Convolutional Network (DenseNet)	&	DenseNet161	&	Hauser et al. \cite{Hauser2022} \\ \hline
IA	&	Dense Convolutional Network (DenseNet)	&	DenseNet121	&	Hauser et al. \cite{Hauser2022} \\ \hline
IA	&	Residual Neural Network (ResNet)	&	ResNet50	&	Hauser et al. \cite{Hauser2022} \\ \hline
IA	&	Visual Geometry Group network (VGG) 	&	VGG16	&	Hauser et al. \cite{Hauser2022} \\ \hline
IA	&	Residual Neural Network (ResNet)	&	ResNet101	&	Hauser et al. \cite{Hauser2022} \\ \hline
IA	&	Residual Neural Network (ResNet)	&	ResNet18	&	Hauser et al. \cite{Hauser2022} \\ \hline
IA	&	Convolutional Neural Network (CNN)	&	AlexNet	&	Hauser et al. \cite{Hauser2022} \\ \hline
IA	&	Convolutional Neural Network (CNN)	&	Inception-v3	&	Hauser et al. \cite{Hauser2022} \\ \hline
IA	&	Convolutional Neural Network (CNN)	&	PNASNet-5	&	Hauser et al. \cite{Hauser2022} \\ \hline
IA	&	Squeeze-and-Excitation network	&	SENet154	&	Hauser et al. \cite{Hauser2022} \\ \hline
IA	&	Visual Geometry Group network (VGG) 	&	VGG19	&	Hauser et al. \cite{Hauser2022} \\ \hline
IA	&	Convolutional Neural Network (CNN)	&	Inception-v4	&	Hauser et al. \cite{Hauser2022} \\ \hline
IA	&	EfficientNet	&	EfficientNetB0	&	Hauser et al. \cite{Hauser2022} \\ \hline
IA	&	Extreme Inception (Xception)	&	Xception	&	Hauser et al. \cite{Hauser2022} \\ \hline
IA	&	Residual Neural Network (ResNet)	&	ResNet34	&	Hauser et al. \cite{Hauser2022} \\ \hline
IA	&	Dense Convolutional Network (DenseNet)	&	DenseNet	&	Wang et al. \cite{Wang2021} \\ \hline
IA	&	Squeeze-and-Excitation network	&	SENet	&	Wang et et al. \cite{Wang2021} \\ \hline
IA	&	Residual Neural Network (ResNet)	&	ResNet152V2	&	Wang et et al. \cite{Wang2021} \\ \hline
IA	&	Dense Convolutional Network (DenseNet)	&	DenseNet201	&	Wang et et al. \cite{Wang2021} \\ \hline
IA	&	Convolutional Neural Network (CNN)	&	Inception-v3	&	Wang et et al. \cite{Wang2021} \\ \hline
IA	&	Otras herramientas	&	QuPath	&	Zehra et al. \cite{Zehra2023} \\ \hline
IA	&	Otras herramientas	&	ImageJ	&	Zehra et al. \cite{Zehra2023} \\ \hline
IA	&	Otras herramientas	&	Cytomine	&	Zehra et al. \cite{Zehra2023} \\ \hline
IA	&	Otras herramientas	&	Orbit	&	Zehra et al. \cite{Zehra2023} \\ \hline
IA	&	Otras herramientas	&	DEEPLIIF	&	Zehra et al. \cite{Zehra2023} \\ \hline
IAX	&	Class Activation Mapping (CAM)	&	Grad-CAM	&	Guleria et al. \cite{Guleria2021} \\ \hline
IA	&	Graph Neural Network (GNN)	&	GNN	&	Finzel et al. \cite{Finzel2022} \\ \hline
IA	&	Inductive Logic Programming (ILP)	&	ILP	&	Finzel et al. \cite{Finzel2022} \\ \hline
IAX	&	Interpretabilidad de GNNs	&	Lifted Neural Networks	&	Finzel et al. \cite{Finzel2022} \\ \hline
IAX	&	Interpretabilidad de GNNs	&	GNNExplainer	&	Finzel et al. \cite{Finzel2022} \\ \hline
IAX	&	Interpretabilidad de GNNs	&	PGExplainer	&	Finzel et al. \cite{Finzel2022} \\ \hline
IAX	&	Interpretabilidad de GNNs	&	GNN-LRP	&	Finzel et al. \cite{Finzel2022} \\ \hline
IAX	&	Interpretabilidad de GNNs	&	MotifExplainer	&	Finzel et al. \cite{Finzel2022} \\ \hline
IAX	&	Interpretabilidad de GNNs	&	GraphLIME	&	Finzel et al. \cite{Finzel2022} \\ \hline
IAX	&	Interpretabilidad de GNNs	&	PGMExplainer	&	Finzel et al. \cite{Finzel2022} \\ \hline
IAX	&	Interpretabilidad de GNNs	&	XGNN	&	Finzel et al. \cite{Finzel2022} \\ \hline
IA	&	Otras redes neuronales artificiales	&	RNN	&	Mohammadi et al. \cite{Mohammadi2022} \\ \hline
IA	&	Framework	&	CELNet	&	Mohammadi et al. \cite{Mohammadi2022} \\ \hline
IA	&	Framework	&	CLAM	&	Mohammadi et al. \cite{Mohammadi2022} \\ \hline
IA	&	Visual Geometry Group network (VGG) 	&	VGG	&	Yin et al. \cite{Yin2020} \\ \hline
IA	&	Otras herramientas	&	ImageJ	&	Yin et al. \cite{Yin2020} \\ \hline
IA	&	Otras herramientas	&	CellProfile	&	Yin et al. \cite{Yin2020} \\ \hline
IAX	&	Otras técnicas de interpretabilidad	&	CBR	&	Sabol et al. \cite{Sabol2020} \\ \hline
IAX	&	Otras técnicas de explicabilidad basadas en propagación	&	LRP	&	Sabol et al. \cite{Sabol2020} \\ \hline
IAX	&	Local Interpretable Model-Agnostic Explanations (LIME)	&	LIME	&	Sabol et al. \cite{Sabol2020} \\ \hline
IA	&	Otras redes neuronales artificiales	&	S-rRBF	&	Sabol et al. \cite{Sabol2020} \\ \hline
IA	&	Convolutional Neural Network (CNN)	&	ARA-CNN	&	Sabol et al. \cite{Sabol2020} \\ \hline
IA	&	Lógica difusa	&	CFCMC	&	Sabol et al. \cite{Sabol2020} \\ \hline
IA	&	Lógica difusa	&	X-CFCMC	&	Sabol et al. \cite{Sabol2020} \\ \hline
IA	&	Generative Adversarial Networks (GANs)	&	GAN	&	Shawi et al. \cite{Shawi2022} \\ \hline
IA	&	Generative Adversarial Networks (GANs)	&	DCGAN	&	Shawi et al. \cite{Shawi2022} \\ \hline
IA	&	Otras herramientas	&	NNI	&	Shawi et al. \cite{Shawi2022} \\ \hline
IAX	&	Otras técnicas de interpretabilidad	&	CAV	&	Shawi et al. \cite{Shawi2022} \\ \hline
IAX	&	Otras técnicas de interpretabilidad	&	RCV	&	Tjoa y Guan \cite{Tjoa&Guan2021} \\ \hline
IAX	&	Otras técnicas de interpretabilidad	&	TCAV	&	Tjoa y Guan \cite{Tjoa&Guan2021} \\ \hline
IAX	&	Local Interpretable Model-Agnostic Explanations (LIME)	&	LIME	&	Roscher et al. \cite{Roscher2020} \\ \hline
IAX	&	Otras técnicas de explicabilidad basadas en propagación	&	LRP	&	Roscher et al. \cite{Roscher2020} \\ \hline
IA	&	Framework	&	AttentiveChrome	&	Roscher et al. \cite{Roscher2020} \\ \hline
IA	&	Otras redes neuronales artificiales	&	RETAIN	&	Roscher et al. \cite{Roscher2020} \\ \hline
IAX	&	Otras técnicas de explicabilidad basadas en propagación	&	LRP	&	Schuhmacher et al. \cite{Schuhmacher2022} \\ \hline
IAX	&	Class Activation Mapping (CAM)	&	Grad-CAM	&	Schuhmacher et al. \cite{Schuhmacher2022} \\ \hline
IA	&	Convolutional Neural Network (CNN)	&	CompSegNet	&	Schuhmacher et al. \cite{Schuhmacher2022} \\ \hline
IA	&	Decision Trees	&	Random Forest	&	Nazar et al. \cite{Nazar2021} \\ \hline
IA	&	Convolutional Neural Network (CNN)	&	No especifica	&	Nazar et al. \cite{Nazar2021} \\ \hline
IA	&	Decision Trees	&	Decision Tree	&	Nazar et al. \cite{Nazar2021} \\ \hline
IA	&	Convolutional Neural Network (CNN)	&	No especifica	&	Palatnik de Sousa et al. \cite{PalatnikdeSousa2019} \\ \hline
IAX	&	Local Interpretable Model-Agnostic Explanations (LIME)	&	LIME	&	Palatnik de Sousa et al. \cite{PalatnikdeSousa2019} \\ \hline
IA	&	Framework	&	RetinaNet	&	Gu et al. \cite{Gu2023NaviPath} \\ \hline
IAX	&	Otras técnicas de interpretabilidad	&	Pixel-wise explanations	&	Sauter et al. \cite{Sauter2022} \\ \hline
IA	&	Generative Adversarial Networks (GANs)	&	GAN	&	Sauter et al. \cite{Sauter2022} \\ \hline
IA	&	Framework	&	SpRAy	&	Sauter et al. \cite{Sauter2022} \\ \hline
IAX	&	Otras técnicas de explicabilidad basadas en propagación	&	LRP	&	Sauter et al. \cite{Sauter2022} \\ \hline
IAX	&	Detección de sesgos	&	InsideBias	&	Sauter et al. \cite{Sauter2022} \\ \hline
IAX	&	Detección de sesgos	&	REVISE	&	Sauter et al. \cite{Sauter2022} \\ \hline
IAX	&	Local Interpretable Model-Agnostic Explanations (LIME)	&	LIME	&	Sauter et al. \cite{Sauter2022} \\ \hline
IAX	&	Class Activation Mapping (CAM)	&	Grad-CAM	&	Sauter et al. \cite{Sauter2022} \\ \hline
IAX	&	Otras técnicas de explicabilidad basadas en propagación	&	LRP	&	Sauter et al. \cite{Sauter2022} \\ \hline
IAX	&	Otras técnicas de interpretabilidad	&	MMD-critic	&	Sauter et al. \cite{Sauter2022} \\ \hline
IAX	&	Class Activation Mapping (CAM)	&	Grad-CAM	&	Sauter et al. \cite{Sauter2022} \\ \hline
IAX	&	SHapley Additive exPlanation (SHAP)	&	SHAP	&	Sauter et al. \cite{Sauter2022} \\ \hline
IAX	&	Otras técnicas de interpretabilidad	&	TCAV	&	Sauter et al. \cite{Sauter2022} \\ \hline
IAX	&	Class Activation Mapping (CAM)	&	Guided Grad-CAM	&	Sauter et al. \cite{Sauter2022} \\ \hline
IAX	&	Otras técnicas de interpretabilidad	&	RCV	&	Sauter et al. \cite{Sauter2022} \\ \hline
IA	&	Otras redes neuronales artificiales	&	SMILY	&	Sauter et al. \cite{Sauter2022} \\ \hline
IA	&	Convolutional Neural Network (CNN)	&	No especifica	&	Gu et al. \cite{Gu2023XPath} \\ \hline
IA	&	Otras redes neuronales artificiales	&	No especifica	&	Gu et al. \cite{Gu2023XPath} \\ \hline
IA	&	Otras redes neuronales artificiales	&	RNN	&	Gu et al. \cite{Gu2023XPath} \\ \hline
IA	&	Convolutional Neural Network (CNN)	&	No especifica	&	Gu et al. \cite{Gu2023XPath} \\ \hline
IA	&	Generative Adversarial Networks (GANs)	&	CycleGAN	&	Gu et al. \cite{Gu2023XPath} \\ \hline
IA	&	Framework	&	PathoVA	&	Gu et al. \cite{Gu2023XPath} \\ \hline
IA	&	Residual Neural Network (ResNet)	&	ResNet34	&	Tschandl et al. \cite{Tschandl2020} \\ \hline
IA	&	Visual Geometry Group network (VGG) 	&	VGG16	&	Gallo et al. \cite{Gallo2023} \\ \hline
IAX	&	Otras técnicas de explicabilidad basadas en gradiente	&	I*G	&	Gallo et al. \cite{Gallo2023} \\ \hline
IAX	&	Otras técnicas de explicabilidad basadas en gradiente	&	IG	&	Gallo et al. \cite{Gallo2023} \\ \hline
IAX	&	Otras técnicas de explicabilidad basadas en gradiente	&	EG	&	Gallo et al. \cite{Gallo2023} \\ \hline
IAX	&	Otras técnicas de explicabilidad basadas en gradiente	&	Grad-CAM	&	Gallo et al. \cite{Gallo2023} \\ \hline
IAX	&	Otras técnicas de explicabilidad basadas en gradiente	&	SmoothGrad	&	Gallo et al. \cite{Gallo2023} \\ \hline
IAX	&	Otras técnicas de explicabilidad basadas en gradiente	&	GB	&	Gallo et al. \cite{Gallo2023} \\ \hline
IAX	&	SHapley Additive exPlanation (SHAP)	&	SHAP	&	Gallo et al. \cite{Gallo2023} \\ \hline
IAX	&	Local Interpretable Model-Agnostic Explanations (LIME)	&	LIME	&	Gallo et al. \cite{Gallo2023} \\ \hline
IAX	&	Otras técnicas de explicabilidad basadas en perturbaciones	&	CXPlain	&	Gallo et al. \cite{Gallo2023} \\ \hline
IAX	&	Otras técnicas de explicabilidad basadas en perturbaciones	&	RISE	&	Gallo et al. \cite{Gallo2023} \\ \hline
IAX	&	Otras técnicas de explicabilidad basadas en perturbaciones	&	PDA	&	Gallo et al. \cite{Gallo2023} \\ \hline
IAX	&	Otras técnicas de explicabilidad basadas en perturbaciones	&	OSA	&	Gallo et al. \cite{Gallo2023} \\ \hline
IAX	&	Otras técnicas de explicabilidad basadas en perturbaciones	&	Anchors	&	Gallo et al. \cite{Gallo2023} \\ \hline
IAX	&	Otras técnicas de explicabilidad basadas en propagación	&	DTD	&	Gallo et al. \cite{Gallo2023} \\ \hline
IAX	&	Otras técnicas de explicabilidad basadas en propagación	&	LRP	&	Gallo et al. \cite{Gallo2023} \\ \hline
IAX	&	Otras técnicas de interpretabilidad	&	TCAV	&	Gallo et al. \cite{Gallo2023} \\ \hline
IAX	&	Otras técnicas de interpretabilidad	&	DeconvNet	&	Gallo et al. \cite{Gallo2023} \\ \hline
IAX	&	Otras técnicas de interpretabilidad	&	DeepLIFT	&	Gallo et al. \cite{Gallo2023} \\ \hline
IAX	&	Otras técnicas de interpretabilidad	&	NAM	&	Gallo et al. \cite{Gallo2023} \\ \hline
IAX	&	Interpretabilidad de GNNs	&	GNN-LRP	&	Gallo et al. \cite{Gallo2023} \\ \hline
IAX	&	Interpretabilidad de GNNs	&	GraphLIME	&	Gallo et al. \cite{Gallo2023} \\ \hline
IAX	&	Interpretabilidad de GNNs	&	RelEx	&	Gallo et al. \cite{Gallo2023} \\ \hline
IAX	&	Interpretabilidad de GNNs	&	GNNExplainer	&	Gallo et al. \cite{Gallo2023} \\ \hline
IAX	&	Interpretabilidad de GNNs	&	XGNN	&	Gallo et al. \cite{Gallo2023} \\ \hline
IAX	&	Interpretabilidad de GNNs	&	Sub-GraphX	&	Gallo et al. \cite{Gallo2023} \\ \hline
IAX	&	Interpretabilidad de GNNs	&	SE-GNN	&	Gallo et al. \cite{Gallo2023} \\ \hline
IAX	&	Interpretabilidad de GNNs	&	ProtGNN	&	Gallo et al. \cite{Gallo2023} \\ \hline
IA	&	Otras herramientas	&	TIAToolbox 	&	Dolezal et al. \cite{Dolezal2024} \\ \hline
IA	&	Otras herramientas	&	PathML	&	Dolezal et al. \cite{Dolezal2024} \\ \hline
IA	&	Otras herramientas	&	DeepPath	&	Dolezal et al. \cite{Dolezal2024} \\ \hline
IA	&	Convolutional Neural Network (CNN)	&	Inception-v3	&	Dolezal et al. \cite{Dolezal2024} \\ \hline
IA	&	Otras herramientas	&	Histolab	&	Dolezal et al. \cite{Dolezal2024} \\ \hline
IA	&	Framework	&	MONAI	&	Dolezal et al. \cite{Dolezal2024} \\ \hline
IA	&	Generative Adversarial Networks (GANs)	&	GAN	&	Dolezal et al. \cite{Dolezal2024} \\ \hline
IA	&	Otras herramientas	&	CTransPath	&	Dolezal et al. \cite{Dolezal2024} \\ \hline
IA	&	Otras herramientas	&	SimCLR	&	Dolezal et al. \cite{Dolezal2024} \\ \hline
IA	&	Generative Adversarial Networks (GANs)	&	StyleGAN2	&	Dolezal et al. \cite{Dolezal2024} \\ \hline
IA	&	Generative Adversarial Networks (GANs)	&	StyleGAN3	&	Dolezal et al. \cite{Dolezal2024} \\ \hline
IA	&	Generative Adversarial Networks (GANs)	&	conditional GAN	&	Dolezal et al. \cite{Dolezal2024} \\ \hline
IA	&	Convolutional Neural Network (CNN)	&	MobileNet	&	Vanitha et al. \cite{Vanitha2024} \\ \hline
IA	&	Extreme Inception (Xception)	&	Xception	&	Vanitha et al. \cite{Vanitha2024} \\ \hline
IAX	&	Class Activation Mapping (CAM)	&	Grad-CAM	&	Vanitha et al. \cite{Vanitha2024} \\ \hline
IA	&	Otras herramientas	&	PathML	&	Dörrich et al. \cite{Doerrich2023} \\ \hline
IA	&	EfficientNet	&	EfficientNetB0	&	Dörrich et al. \cite{Doerrich2023} \\ \hline
IAX	&	Class Activation Mapping (CAM)	&	No especifica	&	Dörrich et al. \cite{Doerrich2023} \\ \hline
IA	&	Decision Trees	&	Random Forest	&	Bellantuono et al. \cite{Bellantuono2023} \\ \hline
IA	&	Decision Trees	&	XGBoost	&	Bellantuono et al. \cite{Bellantuono2023} \\ \hline
IA	&	Otras herramientas	&	GNB	&	Bellantuono et al. \cite{Bellantuono2023} \\ \hline
IAX	&	SHapley Additive exPlanation (SHAP)	&	SHAP	&	Bellantuono et al. \cite{Bellantuono2023} \\ \hline
IA	&	Dense Convolutional Network (DenseNet)	&	DenseNet201	&	Shovon et al. \cite{Shovon2023} \\ \hline
IA	&	Extreme Inception (Xception)	&	Xception	&	Shovon et al. \cite{Shovon2023} \\ \hline
IAX	&	Class Activation Mapping (CAM)	&	Grad-CAM	&	Shovon et al. \cite{Shovon2023} \\ \hline
IAX	&	Class Activation Mapping (CAM)	&	Guided Grad-CAM	&	Shovon et al. \cite{Shovon2023} \\ \hline
IA	&	Otras redes neuronales artificiales	&	CNN + RNN	&	Shovon et al. \cite{Shovon2023} \\ \hline
IA	&	Convolutional Neural Network (CNN)	&	HAHNet	&	Shovon et al. \cite{Shovon2023} \\ \hline
IA	&	Generative Adversarial Networks (GANs)	&	DCGAN	&	Liang y Meng \cite{Liang&Meng2023} \\ \hline
IA	&	Visual Geometry Group network (VGG) 	&	VGG16	&	Liang y Meng \cite{Liang&Meng2023} \\ \hline
IA	&	Convolutional Neural Network (CNN)	&	DeTraC	&	Liang y Meng \cite{Liang&Meng2023} \\ \hline
IAX	&	Otras técnicas de interpretabilidad	&	DeconvNet	&	Liang y Meng \cite{Liang&Meng2023} \\ \hline
IAX	&	Otras técnicas de explicabilidad basadas en gradiente	&	No especifica	&	Liang y Meng \cite{Liang&Meng2023} \\ \hline
IAX	&	Otras técnicas de explicabilidad basadas en propagación	&	LRP	&	Liang y Meng \cite{Liang&Meng2023} \\ \hline
IAX	&	Class Activation Mapping (CAM)	&	Grad-CAM	&	Liang y Meng \cite{Liang&Meng2023} \\ \hline
IA	&	Visual Geometry Group network (VGG) 	&	VGG16	&	Liang y Meng \cite{Liang&Meng2023} \\ \hline
IA	&	Residual Neural Network (ResNet)	&	ResNet50	&	Liang y Meng \cite{Liang&Meng2023} \\ \hline
IA	&	Convolutional Neural Network (CNN)	&	Inception-v3	&	Liang y Meng \cite{Liang&Meng2023} \\ \hline
IA	&	Framework	&	Brea-Net	&	Liang y Meng \cite{Liang&Meng2023} \\ \hline
IAX	&	Otras técnicas de interpretabilidad	&	QLattice	&	Palkar et al. \cite{Palkar2024} \\ \hline
IAX	&	Otras técnicas de interpretabilidad	&	Eli5	&	Palkar et al. \cite{Palkar2024} \\ \hline
IAX	&	SHapley Additive exPlanation (SHAP)	&	SHAP	&	Palkar et al. \cite{Palkar2024} \\ \hline
IAX	&	Local Interpretable Model-Agnostic Explanations (LIME)	&	LIME	&	Palkar et al. \cite{Palkar2024} \\ \hline
IA	&	Otras redes neuronales artificiales	&	No especifica	&	Shahamatdar et al. \cite{Shahamatdar2024} \\ \hline
IA	&	Residual Neural Network (ResNet)	&	ResNet19	&	Shahamatdar et al. \cite{Shahamatdar2024} \\ \hline
IA	&	Residual Neural Network (ResNet)	&	ResNet34	&	Shahamatdar et al. \cite{Shahamatdar2024} \\ \hline
IA	&	Residual Neural Network (ResNet)	&	ResNet50	&	Shahamatdar et al. \cite{Shahamatdar2024} \\ \hline
IA	&	Convolutional Neural Network (CNN)	&	Shufflenet-v2	&	Shahamatdar et al. \cite{Shahamatdar2024} \\ \hline
IA	&	Convolutional Neural Network (CNN)	&	Inception-v3	&	Shahamatdar et al. \cite{Shahamatdar2024} \\ \hline
IAX	&	Class Activation Mapping (CAM)	&	Grad-CAM	&	Shahamatdar et al. \cite{Shahamatdar2024} \\ \hline
IA	&	Otras redes neuronales artificiales	&	IVNet	&	Aziz et al. \cite{Aziz2023} \\ \hline
IA	&	Convolutional Neural Network (CNN)	&	No especifica	&	Aziz et al. \cite{Aziz2023} \\ \hline
IA	&	Visual Geometry Group network (VGG) 	&	VGG16	&	Aziz et al. \cite{Aziz2023} \\ \hline
IA	&	Residual Neural Network (ResNet)	&	ResNet50	&	Aziz et al. \cite{Aziz2023} \\ \hline
IA	&	Convolutional Neural Network (CNN)	&	Inception-v3	&	Aziz et al. \cite{Aziz2023} \\ \hline
IA	&	Convolutional Neural Network (CNN)	&	MobileNetV3	&	Aziz et al. \cite{Aziz2023} \\ \hline
IA	&	EfficientNet	&	EfficientNetV3	&	Aziz et al. \cite{Aziz2023} \\ \hline
IAX	&	Local Interpretable Model-Agnostic Explanations (LIME)	&	LIME	&	Aziz et al. \cite{Aziz2023} \\ \hline
IA	&	Generative Adversarial Networks (GANs)	&	CycleGAN	&	Sloboda et al. \cite{Sloboda2024} \\ \hline
IA	&	Generative Adversarial Networks (GANs)	&	GAN	&	Sloboda et al. \cite{Sloboda2024} \\ \hline
IA	&	Generative Adversarial Networks (GANs)	&	DCGAN	&	Sloboda et al. \cite{Sloboda2024} \\ \hline
IA	&	Generative Adversarial Networks (GANs)	&	conditional CycleGAN	&	Sloboda et al. \cite{Sloboda2024} \\ \hline
IAX	&	Otras técnicas de interpretabilidad	&	Semantic Factorization (SeFa)	&	Sloboda et al. \cite{Sloboda2024} \\ \hline
IA	&	SegNet	&	IMFSegNet	&	Praetorius et al. \cite{Praetorius2023} \\ \hline
IA	&	SegNet	&	SegNet	&	Praetorius et al. \cite{Praetorius2023} \\ \hline
IA	&	Residual Neural Network (ResNet)	&	ResNet50	&	Alsubai et al. \cite{Alsubai2024} \\ \hline
IA	&	Convolutional Neural Network (CNN)	&	Darknet.19	&	Alsubai et al. \cite{Alsubai2024} \\ \hline
IA	&	Decision Trees	&	Random Forest	&	Alsubai et al. \cite{Alsubai2024} \\ \hline
IA	&	Generative Adversarial Networks (GANs)	&	GAN	&	Alsubai et al. \cite{Alsubai2024} \\ \hline
IA	&	Convolutional Neural Network (CNN)	&	Inception	&	Alsubai et al. \cite{Alsubai2024} \\ \hline
IA	&	Convolutional Neural Network (CNN)	&	MobileNet	&	Alsubai et al. \cite{Alsubai2024} \\ \hline
IA	&	EfficientNet	&	EfficientNetB4	&	Alsubai et al. \cite{Alsubai2024} \\ \hline
IA	&	Extreme Inception (Xception)	&	Xception	&	Alsubai et al. \cite{Alsubai2024} \\ \hline
IA	&	Residual Neural Network (ResNet)	&	ResNetV2	&	Alsubai et al. \cite{Alsubai2024} \\ \hline
IA	&	Visual Geometry Group network (VGG) 	&	VGG16	&	Amato et al. \cite{Amato2024} \\ \hline
IA	&	Residual Neural Network (ResNet)	&	ResNet18	&	Amato et al. \cite{Amato2024} \\ \hline
IA	&	Convolutional Neural Network (CNN)	&	MobileNet	&	Amato et al. \cite{Amato2024} \\ \hline
IAX	&	Class Activation Mapping (CAM)	&	Grad-CAM	&	Amato et al. \cite{Amato2024} \\ \hline
IA	&	Lógica difusa	&	CFCMC	&	Sabol et al. \cite{Sabol2019} \\ \hline
IA	&	Otras redes neuronales artificiales	&	SAE	&	Sabol et al. \cite{Sabol2019} \\ \hline
IA	&	Otras redes neuronales artificiales	&	No especifica	&	Sabol et al. \cite{Sabol2019} \\ \hline
IAX	&	Otras técnicas de explicabilidad basadas en propagación	&	LRP	&	Finzel et al. \cite{Finzel2024} \\ \hline
IAX	&	Local Interpretable Model-Agnostic Explanations (LIME)	&	LIME	&	Finzel et al. \cite{Finzel2024} \\ \hline
IAX	&	Class Activation Mapping (CAM)	&	Grad-CAM	&	Finzel et al. \cite{Finzel2024} \\ \hline
IA	&	Inductive Logic Programming (ILP)	&	ILP	&	Finzel et al. \cite{Finzel2024} \\ \hline
IA	&	Visual Geometry Group network (VGG) 	&	VGG	&	Finzel et al. \cite{Finzel2024} \\ \hline
IA	&	Residual Neural Network (ResNet)	&	ResNet	&	Finzel et al. \cite{Finzel2024} \\ \hline
IA	&	Framework	&	UV-Net	&	Dy et al. \cite{Dy2024} \\ \hline

\end{longtable}
}

%%%%%%%%%%%%%%%%%%%%%%%%%%%%%%%%%%%%%%%%%%%%%%%%%%%%%%
%   REFERENCIAS
%%%%%%%%%%%%%%%%%%%%%%%%%%%%%%%%%%%%%%%%%%%%%%%%%%%%%%
\newpage
\printbibliography[heading=bibintoc, title={Referencias}]

\end{document}
